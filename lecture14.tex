\documentclass[unknownkeysallowed,xcolor=table]{beamer}
 
\usepackage [T2A,T1] {fontenc}
\usepackage[utf8]{inputenc}
\usepackage[english,russian]{babel}
\usepackage{amsmath}
\usepackage{pifont}
\usepackage{listings}
\usepackage{url}
\usepackage{textcomp}
\usepackage{multirow}
\usepackage{tabulary}

\newcommand{\cmark}{\ding{51}}
\newcommand{\xmark}{\ding{55}}

% to not copy line numbers
\usepackage{accsupp}
% to not copy line numbers

% to not copy line numbers
\newcommand{\noncopynumber}[1]{%
  \BeginAccSupp{method=escape,ActualText={}}%
    \color{blue} #1%
  \EndAccSupp{}%
}
% to not copy line numbers

\setbeamertemplate{navigation symbols}{}

\newcommand{\textapprox}{\raisebox{0.5ex}{\texttildelow}}

\colorlet{mygreen}{green!60!blue}
\colorlet{mymauve}{red!60!blue}
\definecolor{pblue}{rgb}{0.1, 0.2, 0.8}

\newcommand{\rarr}{$\rightarrow$}

\lstset{
      basicstyle=\ttfamily\small,
      commentstyle=\color{mygreen},
      keywordstyle=\color{blue},
%      numberstyle=\tiny\color{blue},
      numberstyle=\noncopynumber,
      stringstyle=\color{mymauve},
      numbers=left,
      stepnumber=1,
      columns=fullflexible,
      breaklines=true,
      postbreak=\mbox{\textcolor{red}{\ensuremath{\hookrightarrow}\space}},
      literate={~} {\textapprox}{1},
      language={[11]C++}
}

\makeatletter
\newcommand{\srcmediumsize}{\@setfontsize{\srcmediumsize}{7pt}{7pt}}
\makeatother

\makeatletter
\newcommand{\srcbigsize}{\@setfontsize{\srcbigsize}{8pt}{8pt}}
\makeatother

\makeatletter
\newcommand{\srcsize}{\@setfontsize{\srcsize}{6pt}{6pt}}
\makeatother

\makeatletter
\newcommand{\srcsmallsize}{\@setfontsize{\srcsmallsize}{5pt}{5pt}}
\makeatother

\title[C++]
{Программирование на языке C++}
 
\subtitle{Вводный курс}
 
\author[А.~Б.~Морозов]
{
  \texorpdfstring{Александр Морозов\newline\href{mailto:gelu.speculum@gmail.com}{gelu.speculum@gmail.com}}
  {Александр Морозов}
}
  
\date[ITMO 2020]
{ИТМО, весенний семестр 2020}
 
\logo{%
  \makebox[0.97\paperwidth]{%
    \includegraphics[align=c,width=2cm,keepaspectratio]{itmo_logo.png}
    \hfill
    \includegraphics[align=c,width=1.5cm,keepaspectratio]{itiviti_logo.png}
  }
}

\AtBeginSection[]
{
  \begin{frame}
    \frametitle{Содержание}
    \tableofcontents[currentsection]
  \end{frame}
}

\begin{document}

\frame{\titlepage}

%-------------------------------------------------

\section{Шаблоны и параметры шаблонов}

\begin{frame}[fragile]{Шаблоны и параметры шаблонов}
  \lstinline{template <} \emph{список шаблонных параметров} \lstinline{>} \emph{объявление}

  \vspace{1em}

  Параметры шаблонов:
  \begin{itemize}
    \item типы \lstinline{class Id} или \lstinline{typename Id} \vspace{0.5em}
    \item шаблоны \lstinline{template <...> class Id} или \lstinline{template <...> typename Id} \vspace{0.5em}
    \item значения \lstinline{type Id} или \lstinline{auto Id}
  \end{itemize}

  \vspace{1em}

  Шаблонный идентификатор -- \emph{имя шаблона} \lstinline{< params list >}.
\end{frame}

\begin{frame}[fragile]{Пример шаблонов}
  \begin{lstlisting}
  template <class T>
  class Queue { ... }; // template

  Queue<int> q_int; // template-id

  template <std::size_t Size>
  class StackString { ... }; // template

  void f(const StackString<10> & str); // template-id
  \end{lstlisting}
\end{frame}

\begin{frame}[fragile]{Параметры шаблонов со значением по умолчанию}
  \begin{lstlisting}
  template <class = int>
  struct X {};

  template <std::size_t Size = 10>
  void f();

  template <template <class> class T = X>
  struct Y {};
  \end{lstlisting}
\end{frame}

\begin{frame}[fragile]{Переменное число шаблонных параметров}
  \begin{lstlisting}[basicstyle=\ttfamily\srcbigsize]
  template <class... Args, class T = char>
  void g(Args &&... args);

  template <class... Args>
  void f(Args &&... args)
  {
      g<Args...>(std::forward<Args>(args)...);
  }
  f();
  f(1, 0.5);

  template <bool... Flags>
  struct S {};
  S<> zero;
  S<false> one;
  S<false, true> two;

  template <template <class...> class... Templates>
  class C {};
  \end{lstlisting}
\end{frame}

\begin{frame}[fragile]{Раскрытие переменного параметра шаблона}
  \emph{нечто} \lstinline{...}

  \vspace{2em}

  \emph{нечто} должно включать в себя хотя бы один переменный параметр.
  Идентификатор переменного параметра должен встречаться там, где допустимо использовать идентификатор.

  \vspace{1em}

  \emph{нечто} \lstinline{...} \rarr \emph{нечто1}, \emph{нечто2}, ...

  \vspace{1em}

  При раскрытии переменный параметр в \emph{нечто} заменяется на очередное значение параметра.
  Если параметров было указано несколько, то все их наборы должны иметь одинаковую длину.
\end{frame}

\begin{frame}[fragile]{Примеры раскрытий переменных параметров}
  \begin{lstlisting}[basicstyle=\ttfamily\srcbigsize]
  template <class... Args>
  void f(Args &&... args)
  {
      g(std::forward<Args>(args)...);
  }

  f(); // g();
  f(1, 2, 3); // g(1, 2, 3);

  template <class... Xs> struct Zip
  {
      template <class... Ys>
      struct With
      {
          using type = std::tuple<std::pair<Xs, Ys>...>;
      };
  };

  using X = Zip<int, double>::With<char, std::string>::type;
  \end{lstlisting}
\end{frame}

\begin{frame}[fragile]{Точки возможного раскрытия переменных параметров}
  \begin{itemize}
    \item список аргументов функции
    \item список аргументов при вызове функции
    \item список аргументов в скобочной инициализации
    \item список параметров шаблонного идентификатора
    \item список параметров шаблона
    \item список наследования и список инициализации конструктора
    \item специальная форма оператора \lstinline{sizeof}
    \item операция свёртки
    \item \lstinline{using}-объявление
    \item список захвата лямбды
  \end{itemize}
\end{frame}

\begin{frame}[fragile]{Некоторые примеры раскрытия переменных параметров}
  \begin{lstlisting}[basicstyle=\ttfamily\srcsmallsize]
  template <int... Is>
  void f()
  {
      std::tuple<decltype(Is)...> t(Is...); // direct initialization
      std::initializer_list<int> i = { // list initialization
          Is...
      };
  }

  template<class... Ts, int... N>
  void g(Ts (&...arr)[N]); // function parameters

  g("Hello", "world");
  // g(const char (&)[5], const char (&)[5])

  template <class... T>
  struct S
  {
      std::tuple<T...> data; // template-id

      template <T... tt> // template parameters
      struct Nested {};
  };

  template <class... Bases>
  class Derived : Bases... // inheritance list
  {
  public:
      Derived(const Bases & ...bases)
        : Bases(bases)... // member initialization list
      {}

      using Bases::operator()...; // using declaration
  };

  template <class... T>
  constexpr std::size_t count = sizeof...(T); // special sizeof

  template <class... T>
  constexpr std::size_t sum_sizes()
  {
      return (sizeof(T) + ...); // fold expression
  }
  \end{lstlisting}
\end{frame}

\begin{frame}{Шаблонные параметры-типы}
  Имя параметра становится идентификатором типа внутри шаблона и ведёт себя, как псевдоним типа.

  \vspace{2em}

  При указании шаблонного параметра-типа это должен быть корректный идентификатор типа, в т.ч. неполного типа.
\end{frame}

\begin{frame}{Шаблонные параметры-значения}
  Допустимые типы:
  \begin{itemize}
    \item lvalue ссылка \vspace{0.3em}
    \item указатель \vspace{0.3em}
    \item интегральный тип \vspace{0.3em}
    \item \lstinline{enum} \vspace{0.3em}
    \item \lstinline{std::nullptr_t}
  \end{itemize}

  \vspace{1em}

  Значение при подстановке шаблонным параметром должно быть известно на этапе компиляции.
  Параметр-указатель или параметр-ссылка не могут быть связаны со строковым литералом, подобъектом или временным объектом.

  \vspace{1em}

  Внутри шаблона такой параметр -- константное значение.
\end{frame}

\begin{frame}[fragile]{Шаблонные параметры-шаблоны}
  Описание параметра имеет форму описания шаблона.

  \vspace{1em}

  Внутри шаблона такой параметр является шаблонным идентификатором.

  \begin{lstlisting}[basicstyle=\ttfamily\srcbigsize]
  template <template <class, bool> class X>
  struct S {};

  template <template <class K = void> class C>
  void f()
  {
      C<> c;
      C<int> cc;
  }

  template <class T>
  struct SS {};
  f<SS>();
  \end{lstlisting}
\end{frame}

\begin{frame}{Виды шаблонов}
  \begin{itemize}
    \item переменные \vspace{0.5em}
    \item функции \vspace{0.5em}
    \item классы \vspace{0.5em}
    \item псевдонимы типов
  \end{itemize}

  \vspace{1em}

  Член класса может быть шаблонным (статическое поле, метод, вложенный класс или псевдоним типа).
\end{frame}

\begin{frame}[fragile]{Примеры шаблонов переменных}
  \begin{lstlisting}[basicstyle=\ttfamily\srcbigsize]
  template <class T>
  T t_value = T();

  struct S
  {
      template <int I>
      static int x;
  };

  template <int I>
  int S::x = I;

  template <class T>
  constexpr int t_tag = 11;

  int main()
  {
      return t_value<int> + S::x<1> + t_tag<double>;
  }
  \end{lstlisting}
\end{frame}

\begin{frame}[fragile]{Примеры шаблонов функций}
  \begin{lstlisting}[basicstyle=\ttfamily\srcbigsize]
  template <class T>
  T max(const T & a, const T & b)
  {
      return a < b ? b : a;
  }

  template <class R, class T>
  R convert(const T & t)
  {
      return t;
  }

  int main()
  {
      char a = '\n', b = '\t';
      return convert<int>(max(a, b));
  }
  \end{lstlisting}
\end{frame}

\begin{frame}[fragile]{Примеры шаблонов классов}
  \begin{lstlisting}[basicstyle=\ttfamily\srcmediumsize]
  template <class A, bool B>
  struct S
  {
      A data;
      static A s_data;

      S();
      bool get() const;
      void set(const A & data_)
      { data = data_; }
  };

  template <class A, bool B>
  A S<A, B>::s_data = A();

  template <class A, bool B>
  S<A, B>::S()
  {}

  template <class A, bool B>
  bool S<A, B>::get() const
  { return B; }

  int main()
  {
      S<int, false> s;
      // s.set(11);
      return s.data + s.s_data;
  }
  \end{lstlisting}
\end{frame}

\begin{frame}[fragile]{Примеры шаблонов методов классов}
  \begin{lstlisting}[basicstyle=\ttfamily\srcmediumsize]
  struct S
  {
      template <class T>
      operator T ();
  };

  template <class T>
  S::operator T ()
  { return {}; }

  template <class A, class B>
  class C
  {
      template <class X>
      void f();
  };

  template <class A, class B>
  template <class X>
  void C<A, B>::f()
  { }

  int main()
  {
      C<int, char> c;
      c.f<double>(); // ?
      S s;
      return s;
  }
  \end{lstlisting}
\end{frame}

\begin{frame}[fragile]{Примеры шаблонов псевдонимов типов}
  \begin{lstlisting}
  template <class T>
  using Map = std::map<T, T>;

  template <class T>
  using Identity = T;
  \end{lstlisting}
\end{frame}

\section{Специализация шаблонов}

\begin{frame}[fragile]{Полная специализация шаблонов}
  \begin{lstlisting}[basicstyle=\ttfamily\srcsize]
  template <class T>
  constexpr int t_tag = 11;

  template <>
  constexpr int t_tag<int> = 20;
  template <>
  constexpr int t_tag<char> = 30;
  template <>
  constexpr int t_tag<double> = 40;

  template <class T>
  T max(const T & a, const T & b)
  { return a < b ? b : a; }

  template <>
  int max(const int & a, const int & b)
  { return a - b; }

  template <bool>
  struct S;

  template <>
  struct S<false> {};

  template <class T>
  std::size_t check(const T & x)
  { return sizeof(S<t_tag<T> == 20>); }

  int main()
  {
      check('a'); // ?
      check(1); // ?
  }
  \end{lstlisting}
\end{frame}

\begin{frame}[fragile]{Частичная специализация шаблонов}
  \begin{lstlisting}[basicstyle=\ttfamily\srcsize]
  template <class T>
  constexpr int t_tag = 11;

  template <template <class...> class L, class... Ts>
  constexpr int t_tag<L<Ts...>> = 111;

  template <class T, bool = true>
  struct MakeNonAbstract
  {
      template <class... Args>
      static std::unique_ptr<T> make(Args &&... args)
      { return {}; }
  };
  template <class T>
  struct MakeNonAbstract<T, false>
  {
      template <class... Args>
      static std::unique_ptr<T> make(Args &&... args)
      { return std::make_unique<T>(std::forward<Args>(args)...); }
  };

  template <class T, class... Args>
  auto make_non_abstract(Args &&... args)
  {
      return MakeNonAbstract<T, std::is_abstract_v<T>>::make(std::forward<Args>(args)...);
  }

  struct A { virtual void f() = 0; };

  int main()
  {
      auto p_a = make_non_abstract<A>();
      auto p_i = make_non_abstract<int>(1);
      return t_tag<std::tuple<>>;
  }
  \end{lstlisting}
\end{frame}

\section{Особенности использования шаблонов}

\begin{frame}[fragile]{Инстанциация шаблонов классов}
  \begin{lstlisting}[basicstyle=\ttfamily\srcbigsize]
  template <class T>
  struct A;

  template <class T>
  struct B
  {
      T create()
      { return {}; }

      void use(const T & x)
      { x.use(); }
  };

  int main()
  {
      A<int> * p_a = nullptr;

      B<int> b;
      return b.create();
  }
  \end{lstlisting}
\end{frame}

\begin{frame}[fragile]{Разрешение неоднозначности зависимых имён}
  \begin{lstlisting}
  template <class T>
  struct S
  {
      using value_type = typename T::value_type;

      void f(const T & x)
      {
          x.template f<int>();
      }
  };
  \end{lstlisting}
\end{frame}

\begin{frame}[fragile]{CRTP}
  \begin{lstlisting}
  template <class T>
  class Base
  {
  public:
      void f()
      {
          static_cast<T *>(this)->f_impl();
      }
  };

  class Derived : public Base<Derived>
  {
      friend class Base<Derived>;

      void f_impl();
  };
  \end{lstlisting}
\end{frame}

\end{document}
