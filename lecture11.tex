\documentclass[unknownkeysallowed,xcolor=table]{beamer}
 
\usepackage [T2A,T1] {fontenc}
\usepackage[utf8]{inputenc}
\usepackage[english,russian]{babel}
\usepackage{amsmath}
\usepackage{listings}
\usepackage{url}
\usepackage{textcomp}
\usepackage{multirow}

\setbeamertemplate{navigation symbols}{}

\newcommand{\textapprox}{\raisebox{0.5ex}{\texttildelow}}

\lstnewenvironment{cmdline}
  {\lstset{basicstyle=\ttfamily\scriptsize,keywordstyle=\color{blue},
           language={bash}}}
  {}
  
\lstnewenvironment{cmdlinelarge}
  {\lstset{basicstyle=\ttfamily\small,keywordstyle=\color{blue},
           language={bash}}}
  {}
  
\colorlet{mygreen}{green!60!blue}
\colorlet{mymauve}{red!60!blue}
\definecolor{pblue}{rgb}{0.1, 0.2, 0.8}

\newcommand{\rarr}{$\rightarrow$}

\lstset{
      basicstyle=\ttfamily\small,
      commentstyle=\color{mygreen},
      keywordstyle=\color{blue},
      numberstyle=\tiny\color{blue},
      stringstyle=\color{mymauve},
      numbers=left,
      stepnumber=1,
      columns=fullflexible,
      breaklines=true,
      postbreak=\mbox{\textcolor{red}{\ensuremath{\hookrightarrow}\space}},
      literate={~} {\textapprox}{1},
      language={[11]C++}
}

\makeatletter
\newcommand{\srcmediumsize}{\@setfontsize{\srcmediumsize}{7pt}{7pt}}
\makeatother

\makeatletter
\newcommand{\srcbigsize}{\@setfontsize{\srcbigsize}{8pt}{8pt}}
\makeatother

\makeatletter
\newcommand{\srcsize}{\@setfontsize{\srcsize}{6pt}{6pt}}
\makeatother

\makeatletter
\newcommand{\srcsmallsize}{\@setfontsize{\srcsmallsize}{5pt}{5pt}}
\makeatother

\newcommand{\rot}[1]{\makebox[1em][l]{\rotatebox{45}{#1}}}

\title[C++]
{Программирование на языке C++}
 
\subtitle{Вводный курс}
 
\author[А.~Б.~Морозов]
{
  \texorpdfstring{Александр Морозов\newline\href{mailto:gelu.speculum@gmail.com}{gelu.speculum@gmail.com}}
  {Александр Морозов}
}
  
\date[ITMO 2019]
{ИТМО, весенний семестр 2019}
 
\logo{%
  \makebox[0.97\paperwidth]{%
    \includegraphics[align=c,width=2cm,keepaspectratio]{itmo_logo.png}
    \hfill
    \includegraphics[align=c,width=1.5cm,keepaspectratio]{new_itiviti_logo.png}
  }
}

\AtBeginSection[]
{
  \begin{frame}
    \frametitle{Содержание}
    \tableofcontents[currentsection]
  \end{frame}
}

\begin{document}

\frame{\titlepage}


%-------------------------------------------------

\section{Неявные преобразования типов}

\begin{frame}{Последовательность неявного преобразования типов}
  \begin{enumerate}
    \item 0 или 1 цепочка стандартных преобразований \vspace{0.5em}
    \item 0 или 1 пользовательское преобразование \vspace{0.5em}
    \item 0 или 1 цепочка стандартных преобразований
  \end{enumerate}

  \vspace{1em}

  Цепочка стандартных преобразований: \vspace{0.5em}
  \begin{enumerate}
    \item 0 или 1 преобразование категории значения, преобразование массива или функции к указателю \vspace{0.5em}
    \item 0 или 1 числовое расширение или числовое преобразование \vspace{0.5em}
    \item 0 или 1 повышение квалификации
  \end{enumerate}
\end{frame}

\begin{frame}[fragile]{Пользовательские преобразования}
  \begin{itemize}
    \item не \lstinline{explicit} конструктор \vspace{0.5em}
    \item не \lstinline{explicit} оператор приведения типа
  \end{itemize}
  \begin{lstlisting}
  struct S
  {
    S(int);
    operator double() const;
  };

  double f(const S & s)
  { return s; }

  double g(const int x)
  { return f(x); }
  \end{lstlisting}
\end{frame}

%-------------------------------------------------

\section{Перегрузка функций}

\begin{frame}[fragile]{Перегрузка функций}

Допустимо иметь несколько функций с одним именем, если в каждом случае вызова функции с таким именем компилятор может найти один наиболее подходящий вариант.

\vspace{1em}

\begin{lstlisting}
void f(int) { ... }
void f(double) { ... }
void f(const S &) { ... }

f(10);
\end{lstlisting}

\end{frame}

\begin{frame}{Этапы выбора функции}

\begin{itemize}
  \item поиск кандидатов -- поиск сущностей по имени
  \item выбор подходящих к контексту вызова
  \item выбор единственного наилучшего кандидата
\end{itemize}

\vspace{1em}

При этом в зависимости от контекста вызова ищутся:
\begin{itemize}
  \item контекст вызова функции -- функции с таким именем
  \item контекст функционального вызова объекта -- члены-операторы функционального вызова соответствующего класса и его предков, а также операторы преобразования типа к типу подходящей функции
  \item контекст вызова оператора -- члены-операторы (если первый или единственный операнд имеет сложный тип), свободные операторы, встроенные операторы
\end{itemize}

\end{frame}

\begin{frame}{Правила выбора подходящих к контексту вызова}
Учитываются факторы:

\vspace{1em}

\begin{itemize}
  \item число аргументов \vspace{1em}
  \item типы аргументов \vspace{1em}
  \item допустимая ссылочная связываемость
\end{itemize}
\end{frame}

\begin{frame}{Ранжирование неявных преобразований типов}
  Каждому из 3 возможных этапов последовательности неявного преобразования назначается ранг:
  \begin{enumerate}
    \item точное совпадение
    \item числовое расширение
    \item преобразование
  \end{enumerate}

  \vspace{1em}

  Ранг всей последовательности = наибольшему рангу этапов.

  \vspace{1em}

  Связывание со ссылкой -- либо точное совпадение, либо преобразование (к одному из базовых типов).

  \vspace{1em}

  Отсутствие необходимости преобразования считается последовательностью нулевой длины с минимальным рангом.
\end{frame}

\begin{frame}{Ранжирование неявных преобразований типов, продолжение}
  Стандартное преобразование лучше пользовательского.

  \vspace{1em}

  Чем короче необходимая цепочка стандартных преобразований -- тем она лучше.

  \vspace{1em}

  Связывание rvalue с rvalue ссылкой лучше связывания rvalue с const lvalue ссылкой.

  \vspace{1em}

  Среди связываний lvalue с lvalue ссылкой лучше то, у которого меньше cv-квалификаторов.

  \vspace{1em}

  Из двух последовательностей, содержащих пользовательское преобразование к одному типу, лучше та, у которой лучше третий этап.
\end{frame}

\begin{frame}{Выбор единственного наилучшего кандидата}
Все подходящие кандидаты попарно сравниваются и выбирается наилучший из них. Несколько одинаково наилучших кандидатов -- ошибка компиляции.

\vspace{1em}

Один кандидат лучше другого, если требуемые неявные преобразования для всех его аргументов не хуже (имеют ранг не ниже) таковых для конкурента, а кроме того:

\begin{enumerate}
  \item для хотя бы одного аргумента требуемое неявное преобразование лучше, чем у конкурента
  \item либо этот кандидат -- не шаблонный, а конкурент -- шаблонный
  \item либо этот кандидат более шаблонно-специализированный, чем конкурент
\end{enumerate}

\end{frame}

%-------------------------------------------------

\section{Друзья классов}

\begin{frame}{Друзья классов}
\lstinline{friend} объявление: в теле класса можно объявить другой класс или функцию ``другом'', что даст такому другу доступ ко всем членам класса (игнорируя спецификаторы доступа).

\vspace{1em}

Отношение дружбы -- не транзитивно, не наследуемо.

\vspace{1em}

Возможные формы:
\begin{itemize}
  \item объявление функции (оператора)
  \item определение функции (оператора)
  \item предварительное объявление класса
  \item объявление видимого класса / \lstinline{union}
\end{itemize}

\end{frame}

\begin{frame}[fragile]{Примеры friend}
\begin{lstlisting}
struct A {
    void f();
    int h();
    int g(double) const;
};
class B {};
class C {
    int x;
    
    friend std::ostream & operator << (std::ostream & strm, const C & c); // definition later
    friend void A::f();
    friend int A::f(), A::g(double) const;
    friend class D; // definition later
    friend B;
};
\end{lstlisting}
\end{frame}

\begin{frame}[fragile]{Пример friend определения функции}
\begin{lstlisting}
class C {
public:
    C(int i) : x(i) {}
private:
    int x;
    
    friend int get_x(const C & c)
    { return c.x; }
};

C c(10);
int a = get_x(c); // OK (ADL)
int b = get_x(10); // error: 'get_x' is not declared
\end{lstlisting}
\end{frame}

\begin{frame}[fragile]{Пример ограничения friend}
\begin{lstlisting}
class BigOne {
    // lots of stuff
    friend class Backdoor;
};

class Backdoor {
    friend class UserA;
    friend class UserB;
    friend class UserC;
    friend class UserD;
    
    void access();
};
\end{lstlisting}
\end{frame}


%-------------------------------------------------

\section{Перегрузка операторов}

\begin{frame}{Перегрузка операторов}
Перегруженные операторы -- функции со специальным именем.

\vspace{1em}

\begin{itemize}
  \item \lstinline{operator} \emph{op} -- обычный оператор
  \item \lstinline{operator} \emph{type} -- пользовательское приведение типа
  \item \lstinline{operator new} -- пользовательская аллокация
  \item \lstinline{operator new []} -- пользовательская аллокация
  \item \lstinline{operator delete} -- пользовательская деаллокация
  \item \lstinline{operator delete []} -- пользовательская деаллокация
  \item \lstinline{operator ""} \emph{suffix} -- пользовательский литерал
\end{itemize}

\vspace{1em}

Если в выражении один из операндов -- сложный тип (класс, \lstinline{enum}), то производится поиск перегруженной функции, реализующей данный оператор.

\end{frame}

\begin{frame}{Ограничения перегрузки операторов}

\begin{itemize}
  \item Нельзя перегружать некоторые операторы (например, \lstinline{::}) \vspace{0.5em}
  \item Нельзя менять приоритет, ассоциативность операторов \vspace{0.5em}
  \item Нельзя менять число аргументов операторов \vspace{0.5em}
  \item Нельзя создавать новые операторы \vspace{0.5em}
  \item Перегруженный оператор \lstinline{->} должен возвращать либо указатель, либо объект, для которого перегружен \lstinline{->} \vspace{0.5em}
  \item Перегруженные операторы \lstinline{&&} и \lstinline{||} не имеют ленивого вычисления аргументов (но сохраняют порядок вычисления аргументов)
\end{itemize}

\end{frame}

\begin{frame}[fragile]{Операторы -- члены классов}

Оператор может быть определён, как член класса:

\vspace{0.5em}

\begin{lstlisting}
struct SS {};
struct S {
    S & operator += (const S & other) { ... }
    const SS & operator * () const { ... }
};

S s1, s2;
s1 += s2;
const SS & ss = *s1;
\end{lstlisting}

\vspace{1em}

При поиске имён, если первый операнд в выражении вызова оператора имеет сложный тип, то осуществляется как поиск свободных функций, так и членов классов.
\end{frame}

\begin{frame}[fragile]{Отображение синтаксиса операторов}

\begin{center}
\begin{tabular}{l l l}
  \hline
    Оператор & Член класса & Свободная функция \\
  \hline
    & &\\
    @a & (a).operator@ () & operator@ (a) \\
    a@ & (a).operator@ (0) & operator@ (a, 0) \\
    a@b & (a).operator@ (b) & operator@ (a, b) \\
    a=b & (a).operator= (b) & -- \\
    a(b…) & (a).operator() (b…) & -- \\
    a[b] & (a).operator[] (b) & -- \\
    a-> & (a).operator-> () & -- \\
\end{tabular}
\end{center}

\end{frame}

\begin{frame}[fragile]{Доступные для перегрузки операторы}

\begin{itemize}
  \item арифметические \lstinline{+, -, *, /, %, +=, -=, *=, /=, %=}
  \item инкремента/декремента \lstinline{++, --}
  \item битовые \lstinline{^, &, |, ~, ^=, &=, |=}
  \item сдвига \lstinline{<<, >>, <<=, >>=}
  \item логические \lstinline{&&, ||, !}
  \item сравнения \lstinline{<, >, ==, !=, <=, >=}
  \item присваивания \lstinline{=}
  \item разыменования, доступа к члену через указатель и указатель на член \lstinline{*, ->, ->*}
  \item индексации \lstinline{[]}
  \item функционального вызова \lstinline{()}
  \item запятой \lstinline{,}
\end{itemize}

\end{frame}

\begin{frame}[fragile]{Оператор пользовательского приведения типа}

\begin{lstlisting}
struct S {
    operator X ();
    explicit operator Y ();
    operator auto () { return 1; }
    template <class T>
    operator T ();
};

S s;
X x = s;
Y y = static_cast<Y>(s);
Y yy(s);
\end{lstlisting}

\vspace{1em}

Такой метод объявляется нестатическим, без параметров и без возвращаемого типа. Может иметь cv-квалификатор, \lstinline{inline}, \lstinline{virtual}, \lstinline{constexpr} спецификаторы.

\end{frame}

\end{document}
