\documentclass[unknownkeysallowed,xcolor=table]{beamer}
 
\usepackage [T2A,T1] {fontenc}
\usepackage[utf8]{inputenc}
\usepackage[english,russian]{babel}
\usepackage{amsmath}
\usepackage{pifont}
\usepackage{listings}
\usepackage{url}
\usepackage{textcomp}
\usepackage{multirow}

% to not copy line numbers
\usepackage{accsupp}
% to not copy line numbers

% to not copy line numbers
\newcommand{\noncopynumber}[1]{%
  \BeginAccSupp{method=escape,ActualText={}}%
    \color{blue} #1%
  \EndAccSupp{}%
}
% to not copy line numbers

\setbeamertemplate{navigation symbols}{}

\newcommand{\textapprox}{\raisebox{0.5ex}{\texttildelow}}

\colorlet{mygreen}{green!60!blue}
\colorlet{mymauve}{red!60!blue}
\definecolor{pblue}{rgb}{0.1, 0.2, 0.8}

\newcommand{\rarr}{$\rightarrow$}

\lstset{
      basicstyle=\ttfamily\small,
      commentstyle=\color{mygreen},
      keywordstyle=\color{blue},
%      numberstyle=\tiny\color{blue},
      numberstyle=\noncopynumber,
      stringstyle=\color{mymauve},
      numbers=left,
      stepnumber=1,
      columns=fullflexible,
      breaklines=true,
      postbreak=\mbox{\textcolor{red}{\ensuremath{\hookrightarrow}\space}},
      literate={~} {\textapprox}{1},
      language={[11]C++}
}

\makeatletter
\newcommand{\srcmediumsize}{\@setfontsize{\srcmediumsize}{7pt}{7pt}}
\makeatother

\makeatletter
\newcommand{\srcbigsize}{\@setfontsize{\srcbigsize}{8pt}{8pt}}
\makeatother

\makeatletter
\newcommand{\srcsize}{\@setfontsize{\srcsize}{6pt}{6pt}}
\makeatother

\makeatletter
\newcommand{\srcsmallsize}{\@setfontsize{\srcsmallsize}{5pt}{5pt}}
\makeatother

\title[C++]
{Программирование на языке C++}
 
\subtitle{Вводный курс}
 
\author[А.~Б.~Морозов]
{
  \texorpdfstring{Александр Морозов\newline\href{mailto:gelu.speculum@gmail.com}{gelu.speculum@gmail.com}}
  {Александр Морозов}
}
  
\date[ITMO 2020]
{ИТМО, весенний семестр 2020}
 
\logo{%
  \makebox[0.97\paperwidth]{%
    \includegraphics[align=c,width=2cm,keepaspectratio]{itmo_logo.png}
    \hfill
    \includegraphics[align=c,width=1.5cm,keepaspectratio]{itiviti_logo.png}
  }
}

\AtBeginSection[]
{
  \begin{frame}
    \frametitle{Содержание}
    \tableofcontents[currentsection]
  \end{frame}
}

\begin{document}

\frame{\titlepage}

%-------------------------------------------------

\section{Инкапсуляция, наследование, полиморфизм}

\begin{frame}{Инкапсуляция}
  \begin{itemize}
    \item сокрытие логически связанных подробностей в одном месте
    \item объединение данных и логики их обработки
  \end{itemize}

  \vspace{1em}

  Помогает разделять программу на слабо связанные компоненты.

  \vspace{1em}

  Варианты:
  \begin{itemize}
    \item ООП
    \item пространства имён и модули (Erlang, Haskell)
    \item функции с локальным контекстом (Javascript, Scheme, Pascal)
  \end{itemize}
\end{frame}

\begin{frame}{Зачем нужно разделение?}
  \begin{itemize}
    \item легче читать \vspace{1em}
    \item легче менять \vspace{1em}
    \item легче тестировать
  \end{itemize}
\end{frame}

\begin{frame}{Наследование}
  \begin{itemize}
    \item возможность переиспользовать код \vspace{1em}
    \item возможность расширять функциональность \vspace{1em}
    \item возможность компоновать код
  \end{itemize}
\end{frame}

\begin{frame}{Статический и динамический полиморфизм}
  Статический:
  \begin{itemize}
    \item использование одного кода для разных типов
    \item на этапе компиляции полиморфизм раскрывается в обычный код
    \item на этапе исполнения не остаётся следов
  \end{itemize}

  \vspace{1em}

  Динамический:
  \begin{itemize}
    \item возможность работать с объектами разных типов через один интерфейс
    \item поведение выбирается на этапе исполнения
    \item рефлексия
    \item гетерогенные коллекции
  \end{itemize}
\end{frame}

\begin{frame}{Динамический полиморфизм в C++}
  \begin{itemize}
    \item наследование классов \vspace{1em}
    \item ссылки или указатели \vspace{1em}
    \item приведение типов \vspace{1em}
    \item виртуальные методы
  \end{itemize}
\end{frame}

\begin{frame}[fragile]{Наследование без полиморфизма}
  \begin{lstlisting}[basicstyle=\ttfamily\srcbigsize]
  class A
  {
      int m_a = 11;
  public:
      int compute() const
      { return m_a; }
  };

  class B : public A
  {
      int m_b = 22;
  public:
      int compute() const
      { return m_a + m_b; }
  };

  void process(const A & a)
  { std::cout << a.compute() << std::endl; }

  int main()
  {
      B b;
      process(b);
  }
  \end{lstlisting}
\end{frame}

\begin{frame}[fragile]{Наследование с полиморфизмом}
  \begin{lstlisting}[basicstyle=\ttfamily\srcbigsize]
  class A
  {
      int m_a = 11;
  public:
      virtual int compute() const
      { return m_a; }
  };

  class B : public A
  {
      int m_b = 22;
  public:
      int compute() const
      { return m_a + m_b; }
  };

  void process(const A & a)
  { std::cout << a.compute() << std::endl; }

  int main()
  {
      B b;
      process(b);
  }
  \end{lstlisting}
\end{frame}

\section{Виртуальные методы классов}

\begin{frame}[fragile]{Виртуальные методы}
  \lstinline{virtual} \emph{объявление метода}

  \vspace{1em}

  \begin{lstlisting}
  class A
  {
      virtual void f();
  };

  void A::f() { ... }
  \end{lstlisting}
\end{frame}

\begin{frame}{Ограничения применения виртуальности}
  Не могут быть виртуальными: \vspace{1em}
  \begin{itemize}
    \item свободные функции \vspace{0.5em}
    \item статические методы \vspace{0.5em}
    \item шаблонные методы
  \end{itemize}
\end{frame}

\begin{frame}{Переопределение виртуальных методов}
  Если класс ниже по иерархии наследования определяет метод
  \begin{itemize}
    \item с тем же именем
    \item с тем же списком параметров
    \item с теми же cv-квалификаторами
    \item с теми же ссылочными квалификаторами
  \end{itemize}
  то этот метод тоже становится виртуальным и переопределяет исходный.
\end{frame}

\begin{frame}[fragile]{Особенности переопределения и вызова виртуальных методов}
  \begin{lstlisting}[basicstyle=\ttfamily\srcbigsize]
  class A
  {
  public:
      virtual void f();
      virtual int g();
  };

  class B
  {
  private:
      void f();
      int g() const; // warning
  };

  class C
  {
  public:
      double g(); // error
  };

  void call(A & a)
  {
      a.f();
      a.A::f();
  }

  int main()
  {
      B b;
      call(b);
  }
  \end{lstlisting}
\end{frame}

\begin{frame}[fragile]{Скрытие и переопределение методов при наследовании}
  \lstinline{Base} \rarr ... \rarr \lstinline{Derived}

  \vspace{0.5em}

  \lstinline{T Base::f(int);}

  \vspace{1em}

  \begin{center}
    \begin{tabular}{ l | c | c }
      & не виртуальный & виртуальный \\
      \hline
      T Derived::f(int) & скрывает & переопределяет \\
      P Derived::f(int) & скрывает & переопределяет, если P ковариантен T \\
      T Derived::f() & скрывает & скрывает \\
      T Derived::f() const & скрывает & скрывает \\
    \end{tabular}
  \end{center}
\end{frame}

\begin{frame}{Ковариантный тип}
  Y ковариантен X, если
  \begin{itemize}
    \item оба являются ссылкой или указателем на класс \vspace{0.5em}
    \item Y не более cv-квалифицирован, нежели X \vspace{0.5em}
    \item класс, на который сслыается/указывает X, должен быть однозначным и доступным предком класса, на который ссылается/указывает Y
  \end{itemize}
\end{frame}

\begin{frame}[fragile]{Пример с ковариантным типом}
  \begin{lstlisting}
  struct Base
  {
      const Base * clone()
      { return new Base(*this); }
  };

  struct Derived : Base
  {
      Derived * clone()
      { return new Derived(*this); }
  };

  int main()
  {
      Derived d;
      auto c = d.clone(); // Derived *
      Base & b = d;
      auto cc = b.clone(); // const Base *
  }
  \end{lstlisting}
\end{frame}

\begin{frame}[fragile]{Пример без ковариантного типа}
  \begin{lstlisting}
  struct Base
  {
      const Base * clone()
      { return new Base(*this); }
  };

  struct Derived : Base
  {
      const Base * clone()
      { return new Derived(*this); }
  };

  int main()
  {
      Derived d;
      auto c = d.clone(); // const Base *
      Base & b = d;
      auto cc = b.clone(); // const Base *
  }
  \end{lstlisting}
\end{frame}

\begin{frame}{\lstinline{override} и \lstinline{final}}
  Спецификаторы \lstinline{override} и \lstinline{final} можно использовать в конце объявления метода.

  \vspace{2em}

  \lstinline{override} -- метод переопределяет унаследованный виртуальный метод.

  \vspace{1em}

  \lstinline{final} -- метод переопределяет унаследованный виртуальный метод и не может быть переопределён в наследниках.
\end{frame}

\begin{frame}[fragile]{Пример использования \lstinline{override}}
  \begin{lstlisting}[basicstyle=\ttfamily\srcbigsize]
  struct A
  {
      virtual void f(const X &);
      void g(int);
  };

  struct B : A
  {
      void f(const X);
  };

  struct C : A
  {
      void f(const X &) const override;
      void g(int) override;
  };

  void call(A & a, const X & x)
  {
      a.f(x);
  }
  \end{lstlisting}
\end{frame}

\begin{frame}[fragile]{Пример использования \lstinline{final}}
  \begin{lstlisting}[basicstyle=\ttfamily\srcbigsize]
  struct A
  {
      virtual void f(const X &);
  };

  struct B : A
  {
      void f(const X &) final;
  };

  struct C : B
  {
      void f(const X &); // error
  };
  \end{lstlisting}
\end{frame}

\begin{frame}[fragile]{\lstinline{final} в наследовании}
  \begin{lstlisting}
  struct A
  {
  };

  struct B final : A
  {
  };

  struct C : B // error
  {
  };
  \end{lstlisting}
\end{frame}

\begin{frame}[fragile]{Полиморфизм и деструкторы}
  \begin{lstlisting}[basicstyle=\ttfamily\srcbigsize]
  class Base
  {
  public:
      virtual int compute()
      { return 0; }
  };

  class Derived : public Base
  {
      std::vector<int> m_data;

      int compute()
      { return std::accumulate(m_data.begin(), m_data.end(), 0); }
  public:
      Derived(const std::size_t size)
          : m_data(size)
      { std::iota(m_data.begin(), m_data.end(), 1); }
  };

  int main()
  {
      Base * b = new Derived(100);
      int res = b->compute();
      delete b;
      return res;
  }
  \end{lstlisting}
\end{frame}

\begin{frame}[fragile]{Корректное полиморфное удаление}
  \begin{lstlisting}
  class Base
  {
  public:
      virtual ~Base() = default;
  };

  class Derived : public Base
  { ... };

  int main()
  {
      Base * b = new Derived;
      delete b; // ~Derived() is called
  }
  \end{lstlisting}
\end{frame}

\begin{frame}[fragile]{Чисто виртуальный метод и абстрактные классы}
  \begin{lstlisting}
  struct A
  {
      virtual void f() = 0;
  };

  struct B : A
  {
      void f();
  };

  int main()
  {
      A * a = new B; // OK
      A * aa = new A; // error
  }
  \end{lstlisting}
\end{frame}

\begin{frame}[fragile]{Больше чистой виртуальности}
  \begin{lstlisting}[basicstyle=\ttfamily\srcbigsize]
  struct A
  {
      virtual void f() = 0;
  }

  void A::f() { ... }

  struct B
  {
      void f()
      { A::f(); }
  };

  struct C
  {
      virtual ~C() = 0;
  };

  C::~C() { ... } // mandatory

  struct D : C
  {
      ~D(); // or else D is abstract
  };

  struct E
  {
      virtual void g() { ... }
  };

  struct F : E
  {
      void g() = 0;
  };
  \end{lstlisting}
\end{frame}

\begin{frame}[fragile]{Параметры по умолчанию в виртуальных методах}
  \begin{lstlisting}[basicstyle=\ttfamily\srcbigsize]
  int m = 111;

  struct A
  {
      virtual int f(int k = m) { return k; }
  };

  struct B : A
  {
      int f(int k = -13) { return k; }
  };

  int get_n(A & a)
  {
      return a.f();
  }

  int main()
  {
      B b;
      return get_n(b);
  }
  \end{lstlisting}
\end{frame}

\begin{frame}{Рекомендации по использованию виртуальности}
  \begin{itemize}
    \item интерфейс (в базовом классе) делать не виртуальным \vspace{0.5em}
    \item переопределяемую реализацию оформлять приватными виртуальными методами базового класса \vspace{0.5em}
    \item методы интерфейса вызывают нужные виртуальные методы реализации \vspace{0.5em}
    \item если предполагается полиморфное создание/удаление - сделать виртуальный деструктор в базовом классе \vspace{0.5em}
    \item иначе рассмотреть возможность защищённого не виртуального деструктора
  \end{itemize}
\end{frame}

\section{Технические аспекты наследования}

\begin{frame}{Приведение типов по иерархии наследования}
  Можно приводить указатели и ссылки на классы вверх и вниз по иерархии наследования.

  \vspace{1em}

  \begin{itemize}
    \item Неявное приведение вверх: \textbf{Derived *} \rarr \textbf{Base *} \vspace{0.5em}
    \item Явное приведение вниз: \textbf{Base *} \rarr \textbf{Derived *} \vspace{0.5em}
    \item Типо-безопасное приведение по иерархии наследования
  \end{itemize}

  \vspace{1em}

  При таком приведении сам объект \textbf{не меняется} -- меняется лишь указатель или ссылка для доступа к нему.
\end{frame}

\begin{frame}[fragile]{Пример иерархии наследования}
  \begin{lstlisting}[basicstyle=\ttfamily\srcsize]
  struct LeftRoot
  {
      virtual void print() const
      { std::cout << "LeftRoot" << std::endl; }
  };

  struct LeftMiddle : LeftRoot
  {
      void print() const
      { std::cout << "LeftMiddle" << std::endl; }
  };

  struct LeftLeaf : LeftMiddle
  {
      void print() const
      { std::cout << "LeftLeaf" << std::endl; }
  };

  struct RightRoot
  {
      virtual void another_print() const
      { std::cout << "RightRoot" << std::endl; }
  };

  struct Middle : LeftMiddle, RightRoot
  {
      void print() const
      { std::cout << "Middle" << std::endl; }
      void another_print() const
      { std::cout << "Middle" << std::endl; }
  };

  struct RightLeaf : Middle
  {
      void print() const
      { std::cout << "RightLeaf" << std::endl; }
      void another_print() const
      { std::cout << "RightLeaf" << std::endl; }
  };
  \end{lstlisting}
\end{frame}

\begin{frame}[fragile]{Разные варианты приведения по иерархии}
  \begin{center}
    \begin{tabular}{ l | c | c | c }
      & Неявное & static_cast & dynamic_cast \\
      \hline
      LeftLeaf \rarr LeftRoot & \cmark & \cmark & \cmark \\
      RightLeaf \rarr LeftRoot & \cmark & \cmark & \cmark \\
      RightLeaf \rarr RightRoot & \cmark & \cmark & \cmark \\
      LeftRoot \rarr LeftLeaf & \xmark & \cmark (!) & \cmark \\
      Middle \rarr LeftRoot \rarr LeftLeaf & \xmark & \textcolor{red}{\cmark} & \xmark \\
      RightLeft \rarr LeftRoot \rarr RightRoot & \xmark & \xmark & \cmark \\
      LeftRoot \rarr ... & \xmark & \xmark & \cmark \\
    \end{tabular}
  \end{center}
\end{frame}

\begin{frame}[fragile]{Варианты использования \lstinline{dynamic_cast}}
  \lstinline{X x = dynamic_cast<X>(y);}

  \vspace{1em}

  \begin{itemize}
    \item \lstinline{X} $\equiv$ \lstinline{decltype(y)}
    \item \lstinline{X} $\equiv$ \lstinline{const decltype(y)}
    \item \lstinline{y == nullptr} \rarr \lstinline{x == nullptr}
    \item \lstinline{X} -- однозначный, доступный предок \lstinline{decltype(y)}
    \item \lstinline{X = void *}, \lstinline{decltype(y)} -- полиморфный
    \item \lstinline{decltype(y)} -- полиморфный, \lstinline{X} -- ссылка или указатель на класс
      \begin{itemize}
        \item \lstinline{X} -- указывает/ссылается на однозначного, публично унаследованного потомка \lstinline{y}
        \item \lstinline{y} -- публичный предок объекта класса \lstinline{Z}, а \lstinline{X} -- другой однозначный, доступный предок \lstinline{Z}
        \item иначе,
          \begin{itemize}
            \item \lstinline{X} -- указатель, \lstinline{x == nullptr}
            \item \lstinline{X} -- ссылка, \lstinline{std::bad_cast} исключение
          \end{itemize}
      \end{itemize}
  \end{itemize}
\end{frame}

\begin{frame}[fragile]{RTTI}
  Run-time type information

  \vspace{2em}

  \begin{itemize}
    \item \lstinline{dynamic_cast} \vspace{1em}
    \item \lstinline{typeid}
  \end{itemize}
\end{frame}

\begin{frame}[fragile]{Множественное наследование}
  \begin{lstlisting}[basicstyle=\ttfamily\srcbigsize]
  struct A
  {
      int a;

      A(int a_ = 0) : a(a_) {}
  };

  struct B : A
  {
      B(int a_) : A(a_) {}

      int b_get_a() const
      { return a; }
  };

  struct C : A
  {
      C(int a_) : A(a_) {}

      int c_get_a() const
      { return a; }
  };

  struct D : B, C
  {
      D(int b, int c) : B(b), C(c) {}
  };

  int main()
  {
      D d(11, 22);
      std::cout << d.a << std::endl; // error
      std::cout << d.b_get_a() << std::endl; // OK
      std::cout << d.c_get_a() << std::endl; // OK
  }
  \end{lstlisting}
\end{frame}

\begin{frame}[fragile]{Множественное наследование и абстрактные классы}
  \begin{lstlisting}[basicstyle=\ttfamily\srcbigsize]
  struct Interface
  {
      virtual void f() = 0;
      virtual void g() = 0;
  };

  struct A : Interface
  {
      void f();
  };

  struct B : Interface
  {
      void g();
  };

  struct C : A, B
  {};

  int main()
  {
      C c; // error: C is abstract
  }
  \end{lstlisting}
\end{frame}

\begin{frame}{Виртуальное наследование}
  При указании наследования используется ключевое слово \lstinline{virtual}

  \vspace{2em}

  \begin{itemize}
    \item в каждом конкретном типе-наследнике виртуальный предок встречается как подобъект ровно один раз \vspace{0.5em}
    \item конкретный тип-наследник создаёт подобъекты виртуальных предков (даже если в иерархии наследования между ними другие классы), причём до создания подобъектов
      обычных предков \vspace{0.5em}
    \item наследник должен иметь доступ к конструктору виртуального предка \vspace{0.5em}
    \item в списке инициализации конструктора наследника можно явно вызвать конструктор виртуального предка \vspace{0.5em}
    \item операция копирования конкретного наследника должна учитывать эти особенности \vspace{0.5em}
    \item для приведения типов по иерархии наследования с виртуальным наследованием необходимо использовать \lstinline{dynamic_cast}
  \end{itemize}
\end{frame}

\begin{frame}[fragile]{Решение проблемы неоднозначности с виртуальным наследование}
  \begin{lstlisting}[basicstyle=\ttfamily\srcbigsize]
  struct A
  {
      int a;

      A(int a_) : a(a_) {}
  };

  struct B : virtual A
  {
      B(int a_) : A(a_) {}

      int b_get_a() const
      { return a; }
  };

  struct C : virtual A
  {
      C(int a_) : A(a_) {}

      int c_get_a() const
      { return a; }
  };

  struct D : B, C
  {
      D(int b, int c) : A(b+c), B(b), C(c) {}
  };

  int main()
  {
      D d(11, 22);
      std::cout << d.a << std::endl; // OK
      std::cout << d.b_get_a() << std::endl; // OK
      std::cout << d.c_get_a() << std::endl; // OK
  }
  \end{lstlisting}
\end{frame}

\begin{frame}[fragile]{Решение проблемы частичной реализации интерфейса и объединения этих реализаций}
  \begin{lstlisting}[basicstyle=\ttfamily\srcbigsize]
  struct Interface
  {
      virtual void f() = 0;
      virtual void g() = 0;
  };

  struct A : virtual Interface
  {
      void f()
      { g(); }
  };

  struct B : virtual Interface
  {
      void g();
  };

  struct C : A, B
  {};

  int main()
  {
      C c; // OK
      c.f(); // call B::g()
  }
  \end{lstlisting}
\end{frame}

\end{document}
