\documentclass[unknownkeysallowed,xcolor=table]{beamer}
 
\usepackage[T2A,T1]{fontenc}
\usepackage[utf8]{inputenc}
\usepackage[english,russian]{babel}
\usepackage{listings}
\usepackage{amsmath}
\usepackage{url}
\usepackage{textcomp}
\usepackage{multirow}
\usepackage{tikz}
\usepackage{beramono}

\setbeamertemplate{navigation symbols}{}

\newcommand{\textapprox}{\raisebox{0.5ex}{\texttildelow}}

\newcommand{\rarr}{$\rightarrow$}
 
\colorlet{mygreen}{green!60!blue}
\colorlet{mymauve}{red!60!blue}
\definecolor{light-gray}{gray}{0.9}

\lstset{
      basicstyle=\ttfamily\small,
      commentstyle=\color{mygreen},
      keywordstyle=\color{blue},
      numberstyle=\tiny\color{blue},
      stringstyle=\color{mymauve},
      numbers=left,
      stepnumber=1,
      columns=fullflexible,
      breaklines=true,
      postbreak=\mbox{\textcolor{red}{\ensuremath{\hookrightarrow}\space}},
      literate={~} {\textapprox}{1},
      language={[11]C++}
}

\lstnewenvironment{cmdline}
  {\lstset{
      basicstyle=\ttfamily\scriptsize,
      keywordstyle=\color{blue},
      backgroundcolor=\color{light-gray},
      language={bash}
  }}
  {}

\lstnewenvironment{cmdlinelarge}
  {\lstset{
      basicstyle=\ttfamily\small,
      keywordstyle=\color{blue},
      backgroundcolor=\color{light-gray},
      language={bash}
  }}
  {}

\makeatletter
\newcommand{\srcmediumsize}{\@setfontsize{\srcmediumsize}{7pt}{7pt}}
\makeatother

\makeatletter
\newcommand{\srcbigsize}{\@setfontsize{\srcbigsize}{8pt}{8pt}}
\makeatother

\makeatletter
\newcommand{\srcsize}{\@setfontsize{\srcsize}{6pt}{6pt}}
\makeatother

\makeatletter
\newcommand{\srcsmallsize}{\@setfontsize{\srcsmallsize}{5pt}{5pt}}
\makeatother

\makeatletter
\newenvironment<>{btHighlight}[1][]
{\begin{onlyenv}#2\begingroup\tikzset{bt@Highlight@par/.style={#1}}\begin{lrbox}{\@tempboxa}}
{\end{lrbox}\bt@HL@box[bt@Highlight@par]{\@tempboxa}\endgroup\end{onlyenv}}

\newcommand<>\btHL[1][]{%
  \only#2{\begin{btHighlight}[#1]\bgroup\aftergroup\bt@HL@endenv}%
}
\def\bt@HL@endenv{%
  \end{btHighlight}%
  \egroup
}
\newcommand{\bt@HL@box}[2][]{%
  \tikz[#1]{%
    \pgfpathrectangle{\pgfpoint{1pt}{0pt}}{\pgfpoint{\wd #2}{\ht #2}}%
    \pgfusepath{use as bounding box}%
    \node[anchor=base west, fill=orange!30,outer sep=0pt,inner xsep=1pt, inner ysep=0pt, rounded corners=3pt, minimum height=\ht\strutbox+1pt,#1]{\raisebox{1pt}{\strut}\strut\usebox{#2}};
  }%
}
\makeatother

\title[C++]
{Программирование на языке C++}
 
\subtitle{Вводный курс}
 
\author[А.~Б.~Морозов]
{
  \texorpdfstring{Александр Морозов\newline\href{mailto:gelu.speculum@gmail.com}{gelu.speculum@gmail.com}}
  {Александр Морозов}
}
  
\date[ITMO 2020]
{ИТМО, весенний семестр 2020}
 
\logo{%
  \makebox[0.97\paperwidth]{%
    \includegraphics[align=c,width=2cm,keepaspectratio]{itmo_logo.png}
    \hfill
    \includegraphics[align=c,width=1.5cm,keepaspectratio]{itiviti_logo.png}
  }
}

\AtBeginSection[]
{
  \begin{frame}
    \frametitle{Содержание}
    \tableofcontents[currentsection]
  \end{frame}
}

\begin{document}
 
\frame{\titlepage}

%-------------------------------------------------
\section{Структуры и классы}

\begin{frame}{Структуры и классы}
  \begin{itemize}
    \item задаёт новый тип в программе \vspace{1em}
    \item может иметь члены:
      \begin{itemize}
        \item поля данных \vspace{0.5em}
        \item методы (функции \vspace{0.5em})
        \item типы \vspace{0.5em}
        \item псевдонимы типов \vspace{0.5em}
        \item члены вложенных нестрогих \lstinline{enum} \vspace{0.5em}
        \item шаблоны \vspace{1em}
      \end{itemize}
    \item можно наследовать от других классов
  \end{itemize}
\end{frame}

\begin{frame}[fragile]{Определение}
  \begin{itemize}
    \item \emph{класс имя} \lstinline|{| \emph{члены} \lstinline|}|
    \item \emph{класс имя} \lstinline{:} \emph{список наследования} \lstinline|{| \emph{члены} \lstinline|}|
  \end{itemize}

  \vspace{1em}

  где
  \begin{itemize}
    \item \emph{класс} -- это ключевое слово \lstinline{class} или \lstinline{struct} \vspace{0.5em}
    \item \emph{имя} -- идентификатор \vspace{0.5em}
    \item \emph{члены} -- объявление членов класса \vspace{0.5em}
    \item \emph{список наследования} -- \emph{base1, ..., baseN}
  \end{itemize}
\end{frame}

\begin{frame}[fragile]{Примеры определений классов}
  \begin{lstlisting}[basicstyle=\ttfamily\srcbigsize]
    struct A {};

    struct B
    {
      A a;
      int b;
      char c;
    } b1, b2;

    class C
    {
      const std::size_t n = 0;
    public:
      std::size_t value() const
      { return n; }
    };

    int main()
    {
      A a;
      C c = C();
      std::cout << sizeof(a) << " : " << sizeof(b1) << " : " << sizeof(c) << std::endl;
      std::cout << C().value() << std::endl;
    }
  \end{lstlisting}
\end{frame}

\begin{frame}[fragile]{Область видимости класса}
  \begin{lstlisting}[moredelim={**[is][\btHL<1>]{@1}{@}}]
    struct S {
      struct T {
        int z = @110 * n@;
        int baz();
      };
      int foo(int x) noexcept(@1n > 100@) {
        @1return x * n;@
      }
      void bar(int y = @1n@);
      int z = @13 * n@;
      static const int n = 101;
      @1
      char str[n];
    @};
    int S::T::baz()
    { @1return z * 3;@ }
    void S::bar(int y)
    { @1z *= y;@ }
  \end{lstlisting}
\end{frame}

%-------------------------------------------------
\section{Enum}

%-------------------------------------------------
\section{Полные и неполные типы}

\begin{frame}[fragile]{Полные и неполные типы}
  Полный тип -- определение известно.

  \vspace{1em}

  Неполный тип:
  \begin{itemize}
    \item предварительное объявление класса \vspace{0.5em}
    \item \lstinline{enum} до момента определения его типа реализации \vspace{0.5em}
    \item \lstinline{void} \vspace{0.5em}
    \item ...
  \end{itemize}
\end{frame}

\begin{frame}[fragile]{Необходимость полного типа}
  Требуется знать полный тип \lstinline{T} к моменту:

  \vspace{1em}
  \begin{itemize}
    \item вызов функции, возвращающей \lstinline{T} \vspace{0.5em}
    \item объявление переменной типа \lstinline{T} \vspace{0.5em}
    \item объявление нестатического поля типа \lstinline{T} \vspace{0.5em}
    \item явное или неявное преобразование к \lstinline{T} \vspace{0.5em}
    \item обращение к членам класса типа \lstinline{T} \vspace{0.5em}
    \item использование \lstinline{T} как родителя объявляемого класса \vspace{0.5em}
    \item ...
  \end{itemize}
\end{frame}

\begin{frame}[fragile]{Примеры неполных типов}
  \begin{lstlisting}[basicstyle=\ttfamily\srcbigsize]
    class C;

    void f(const C *); // OK
    C g(); // OK

    auto c = g(); // error
    auto x = C::get(); // error
    C cc; // error

    struct S
    {
      C & c; // OK;
      C cc; // error
    };

    enum E {
      A,
      B = sizeof(E), // error
      C
    };

    enum class EE {
      A,
      B = sizeof(EE), // OK
      C
    };
  \end{lstlisting}
\end{frame}

\begin{frame}[fragile]{Opaque enum declaration}
  \begin{lstlisting}
    enum A : int;

    enum class B;

    enum class C : unsigned;

    A get_a()
    {
      return static_cast<A>(10); // potential UB
    }

    B a_to_b(const A a)
    {
      return static_cast<B>(a); // potential UB
    }
  \end{lstlisting}
\end{frame}

%-------------------------------------------------
\section{Некоторые классы стандартной библиотеки}


\end{document}
