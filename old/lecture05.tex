\documentclass[unknownkeysallowed,xcolor=table]{beamer}
 
\usepackage [T2A,T1] {fontenc}
\usepackage[utf8]{inputenc}
\usepackage[english,russian]{babel}
\usepackage{amsmath}
\usepackage{listings}
\usepackage{url}
\usepackage{textcomp}
\usepackage{multirow}

\setbeamertemplate{navigation symbols}{}

\newcommand{\textapprox}{\raisebox{0.5ex}{\texttildelow}}

\lstnewenvironment{cmdline}
  {\lstset{basicstyle=\ttfamily\scriptsize,keywordstyle=\color{blue},
           language={bash}}}
  {}
  
\colorlet{mygreen}{green!60!blue}
\colorlet{mymauve}{red!60!blue}
\definecolor{pblue}{rgb}{0.1, 0.2, 0.8}

\newcommand{\rarr}{$\rightarrow$}

\lstset{
      basicstyle=\ttfamily\small,
      commentstyle=\color{mygreen},
      keywordstyle=\color{blue},
      numberstyle=\tiny\color{blue},
      stringstyle=\color{mymauve},
      columns=fullflexible,
      breaklines=true,
      postbreak=\mbox{\textcolor{red}{\ensuremath{\hookrightarrow}\space}},
      literate={~} {\textapprox}{1},
      language={[11]C++}
}

\makeatletter
\newcommand{\srcmediumsize}{\@setfontsize{\srcmediumsize}{7pt}{7pt}}
\makeatother

\makeatletter
\newcommand{\srcbigsize}{\@setfontsize{\srcbigsize}{8pt}{8pt}}
\makeatother

\makeatletter
\newcommand{\srcsize}{\@setfontsize{\srcsize}{6pt}{6pt}}
\makeatother

\makeatletter
\newcommand{\srcsmallsize}{\@setfontsize{\srcsmallsize}{5pt}{5pt}}
\makeatother

\title[C++]
{Программирование на языке C++}
 
\subtitle{Вводный курс}
 
\author[А.~Б.~Морозов]
{
  \texorpdfstring{Александр Морозов\newline\href{mailto:gelu.speculum@gmail.com}{gelu.speculum@gmail.com}}
  {Александр Морозов}
}
  
\date[ITMO 2019]
{ИТМО, весенний семестр 2019}
 
\logo{%
  \makebox[0.97\paperwidth]{%
    \includegraphics[align=c,width=2cm,keepaspectratio]{itmo_logo.png}
    \hfill
    \includegraphics[align=c,width=1.5cm,keepaspectratio]{new_itiviti_logo.png}
  }
}

\AtBeginSection[]
{
  \begin{frame}
    \frametitle{Содержание}
    \tableofcontents[currentsection]
  \end{frame}
}

\begin{document}

\frame{\titlepage}

%-------------------------------------------------

\section{Указатели и ссылки}

\begin{frame}[fragile]{Указатели}

Указатель -- это специальный тип, производный от другого типа, чьё значение является одним из:

\vspace{0.7em}

\begin{itemize}
  \item указателем на объект или функцию (представляет собой адрес в памяти функции или первого байта объекта) \vspace{0.5em}
  \item указателем на после объекта (представляет адрес в памяти на первый байт, следующий за объектом) \vspace{0.5em}
  \item нулевым указателем \vspace{0.5em}
  \item неопределенным
\end{itemize}

\vspace{1.5em}

\emph{специф-ия типа [имя класса]} \lstinline{*} \emph{[cv-квалиф-ор] декларатор} 

\end{frame}

\begin{frame}[fragile]{Примеры объявлений}

\begin{lstlisting}
int * a;

const double * b;

char ** c;

void (*d)(int);

int A::* e;

void (B::* f)(int);
\end{lstlisting}

\end{frame}

\begin{frame}{Реализация указателей}

Размер указателя -- зависит от платформы/реализации.

\vspace{2em}

С т.з. языка значение указателя и адрес в памяти, которое оно представляет -- не эквивалентные вещи.

\vspace{2em}

Указатель связан не с конкретным объектом, а с местом его хранения. Период существования места хранения определяется типом размещения (автоматический, статический, тред-локальный, динамический), за пределами периода существования места хранения значение указателя на него является неопределенным.

\end{frame}

\begin{frame}{Значения указателей}

Разыменование или освобождение неопределенного указателя является UB, любое другое использование неопределенного указателя является поведением, зависящим от реализации.

\vspace{2em}

Нулевой указатель - это специальное значение указателям конкретного типа, служащее для указания отсутствия сущности, на которую указатель ссылается. Разыменование нулевого указателя -- UB.

\end{frame}

\begin{frame}[fragile]{std::nullptr\_t}

\begin{center}
\lstinline{std::nullptr_t} $\Leftrightarrow$ \lstinline{nullptr}
\end{center}

\vspace{1em}

\begin{itemize}
  \item \lstinline{std::nullptr_t} -- не является указателем
  \item \lstinline{nullptr} -- неявно приводится к нулевому указателю любого типа
\end{itemize}

\begin{lstlisting}
int * x = nullptr; // null pointer to int
const C * c = nullptr; // null pointer to const C
void * u = nullptr; // null pointer to void
\end{lstlisting}

\end{frame}

\begin{frame}[fragile]{Указатели на объекты}
Значение указателя на объект можно получить с помощью оператора взятия адреса:

\begin{lstlisting}
A a;
A * p_a = &a;
A ** pp_a = &p_a;
struct S { int n; }
S s;
int * p_i = &s.n;
\end{lstlisting}

Разыменование указателя: оператор разыменования \lstinline{*} возвращает lvalue выражение, идентифицирующее объект, на который указывает разыменованный указатель.

\begin{lstlisting}
A a;
A * p_a = &a;
*p_a = A();
\end{lstlisting}

\end{frame}

\begin{frame}[fragile]{Разыменование указателей сложных типов}

Оператор \lstinline{->} используется для упрощения доступа к членам сложных типов через указатель на значение типа и эквивалентен последовательному разыменованию и затем доступу к элементу:

\vspace{1em}

\begin{lstlisting}
A a;
A * p_a = &a;

p_a->foo();
(*p_a).foo();
a.foo();
\end{lstlisting}

\end{frame}

\begin{frame}[fragile]{Указатели на экземпляры классов и наследование}

\begin{lstlisting}
class A {};
class B {};
class C : public A, public B {};

C c;
A * a = &c;
B * b = &c;
C * c_from_a = static_cast<C *>(a);
C * c_from_b = static_cast<C *>(b);
\end{lstlisting}

\end{frame}

\begin{frame}[fragile]{Указатели на экземпляры классов и наследование, ограничения}

\begin{lstlisting}
class D : A, B {};
class E : public A, public C {};
class F : public B {};

D d;
A * a_from_d = &d; // compilation error

E e;
C * c_from_e = &e;
A * a_from_e = &e; // compilation error

C c;
B * b_from_c = &c;
F * f_from_b = static_cast<F *>(b_from_c); // wrong, but compiles
\end{lstlisting}

\end{frame}

\begin{frame}[fragile]{Безтиповые указатели}

Возможен указатель на \lstinline{void} (в т.ч. cv-квалифицированный). Это безтиповый указатель, к которому неявно приводятся обычные указатели. Обратная операция приведения должна быть явной (\lstinline{static_cast}).

\vspace{2em}

\begin{lstlisting}
struct S {};
S s;
const void * p = &s;
const S * ps = static_cast<const S *>(p);
\end{lstlisting}

\end{frame}

\begin{frame}[fragile]{Указатели на функции}

Выражение, идентифицирующее функцию, неявно приводится к указателю на функцию. Указатель на функцию может непосредственно являться левым операндом оператора функционального вызова.

\vspace{1em}

\begin{lstlisting}
double foo(int, char);
double (*p) (int, char) = foo;
double (*pp) (int, char) = &foo;

assert(p == pp);
p(x, y);
(*p)(x, y);
\end{lstlisting}

\end{frame}

\begin{frame}[fragile]{Указатели на нестатические методы}

\begin{lstlisting}
struct S {
    bool check(int, char);
};

bool (S::* p) (int, char) = *S::check;

int i = 0;
char c = 'a';
S s, *ps = &s;
bool x = (s.*p) (i, c);
bool y = (ps->*p)(i, c);
\end{lstlisting}

\end{frame}

\begin{frame}[fragile]{Указатели на нестатические методы, продолжение}

\begin{lstlisting}
struct A {
    int get_n(char);
};
struct B : A {
    bool empty();
};
int (A::*pa) (char) = &A::get_n;
int (B::*pb) (char) = pa;
A a;
B b;
(a.*pa)('a');
(b.*pb)('b');
bool (B::*eb) () = &B::empty;
bool (A::*ea) () = static_cast<bool (A::*) ()>(eb);
(b.*ea) (); // OK
(a.*ea) (); // UB
\end{lstlisting}

\end{frame}

\begin{frame}[fragile]{Указатели на нестатические поля}

Аналогично указателям на нестатические методы (включая вопросы наследования):

\vspace{0.7em}

\begin{lstlisting}
struct A { int n; };

int A::* p = &A::n;
A a{101};
A * pa = &a;

int x = a.*p; // 101
int y = pa->*p; // 101
\end{lstlisting}

\end{frame}

\begin{frame}[fragile]{Массивы}

Массивы -- агрегаты, состоящие из конечного числа элементов одного типа.
Занимают непрерывную область в памяти.
Элементы массива считаются его подобъектами.

\vspace{0.5em}

\begin{lstlisting}
int x[10];
const char str[] = "Hello"; // implicit size
\end{lstlisting}

\vspace{1em}

Элементы массива нумеруются с 0, доступ к ним возможен через оператор индекса:

\vspace{0.5em}

\begin{lstlisting}
for (int i = 0; i < 10; ++i) {
    x[i] = i;
}
\end{lstlisting}

\vspace{0.5em}

Массив неявно преобразуется к указателю на первый элемент массива.

\end{frame}

\begin{frame}[fragile]{Ограничения массивов}

Массив, как целое -- иммутабелен, хотя его элементы можно менять, если их тип -- не \lstinline{const}.

\vspace{1.5em}

Массивы неопределенной длины являются неполным типом:

\vspace{1em}

\begin{lstlisting}
extern int x[];
\end{lstlisting}

\vspace{1em}

Массивы нельзя возвращать из функций.

\vspace{1em}

Размер объекта массива:

\begin{lstlisting}
int x[10];
size_t n = sizeof(x) / sizeof(x[0]); // 10
\end{lstlisting}

\end{frame}

\begin{frame}{Арифметика указателей}

Указатели могут выступать операндами некоторых арифметических операций, при этом указатель на отдельный объект считается указателем на элемент массива единичной длины.

\vspace{1em}

Допустимые операции:

\vspace{0.5em}

\begin{itemize}
  \item сложение -- указатель и целое число (в любом порядке)
  \item вычитание -- указатель (слева) и целое число
  \item вычитание двух указателей одного типа
  \item инкремент
  \item декремент
\end{itemize}

\end{frame}

\begin{frame}[fragile]{Правила арифметики указателей: указатель и число}

Если $P$ -- указатель на $i$ элемент массива $x$, то

\vspace{0.5em}

$P + n \Leftrightarrow x[i + n]$ \\ \vspace{0.5em}
$n + P \Leftrightarrow x[i + n]$ \\ \vspace{0.5em}
$P - n \Leftrightarrow x[i-n]$

\vspace{1em}

Результат $x[j]$ -- должен быть корректным элементом массива, либо элементом после последнего.

\end{frame}

\begin{frame}[fragile]{Правила арифметики указателей: вычитание указателей}

Оба указателя должны быть одного типа не считая cv-квалификаторов.

\vspace{1em}

Если $P$ -- указатель на $i$ элемент массива $x$, а $Q$ -- указатель на $j$ элемент массива $x$, то

\vspace{1em}

$P - Q = i - j$

\vspace{1em}

Результирующее значение имеет тип \lstinline{std::ptrdiff_t}.

\vspace{2em}

Любые другие арифметические операции над указателями -- UB.

\end{frame}

\begin{frame}[fragile]{Ссылки}

Ссылка -- это ассоциация (``связывание'') имени с каким-либо объектом или функцией, псевдоним.
Реализация ссылок в стандарте не оговорена.

\vspace{2em}

Объявление:

\vspace{0.7em}

\emph{спецификация типа} \lstinline{&} \emph{декларатор}

\vspace{0.5em}

\emph{спецификация типа} \lstinline{&&} \emph{декларатор}

\vspace{1em}

\begin{lstlisting}
int x;
int & rx = x;
int && rrx = 55;
\end{lstlisting}

\end{frame}

\begin{frame}[fragile]{Ограничения ссылок}

\begin{itemize}
  \item обязательная инициализация \vspace{0.5em}
  \item иммутабельность \vspace{0.5em}
  \item не является объектом и не имеет своего значения \vspace{0.5em}
  \item всегда указывает на какой-то объект или функцию \vspace{0.5em}
  \item невозможность cv-квалификации \vspace{0.5em}
  \item необязательность наличия размера \vspace{0.5em}
  \item время жизни эквивалентно области видимости
\end{itemize}

\vspace{0.5em}

Если время жизни объекта закончится раньше времени жизни ссылки, ссылка становится ``висящей'' и её использование является UB.

\end{frame}

\begin{frame}[fragile]{Ограничения ссылок, примеры}

\begin{lstlisting}
X & x; // error
void & v; // error
X & x[3]; // error
X &* px; // error
X & & xx; // error

X & foo() { X x; return x; }
X & x = foo(); // dangling reference
X y = x; // UB
\end{lstlisting}

\end{frame}

\begin{frame}[fragile]{Типы ссылок и их связывание}

\begin{itemize}
  \item lvalue и rvalue \vspace{0.5em}
  \item \lstinline{const} и не-\lstinline{const}
\end{itemize}

\vspace{1em}

\begin{itemize}
  \item lvalue ссылка связывается с lvalue выражением \vspace{0.5em}
  \item \lstinline{const} lvalue ссылка связывается с lvalue или xvalue \vspace{0.5em}
  \item rvalue ссылка связывается с xvalue
\end{itemize}

\vspace{1em}

\lstinline{auto &&} -- будет выведена в самую подходящую ссылку.

\end{frame}

\begin{frame}[fragile]{Продление времени жизни временного объекта}

Если временный объект или его подобъект связывается со ссылкой, его время жизни продлевается на время жизни ссылки.

\vspace{2em}

Исключения:
\begin{itemize}
  \item временный объект в инструкции return, если функция возвращает ссылку, разрушается в конце исполнения инструкции \vspace{1em}
  \item временный объект, связывающийся со ссылкой внутри выражения (например, к ссылочному параметру функции в выражении её вызова), разрушается в конце исполнения полного выражения.
\end{itemize}

\end{frame}

\begin{frame}[fragile]{Продление времени жизни временного объекта, примеры}

\begin{lstlisting}
const int & a = 1 + 2; // extended
int && a = 2 + 3; // extended
const int & l = std::array<int, 3>{1, 2, 3}[1]; // extended

const double & bar() {
    return 0.5 * 1.5;
}
const double & d = bar(); // dangling reference

const std::string & pass(const std::string & s) {
    return s; // OK
}
const std::string & s = pass("hello");
std::cout << s; // error, dangling reference
\end{lstlisting}

\end{frame}

\begin{frame}{Схлопывание ссылок}

Допускается путём манипуляций над псевдонимами типов и шаблонами получать ссылку на ссылку, в этом случае применяются правила ``схлопывания'' ссылок (reference collapsing) и результатом всегда является обычная ссылка:

\vspace{2em}

\begin{center}
\begin{tabular}{ c c | c }
  \hline
    \multicolumn{2}{c|}{$+$} & $=$ \\
     \& & \& & \& \\
     \& & \&\& & \& \\
     \&\& & \& & \& \\
     \&\& & \&\& & \&\& \\
\end{tabular}
\end{center}

\end{frame}

\begin{frame}[fragile]{Пример схлопывания ссылок}

\begin{lstlisting}
using X = T &;
using Y = T &&;

T t;

X & x = t; // decltype(x) == T &
X && xx = t; // decltype(xx) == T &
Y & y = t; // decltype(y) == T &
Y && yy = std::move(t); // decltype(yy) == T &&
\end{lstlisting}

\end{frame}

\begin{frame}[fragile]{Реализация ссылок}

Реализация ссылок не определена стандартом. Ссылки не обязаны существовать в результате трансляции программы.

\vspace{1em}

Однако, можно ожидать, что:

\vspace{1em}

\begin{lstlisting}
struct A {
    int * x;
};

struct B {
    int & x;
};

static_assert(sizeof(A) == sizeof(B));
\end{lstlisting}

\end{frame}

\begin{frame}[fragile]{Квалификаторы и указатели/ссылки}

cv-квалификатор можно ставить как слева, так и справа от того, на что он действует.

\vspace{1em}

\begin{lstlisting}
const int x = 10;
int const x = 10;

const char * str; // points to a const object
char const * str; // pointer itself is non-const

char * const str_c; // points to a non-const object
using X = char *;
const X str_c;  // pointer itself is const
X const str_c;

const std::string & s; // reference a const object
std::string const & s; // reference itself is inherently immutable

\end{lstlisting}

\end{frame}

\begin{frame}[fragile]{Квалификаторы и многоуровневые указатели}

\begin{lstlisting}

int *** x;
int const *** y;
int * const ** z;
int ** const * w;
int *** const v;

using X = const int;
using Y = const X *;
using Z = const Y *;
const Z * p;
int const * const * const * p;

\end{lstlisting}

\end{frame}

\begin{frame}[fragile]{Передача объектов}

\begin{itemize}
  \item по значению -- объект копируется
  \item по указателю -- копируется указатель, можно передать нулевой указатель
  \item по ссылке -- передаётся связь с объектом
\end{itemize}

\begin{lstlisting}
struct S {};
void f1(S s) {
    s = S();
}
void f2(const S * ps) {
    if (ps != nullptr) {
        std::cout << *ps << std::endl;
    }
}
void f3(const S & rs) {
    std::cout << rs << std::endl;
}
\end{lstlisting}

\end{frame}

%-------------------------------------------------

\section{Работа с динамической памятью}

\begin{frame}[fragile]{Работа с динамической памятью}

Выделение -- делает доступным для использования участок памяти заданного размера.

\vspace{1em}

Освобождение -- возвращает выделенный ранее участок памяти операционной системе.

\vspace{2em}

\lstinline{malloc}/\lstinline{free} -- функции выделения/освобождения памяти из C.

\vspace{1em}

\lstinline{new}/\lstinline{delete} -- операторы выделения/освобождения памяти в C++.

\end{frame}

\begin{frame}[fragile]{new}

\begin{lstlisting}
X * x = new X;
X * x = ::new X;
X * x = new X();
X * x = new X(a, b, c);
X * x = new X{a, b, c};
X * x = new X[10];
X * x = new X[10]{x1, x2, x3};
X * x = new X[0]; // OK, zero element array

auto p = new (int (*[10])()); // array of function pointers

auto m = new double[n][5][10][100];
auto mm = new double[5][n][10]; // error
\end{lstlisting}

\end{frame}

\begin{frame}[fragile]{Размещающее new}

\begin{lstlisting}
struct S { ... };

std::byte * ptr = new std::byte[sizeof(S)];

S * ps = new (ptr) S(...);

ps->~S();

delete [] ptr;

\end{lstlisting}

\end{frame}

\begin{frame}[fragile]{delete}

\begin{lstlisting}
X * x = new X;
delete x;

X * x = nullptr;
delete x;

X * x = new X[10];
delete [] x;

struct A {};
struct B : A {};
A * a = new B;
delete a;
\end{lstlisting}

\end{frame}

%-------------------------------------------------

\section{Некоторые вспомогательные классы}

\begin{frame}[fragile]{std::string\_view}

Указание на участок памяти, интерпретируемый как последовательность символов. Не позволяет менять данные, на которые указывает.

\vspace{1em}

\begin{lstlisting}
std::string x("Some string");
std::string_view y = x;

if (const auto pos = y.find("me"); pos != y.npos) {
    std::string_view z = y.substr(pos, 4);
    assert(z.size() == 4);
    assert(z == "me s");
}

for (char c : y) {
    std::cout << c;
}
\end{lstlisting}

\end{frame}

\begin{frame}[fragile]{std::reference\_wrapper}

\begin{lstlisting}
void f(S &);
void cf(const S &);

S s;
auto r = std::ref(s);
auto cr = std::cref(s);
f(r);
f(r.get());
cf(cr);

std::vector<std::reference_wrapper<const S>> v;
v.push_back(cr);
v.push_back(cr);
auto vv = v;
\end{lstlisting}

\end{frame}

%-------------------------------------------------

\section{Наглядный пример}

\begin{frame}{Абстракция линейной памяти}

Для прикладных приложений в большинстве случаев применима следующая абстракция памяти:

\vspace{1em}

\begin{itemize}
  \item память линейна \vspace{0.5em}
  \item память полностью принадлежит приложению \vspace{0.5em}
  \item память не ограничена ничем, кроме пределов адресации \vspace{0.5em}
  \item адрес в памяти -- целое число \vspace{0.5em}
  \item конкретные выделяемые адреса произвольны, а выделяемые области -- не связаны
\end{itemize}

\end{frame}

\begin{frame}[fragile]{lru.h}
\lstinputlisting[lastline=32,basicstyle=\ttfamily\srcsize]{examples/lru/lru.h}
\end{frame}

\begin{frame}[fragile]{lru.h, продолжение}
\lstinputlisting[firstline=34,basicstyle=\ttfamily\srcsize]{examples/lru/lru.h}
\end{frame}

\begin{frame}[fragile]{main.cpp, типы}
\lstinputlisting[firstline=8,lastline=28,basicstyle=\ttfamily\srcbigsize]{examples/lru/main.cpp}
\end{frame}

\begin{frame}[fragile]{main.cpp, функция main}
\lstinputlisting[firstline=1,lastline=6,basicstyle=\ttfamily\srcbigsize]{examples/lru/main.cpp}
...
\lstinputlisting[firstline=98,lastline=104,basicstyle=\ttfamily\srcbigsize]{examples/lru/main.cpp}
\end{frame}

\begin{frame}[fragile]{main.cpp, тест}
\lstinputlisting[firstline=30,lastline=39,basicstyle=\ttfamily\small]{examples/lru/main.cpp}
...
\lstinputlisting[firstline=93,lastline=95,basicstyle=\ttfamily\small]{examples/lru/main.cpp}
\end{frame}

\begin{frame}[fragile]{main.cpp, тело теста}
\lstinputlisting[firstline=40,lastline=64,basicstyle=\ttfamily\srcmediumsize]{examples/lru/main.cpp}
...
\end{frame}

\begin{frame}[fragile]{main.cpp, тело теста, продолжение}
\lstinputlisting[firstline=66,lastline=91,basicstyle=\ttfamily\srcmediumsize]{examples/lru/main.cpp}
\end{frame}

\begin{frame}[fragile]{Тест}

\begin{cmdline}
$ ./lru
1[1] 2[2]
9[9@] 8[8@] 7[7@] 6[6@] 5[5@] 4[4@] 3[3@] 2[2@] 1[1@] 0[0@]
9[9@] 8[8@] 7[7@] 6[6@] 5[5@] 4[4@] 3[3@] 2[2@] 1[1@] 0[0@]
29[29!] 28[28!] 27[27!] 26[26!] 25[25!] 24[24!] 23[23!] 22[22!] 21[21!] 20[20!]
26[26!] 23[23!] 29[29!] 28[28!] 27[27!] 25[25!] 24[24!] 22[22!] 21[21!] 20[20!]
20[20!] 21[21!] 22[22!] 23[23!] 24[24!] 25[25!] 26[26!] 27[27!] 28[28!] 29[29!]
19[19] 18[18] 17[17] 16[16] 15[15] 14[14] 13[13] 12[12] 11[11] 10[10]
Test passed: true
\end{cmdline}

\end{frame}

\begin{frame}[fragile]{pool.h}
\lstinputlisting[lastline=21,basicstyle=\ttfamily\srcbigsize]{examples/lru/pool.h}
\end{frame}

\begin{frame}[fragile]{pool.h, класс аллокатора}
\lstinputlisting[firstline=23,lastline=64,basicstyle=\ttfamily\srcsmallsize]{examples/lru/pool.h}
\end{frame}

\begin{frame}[fragile]{pool.h, реализация аллокатора}
\lstinputlisting[firstline=66,lastline=100,basicstyle=\ttfamily\srcsize]{examples/lru/pool.h}
\end{frame}

\begin{frame}[fragile]{pool.cpp}
\lstinputlisting[lastline=38,basicstyle=\ttfamily\srcsmallsize]{examples/lru/pool.cpp}
\end{frame}

\begin{frame}[fragile]{pool.cpp, поиск свободного места}
\lstinputlisting[firstline=40,lastline=61,basicstyle=\ttfamily\srcbigsize]{examples/lru/pool.cpp}
\end{frame}

\begin{frame}[fragile]{pool.cpp, выделение/освобождение}
\lstinputlisting[firstline=63,lastline=89,basicstyle=\ttfamily\srcmediumsize]{examples/lru/pool.cpp}
\end{frame}

\begin{frame}[fragile]{pool.cpp, публичный интерфейс}
\lstinputlisting[firstline=91,basicstyle=\ttfamily\srcmediumsize]{examples/lru/pool.cpp}
\end{frame}

\end{document}