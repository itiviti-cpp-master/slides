\documentclass[unknownkeysallowed,xcolor=table]{beamer}
 
\usepackage [T2A,T1] {fontenc}
\usepackage[utf8]{inputenc}
\usepackage[english,russian]{babel}
\usepackage{amsmath}
\usepackage{listings}
\usepackage{url}
\usepackage{textcomp}
\usepackage{multirow}

\setbeamertemplate{navigation symbols}{}

\newcommand{\textapprox}{\raisebox{0.5ex}{\texttildelow}}

\lstnewenvironment{cmdline}
  {\lstset{basicstyle=\ttfamily\scriptsize,keywordstyle=\color{blue},
           language={bash}}}
  {}
  
\lstnewenvironment{cmdlinelarge}
  {\lstset{basicstyle=\ttfamily\small,keywordstyle=\color{blue},
           language={bash}}}
  {}
  
\colorlet{mygreen}{green!60!blue}
\colorlet{mymauve}{red!60!blue}
\definecolor{pblue}{rgb}{0.1, 0.2, 0.8}

\newcommand{\rarr}{$\rightarrow$}

\lstset{
      basicstyle=\ttfamily\small,
      commentstyle=\color{mygreen},
      keywordstyle=\color{blue},
      numberstyle=\tiny\color{blue},
      stringstyle=\color{mymauve},
      numbers=left,
      stepnumber=1,
      columns=fullflexible,
      breaklines=true,
      postbreak=\mbox{\textcolor{red}{\ensuremath{\hookrightarrow}\space}},
      literate={~} {\textapprox}{1},
      language={[11]C++}
}

\makeatletter
\newcommand{\srcmediumsize}{\@setfontsize{\srcmediumsize}{7pt}{7pt}}
\makeatother

\makeatletter
\newcommand{\srcbigsize}{\@setfontsize{\srcbigsize}{8pt}{8pt}}
\makeatother

\makeatletter
\newcommand{\srcsize}{\@setfontsize{\srcsize}{6pt}{6pt}}
\makeatother

\makeatletter
\newcommand{\srcsmallsize}{\@setfontsize{\srcsmallsize}{5pt}{5pt}}
\makeatother

\newcommand{\rot}[1]{\makebox[1em][l]{\rotatebox{45}{#1}}}

\title[C++]
{Программирование на языке C++}
 
\subtitle{Вводный курс}
 
\author[А.~Б.~Морозов]
{
  \texorpdfstring{Александр Морозов\newline\href{mailto:gelu.speculum@gmail.com}{gelu.speculum@gmail.com}}
  {Александр Морозов}
}
  
\date[ITMO 2019]
{ИТМО, весенний семестр 2019}
 
\logo{%
  \makebox[0.97\paperwidth]{%
    \includegraphics[align=c,width=2cm,keepaspectratio]{itmo_logo.png}
    \hfill
    \includegraphics[align=c,width=1.5cm,keepaspectratio]{new_itiviti_logo.png}
  }
}

\AtBeginSection[]
{
  \begin{frame}
    \frametitle{Содержание}
    \tableofcontents[currentsection]
  \end{frame}
}

\begin{document}

\frame{\titlepage}

%-------------------------------------------------

\section{Пространства имён}

\begin{frame}[fragile]{Пространства имён}

Позволяют использовать одинаковые имена для разных сущностей в разных частях программы.

\begin{lstlisting}
namespace A {
    void f(int);
    double g();
}
namespace B {
    bool f(const std::string &);
    void g();
}
int main()
{
    int i = 101;
    A::f(i);
    double x = A::g();
    B::g();
}
\end{lstlisting}

\end{frame}

\begin{frame}[fragile]{Пространства имён и определения}
\begin{lstlisting}
namespace A {
    int f();
    void g();
    double sqrt(double x)
    {
        ...
    }
}

int A::f()
{ ... }

void A::g()
{ ... }
\end{lstlisting}
\end{frame}

\begin{frame}[fragile]{Вложенные пространства имён}
\begin{lstlisting}
namespace A {
    namespace B {
        namespace C {
            class C {...};
        }
    }
}
namespace A::B::C {
    struct S {...};
}

int main()
{
    A::B::C::C c;
    A::B::C::S s;
}
\end{lstlisting}
\end{frame}

\begin{frame}[fragile]{Анонимные пространства имён}
\begin{lstlisting}
namespace {
    class C {...};
    C c;
    void f() {...}
}
namespace A {
    namespace {
        int g() {...};
    }
}

int main()
{
    C cc = c;
    f();
    return A::g();
}
\end{lstlisting}
\end{frame}

\begin{frame}[fragile]{Полностью квалифицированные имена}
\begin{lstlisting}
namespace U {
    namespace X {
        void f();
    }
    namespace Y {
        namespace X {
            double f(int);
        }
        void g(int n) {
            double x = X::f(n);
            ::U::X::f();
        }
    }
}
\end{lstlisting}
\end{frame}

\begin{frame}[fragile]{Псевдонимы пространств имён}
\begin{lstlisting}
namespace A::B::C::D::E {
    class X {...};
}
namespace abcde = A::B::C::D::E;

int main()
{
    abcde::X x;
}
\end{lstlisting}
\end{frame}

\begin{frame}[fragile]{using директива и объявление}
\begin{lstlisting}
namespace A {
    using namespace std; // using directive
    
    void say_hi()
    {
        cout << "Hi!" << endl;
    }
}
namespace B {
    using std::string; // using declaration
    
    size_t strlen(const string & s)
    {
        return s.size();
    }
}
\end{lstlisting}
\end{frame}

\begin{frame}[fragile]{Особенности using}
\begin{lstlisting}
class X {};

void swap(X &, X &) {}

namespace N {
    void swap(X &, X &) {}
}

struct Y { X x; };

void swap(Y & l, Y & r) {
    using std::swap;
    using namespace N; // N::swap is hidden
    swap(l.x, r.x); // ::swap is found by ADL
}
\end{lstlisting}
\end{frame}

\begin{frame}{Добавление в пространство std}
В общем случае добавление в пространство \lstinline{std} запрещено и является UB.

\vspace{1em}

Исключение: специализация шаблонных классов из пространства \lstinline{std}, которая зависит хотя бы от одного пользовательского типа.

\vspace{1em}

Ограничения добавления специализаций:
\begin{itemize}
  \item полная специализация метода класса из \lstinline{std} -- UB
  \item полная или частичная специализация класса, являющегося вложенным в класс из \lstinline{std} -- UB
\end{itemize}
\end{frame}

\begin{frame}[fragile]{Пример правильного расширения std}
\begin{lstlisting}[basicstyle=\ttfamily\srcbigsize]
struct S
{
    std::string str;
};

namespace std {
    template <>
    struct hash<S> : hash<std::string> {
        using argument_type = S;
        using result_type = size_t;
        size_t operator() (const S & s) const noexcept {
            return hash<std::string>::operator() (s.str);
        }
    };
}

void f() {
    std::unordered_set<S> set;
    set.insert("foo");
    set.insert("foo");
    set.insert("bar");
    assert(set.size() == 2);
}
\end{lstlisting}
\end{frame}

%-------------------------------------------------

\section{Поиск имён}

\begin{frame}{Виды поиска имён}
\begin{itemize}
  \item Квалифицированный \vspace{2em}
  \item Неквалифицированный \vspace{2em}
  \item ADL
\end{itemize}
\end{frame}

\begin{frame}{Квалифицированный поиск имён}
Квалифицированное имя -- имя справа от оператора \lstinline{::}.

\vspace{1em}

Может относиться к:

\vspace{0.5em}

\begin{itemize}
  \item члену класса \vspace{0.5em}
  \item сущности из пространства имён \vspace{0.5em}
  \item элементу \lstinline{enum}
\end{itemize}

\vspace{1em}

Слева от оператора \lstinline{::} -- либо ничего, либо уже найденное имя (класса, пространства имён или \lstinline{enum}).
\end{frame}

\begin{frame}[fragile]{Примеры квалифицированного поиска имён}
\begin{lstlisting}
namespace N {   
    struct A {
        int f() const { return 10; }
    };
    struct B : A {
        int f() const { return 15; }
        
        enum class E { A, B, C };
    };
}

void g() {
    std::cout << N::B{}.f() << std::endl;
    std::cout << N::B{}.A::f() << std::endl;
    std::cout << static_cast<int>(N::B::E::A) << std::endl;
}
\end{lstlisting}
\end{frame}

\begin{frame}[fragile]{Поиск неквалифицированных имён вместе с квалифицированными}
\begin{lstlisting}
namespace A {
    int f(int);
    const int n = -13;
}
int A::f(int i = n)
{ return i; }

struct X {
    struct Y {};
    static constexpr size_t N = 100;
    static Y arr[N];
};
Y X::arr[N]; // error: unknown type Y
X::Y X::arr[N]; // OK

const int x = A::f(); // -13
\end{lstlisting}
\end{frame}

\begin{frame}[fragile]{Приоритет поиска в пространствах имён}
\begin{lstlisting}
namespace A {
    void f();
    void g();
}
namespace B {
    using namespace A;
    void f();
}

int main() {
    B::f(); // 'f' from 'B'
    B::g(); // 'g' from 'A'
}
\end{lstlisting}
\end{frame}

\begin{frame}[fragile]{Elaborated type specifier}
\begin{lstlisting}
class X {
public:
    struct Y {};
private:
    int Y;
};

void f() {
    char X;
    X x; // error, 'X' refers to variable X
    class X xx; // OK
    X::Y y; // error, 'X::Y' refers to private member 'Y'
    class X::Y yy; // OK
}
\end{lstlisting}
\end{frame}

\begin{frame}{Неквалифицированный поиск}
Если имя не стоит справа от \lstinline{::} применяется неквалифицированный поиск.

\vspace{1em}

При таком поиске будут анализироваться только подходящие сущности (например, для имени слева от \lstinline{::} -- только пространства имён, классы и \lstinline{enum}).

\vspace{1em}

Такой поиск обрабатывает \lstinline{using} директивы, как если бы содержимое соответствующего пространства имён находилось в ближайшем окружающем пространстве имён.

\vspace{1em}

Поиск будет идти вплоть до обнаружения хотя бы одной сущности с таким именем или до исчерпания вариантов.
\end{frame}

\begin{frame}[fragile]{ADL: пример}
\begin{lstlisting}
std::cout << x << std::endl;
operator << (std::cout, x);
operator << (std::cout, std::endl);

std::cout << endl; // error: 'endl' identifier not found
endl(std::cout); // OK
\end{lstlisting}
\end{frame}

\begin{frame}[fragile]{ADL: пример посложнее}
\begin{lstlisting}[basicstyle=\ttfamily\srcbigsize]
namespace adl {
    struct X {
        int n = 10;
        X & operator + (const X & other) {
            n += other.n;
            return *this;
        }
    };
    X operator + (const X & lhs, const int rhs)
    {
        X x = lhs;
        x.n += rhs;
        return x;
    }
    enum class E { A, B, C };
    std::ostream & operator << (std::ostream & strm, const E & e)
    { return strm << static_cast<int>(e); }
}

int main() {
    adl::X x1, x2 = adl::X{} + 10;
    std::cout << (x1 + x2).n << std::endl;
    adl::E e = adl::E::B;
    std::cout << e << std::endl;
}
\end{lstlisting}
\end{frame}

\begin{frame}{ADL: правила}
Применяется для имени из выражения вызова функции.

\vspace{1em}

Не производится, если неквалифицированный поиск дал:
\begin{itemize}
  \item член класса
  \item объявление функции на уровне блока (не считая \lstinline{using} объявления)
  \item объявление сущности, не являющееся функцией
\end{itemize}

\vspace{1em}

Иначе каждый аргумент функционального вызова добавляет пространства имён и/или классы во множество, в котором затем будет производиться поиск (помимо обычного неквалифицированного поиска).

\vspace{1em}

В процессе поиска отбираются только объявления функций.
\end{frame}

\begin{frame}{ADL: правила для различных типов аргументов}

В зависимости от типа аргумента к поиска добавляется:

\begin{itemize}
  \item базовый тип -- ничего
  \item класс -- сам класс, его предки, его область определения (окружающий класс или пространство имён)
  \item указатель на функцию -- анализ применяется для типа возвращаемого значения и типов всех параметров
  \item шаблон -- в дополнение к обычным правилам, анализ применяется для каждого шаблонного параметра-типа
  \item \lstinline{enum} -- область определения \lstinline{enum}
  \item \lstinline{T *} -- анализ применяется для \lstinline{T}
  \item указатель на метод -- анализ применяется для класса, типа возвращаемого значения и всех параметров
  \item указатель на поле -- анализ применяется для типа поля и класса
\end{itemize}
\end{frame}

\end{document}