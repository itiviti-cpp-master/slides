\documentclass[unknownkeysallowed,xcolor=table]{beamer}
 
\usepackage [T2A,T1] {fontenc}
\usepackage[utf8]{inputenc}
\usepackage[english,russian]{babel}
\usepackage{amsmath}
\usepackage{listings}
\usepackage{url}
\usepackage{textcomp}
\usepackage{multirow}

\setbeamertemplate{navigation symbols}{}

\newcommand{\textapprox}{\raisebox{0.5ex}{\texttildelow}}

\lstnewenvironment{cmdline}
  {\lstset{basicstyle=\ttfamily,keywordstyle=\color{blue},
           language={bash}}}
  {}
  
\colorlet{mygreen}{green!60!blue}
\colorlet{mymauve}{red!60!blue}
\definecolor{pblue}{rgb}{0.1, 0.2, 0.8}

\newcommand{\rarr}{$\rightarrow$}

\lstset{
      basicstyle=\ttfamily\small,
      commentstyle=\color{mygreen},
      keywordstyle=\color{blue},
      numberstyle=\tiny\color{blue},
      stringstyle=\color{mymauve},
      columns=fullflexible,
      breaklines=true,
      postbreak=\mbox{\textcolor{red}{\ensuremath{\hookrightarrow}\space}},
      literate={~} {\textapprox}{1},
      language={[11]C++}
}

\makeatletter
\newcommand{\srcmediumsize}{\@setfontsize{\srcmediumsize}{7pt}{7pt}}
\makeatother

\makeatletter
\newcommand{\srcbigsize}{\@setfontsize{\srcbigsize}{8pt}{8pt}}
\makeatother

\makeatletter
\newcommand{\srcsize}{\@setfontsize{\srcsize}{6pt}{6pt}}
\makeatother

\makeatletter
\newcommand{\srcsmallsize}{\@setfontsize{\srcsmallsize}{5pt}{5pt}}
\makeatother

\title[C++]
{Программирование на языке C++}
 
\subtitle{Вводный курс}
 
\author[А.~Б.~Морозов]
{
  \texorpdfstring{Александр Морозов\newline\href{mailto:gelu.speculum@gmail.com}{gelu.speculum@gmail.com}}
  {Александр Морозов}
}
  
\date[ITMO 2019]
{ИТМО, весенний семестр 2019}
 
\logo{%
  \makebox[0.97\paperwidth]{%
    \includegraphics[align=c,width=2cm,keepaspectratio]{itmo_logo.png}
    \hfill
    \includegraphics[align=c,width=1.5cm,keepaspectratio]{new_itiviti_logo.png}
  }
}

\AtBeginSection[]
{
  \begin{frame}
    \frametitle{Содержание}
    \tableofcontents[currentsection]
  \end{frame}
}

\begin{document}

\frame{\titlepage}

%-------------------------------------------------

\section{Структуры и классы}

\begin{frame}{Структуры и классы}

Классы -- это типы, определенные пользователем. Структура и класс -- практически идентичные понятия.

\vspace{1em}

Каждый класс задаёт уникальный тип, который можно использовать в программе как и любой другой тип.

\vspace{1em}

У класса могут быть члены: поля данных, методы (функции класса), типы, псевдонимы типов, шаблоны.

\vspace{1em}

Как и для любого типа, в программе можно создавать объекты класса. Обычно объекты класса называются экземплярами класса.

\vspace{1em}

 Классы можно наследовать от других классов -- наследник будет неявно включать в себя все базовые классы.

\end{frame}

\begin{frame}[fragile]{Определение}

Определение класса (структуры) является одной из форм выражений объявления и имеет вид:

\begin{itemize}
  \item \emph{класс имя} \lstinline|{| \emph{члены} \lstinline|}|
  \item \emph{класс имя} \lstinline{:} \emph{список наследования} \lstinline|{| \emph{члены} \lstinline|}|
\end{itemize}

\vspace{0.5em}

где \emph{класс} -- это ключевое слово \lstinline{class} или \lstinline{struct}, \emph{имя} -- идентификатор, \emph{члены} -- объявление членов класса, а \emph{список наследования} -- это перечисление через запятую базовых классов.

\vspace{0.5em}

Определение класса может повторяться в разных единицах трансляции, но все повторения должны быть идентичны.

\vspace{0.5em}

Предварительное объявление: \emph{класс имя} \lstinline{;}

Позволяет ссылаться на ещё не определенный класс.

\end{frame}

\begin{frame}[fragile]{Примеры определений и создания}

\begin{lstlisting}
class C {};

C c;
C cc = C();
C(); // temporary

struct S
{
  int a;
} s_a, s_b;
\end{lstlisting}

\end{frame}

\begin{frame}{Область видимости класса}

Область видимости для имен членов класса начинается в точке их объявления, распространяется на всё оставшееся тело класса, а также на:

\vspace{0.5em}

\begin{itemize}
  \item тела всех методов класса \vspace{0.5em}
  \item на инициализацию аргументов \vspace{0.5em}
  \item спецификаторы исключений \vspace{0.5em}
  \item внутриклассовую инициализацию \vspace{0.5em}
  \item все эти вещи во вложенных классах
\end{itemize}

\vspace{0.5em}

вне зависимости от места определения этих элементов.

\end{frame}

\begin{frame}[fragile]{Примеры области видимости}

\begin{lstlisting}
struct S {
  struct T {
    int z = 10 * n;
    int baz();
  };
  int foo(int x) {
    return x * n;
  }
  void bar(int y = n);
  int z = 3 * n;
  static const int n = 101;
  char str[n];
};
int S::T::baz()
{ return z * 3; }
void S::bar(int y)
{ z *= y; }
\end{lstlisting}

\end{frame}

\begin{frame}{Доступ к членам вне области видимости класса}

Вне области видимости класса, доступ к его членам может осуществляться с помощью операторов:

\vspace{1em}

\begin{itemize}
  \item \lstinline{.} -- применимо к выражению, имеющему тип класса или тип его потомка \vspace{0.5em}
  \item \lstinline{->} -- применимо к выражению, имеющему тип указателя на класс или его потомка \vspace{0.5em}
  \item \lstinline{::} -- применимо к имени класса или имени его потомка \vspace{0.5em}
\end{itemize}

\end{frame}

\begin{frame}[fragile]{Примеры доступа к членам вне класса}

\begin{lstlisting}
struct S {
    int x;
    void foo(char c);
    static int size();
};

int f(const S * ss) {
    S s;
    s.x = ss->x;
    s.foo('\n');
    return S::size();
}
\end{lstlisting}

\end{frame}

\begin{frame}{Наследование}

Классы можно наследовать от других классов.

\vspace{0.5em}

В определении класса указывается список базовых классов (с необязательными спецификаторами прав доступа) от которых класс наследуется.

\vspace{0.5em}

Каждый базовый класс включается в объект наследника как подобъект, в порядке следования базовых классов в списке наследования.

\vspace{0.5em}

Наследник имеет доступ ко всем членам базового класса (с учетом прав доступа).

Наследник может объявить собственный член с именем, идентичным имени какого-нибудь члена базового класса. Тогда член базового класса будет ``перекрыт''.

\end{frame}

\begin{frame}[fragile]{Поля данных}

Нестатические -- подобъекты объекта класса, статические -- глобальные объекты, связанные с типом класса.

\vspace{1em}

Объявление аналогично объявлению переменных:

\begin{lstlisting}
struct S {
  int a, b, c = 10;
  const char str[6] = "Hello";
  static const size_t size;
};

const size_t S::size = sizeof(S);

\end{lstlisting}

\end{frame}

\begin{frame}{Нестатические поля}

Нестатические поля нельзя объявлять \lstinline{auto}, \lstinline{extern} или \lstinline{thread_local}.

\vspace{0.5em}

Тип нестатических полей должен быть полностью определен в точке определения класса.

\vspace{0.5em}

Типом поля класса не может быть сам этот класс.

\vspace{0.5em}

Инициализация нестатического поля:
\begin{itemize}
  \item может быть описана в его объявлении
  \item может осуществляться в конструкторе класса
\end{itemize}

\vspace{0.5em}

Нестатические поля класса являются подобъектами, но как и объекты должны иметь ненулевой размер (и их размещение в памяти не может пересекаться).

\vspace{0.5em}

Нестатические поля класса являются частью каждого экземпляра класса и неотделимы от него.

\end{frame}

\begin{frame}[fragile]{Пример нестатических полей}

\begin{lstlisting}
class C {};
struct S
{
  C c;
  int a;
};

static_assert(sizeof(C) == 1);
static_assert(sizeof(SS) >= sizeof(int) + 1);

S x, y, z;
x.a = 1;
y.a = 222;
z.a = -999;
\end{lstlisting}

\end{frame}

\begin{frame}{Статические поля}

Статические поля полностью аналогичны глобальным переменным с внешним связыванием (внутренним, если класс определен в анонимном пространстве имён), за исключением имени.

\vspace{1em}

Статические поля могут быть \lstinline{thread_local}, а также могут быть \lstinline{auto}, если это \lstinline{const} поле и его инициализация описана в объявлении поля.

\vspace{1em}

Описание статических полей в теле класса является объявлением (за исключением \lstinline{constexpr} или \lstinline{inline}), но не определением (и может иметь неполный тип).

В программе должно быть также определение всех статических полей (для \lstinline{const} полей допускается отсутствие определения, если соответствующее поле не ODR-используется в программе).

\end{frame}

\begin{frame}[fragile]{Примеры статических полей}

\begin{lstlisting}
struct S; // forward declaration
class C {
  static int x; // declaration
  static S y; // declaration
};

static_assert(sizeof(C) == 1);

// definitions
int C::x = 101;
struct S {};
S C::y;
\end{lstlisting}

\end{frame}

\begin{frame}{Инициализация статических полей}

\begin{itemize}
  \item для не-\lstinline{const}, не-\lstinline{inline}, не-\lstinline{constexpr} статического поля -- в его определении \vspace{1em}
  \item для \lstinline{const} статического поля -- можно в определении класса, можно в определении поля \vspace{1em}
  \item для \lstinline{constexpr}/\lstinline{inline} статического поля -- только в определении класса (отдельное определение поля в этом случае не требуется)
\end{itemize}

\end{frame}

\begin{frame}[fragile]{Методы}

Это функции, объявленные как члены класса. Как и данные могут быть нестатическими и статическими.

\begin{lstlisting}
struct S {
  int f(); // declaration
  void g() { // definition
    // body
  }
};
int S::f() { // definition
  // body
}
\end{lstlisting}

Метод, определенный в теле класса неявно подразумевает спецификатор \lstinline{inline}.
Определение метода вне тела класса подчиняется правилам определения функций (например, в плане ODR).

\end{frame}

\begin{frame}[fragile]{Статические методы}

Статические методы полностью аналогичны функциям, за исключением прав доступа к другим членам соответствующего класса и обращения к ним самим:

\vspace{1em}

\begin{lstlisting}
struct S {
  static int size() { return 1111; }
};

void foo() {
  int x = S::size();
}
\end{lstlisting}

\vspace{1em}

У статических методов нет доступа к нестатическим полям класса.

\end{frame}

\begin{frame}[fragile]{Нестатические методы}

Нестатические методы аналогичны функциям, но при вызове связаны с конкретным экземпляром своего класса и могут оперировать его нестатическими полями. При этом экземпляр класса, для которого вызван нестатический метод, явно доступен через указатель \lstinline{this}.

\begin{lstlisting}
struct S {
  int a, b, c;
  void sum();
};
void S::sum() {
  a = b + c; // same as this->a = this->b + this->c
}
void foo() {
  S s;
  s.b = 2;
  s.c = 3;
  s.sum(); // s.a = s.b + s.c
}
\end{lstlisting}

\end{frame}

\begin{frame}[fragile]{cv-квалификаторы нестатических методов}

Нестатические методы могут иметь cv-квалификатор:

\vspace{1em}

\begin{lstlisting}
struct S {
  int foo(); // no qualifier
  size_t size() const; // const
  void boom() volatile; // volatile
};
S a;
a.foo();
a.size();
a.boom();
const S b;
b.foo(); // compilation error
b.size();
b.boom(); // compilation error
\end{lstlisting}

\end{frame}

\begin{frame}[fragile]{Операторы}

Это функции, объявленные как члены класса и повторяющие синтаксис вызова обычных операторов языка.

\vspace{1em}

\begin{lstlisting}
class C {
  C & operator++ () {
    // some custom increment logic
  }
};
C c;
++c;
\end{lstlisting}

\end{frame}

\begin{frame}[fragile]{Отображение синтаксиса операторов}

Определение операторов, как членов класса -- это частный случай перегрузки операторов.

\begin{center}
\begin{tabular}{ l l }
  \hline
    Синтаксис оператора & Синтаксис перегруженной функции \\
  \hline
    &\\
    @a & (a).operator@ () \\
    a@ & (a).operator@ (0) \\
    a@b & (a).operator@ (b) \\
    a(b…) & (a).operator() (b…) \\
    a[b] & (a).operator[] (b) \\
    a-> & (a).operator-> () \\
\end{tabular}
\end{center}

\end{frame}

\begin{frame}[fragile]{Члены-типы}

В определении класса можно объявить какой-либо тип или псевдоним типа, это будет считаться членом класса (с т.з. доступа к имени этого типа/псевдонима)

\vspace{2em}

\begin{lstlisting}
struct S {
  using X = int;
  class C { ... }; // nested class
};

S::X x = 10;
S::C c;
\end{lstlisting}

\end{frame}

\begin{frame}[fragile]{Вложенные классы}

В определении тела класса можно определить другой класс -- ``вложенный'' (nested). Кроме иных правил доступа к имени такого вложенного класса, от отдельно стоящего он отличается особенностями прав доступа к членам класса, в котором он определён.

\vspace{1em}

Внешние объявления членов вложенного класса:
\begin{lstlisting}
struct S {
  class C {
    int foo();
  };
};

int S::C::foo() {
  ...
}
\end{lstlisting}

\end{frame}

\begin{frame}{Права доступа к членам класса}

Имя каждого члена класса ассоциировано со спецификатором доступа, который проверяется во всех случаях использования этого имени. Можно сказать, что это ``права доступа''.

\vspace{2em}

Права доступа не распространяются на члены класса (включая на его вложенные классы) -- все члены класса (включая вложенные классы) имеют полный доступ ко всем другим членам класса.

\end{frame}

\begin{frame}{Виды доступа}

\begin{itemize}
  \item \lstinline{public} -- публичный доступ, нет ограничений \vspace{0.5em}
  \item \lstinline{private} -- закрытый доступ, только другие члены этого класса (или его друзья) имеют доступ к этому имени \vspace{0.5em}
  \item \lstinline{protected} -- защищенный доступ, подобно закрытому, но права доступа есть также у наследников класса
\end{itemize}

\vspace{1em}

Права доступа не влияют на видимость членов класса, только на разрешения доступа к ним, проверка прав доступа -- последний этап анализа выражения доступа к члену класса.

Например, анализ перегрузок, неявных приведений и т.п. происходит до проверки права доступа.

\end{frame}

\begin{frame}[fragile]{Назначение прав доступа}

\begin{itemize}
  \item явно в теле класса -- действует на все последующие члены до следующего явного указания спецификатора доступа \vspace{1em}
  \item при наследовании -- действует на все члены родительского класса
\end{itemize}

\begin{lstlisting}
class C {
  public: // inside of class definition
};

class CC : protected C // in inheritance list
{
};
\end{lstlisting}

\end{frame}

\begin{frame}[fragile]{Пример назначения прав доступа}

\begin{lstlisting}
class C
{
  public:
    size_t size() const;
  protected:
    void foo();
  private:
    int x;
};

C c;
c.size(); // OK
c.foo(); // compilation error
c.x = 10; // compilation error
\end{lstlisting}

\end{frame}

\begin{frame}[fragile]{Пример назначения прав доступа при наследовании}

\begin{lstlisting}
class CC : public C
{
    void bar()
    {
      const auto s = size(); // OK
      foo(); // OK
      x = 10; // compilation error
    }
};

CC cc;
cc.size(); // OK
cc.foo(); // compilation error
cc.x = 10; // compilation error
\end{lstlisting}

\end{frame}

\begin{frame}{Права доступа и наследование}

В соответствии с указанным спецификатором доступа при наследовании, наследование называют -- публичным, защищенным или закрытым.

\vspace{2em}

При этом
\begin{itemize}
  \item \lstinline{public} -- спецификаторы доступа родительского класса наследуются без изменений \vspace{0.5em}
  \item \lstinline{protected} -- \lstinline{public} и \lstinline{protected} члены родительского класса считаются \lstinline{protected} у наследника, \lstinline{private} не меняется \vspace{0.5em}
  \item \lstinline{private} -- \lstinline{public} и \lstinline{protected} члены родительского класса наследуются, как \lstinline{private}
\end{itemize}

\end{frame}

\begin{frame}{Права доступа и вложенные классы}

С точки зрения доступа к нему, имя типа вложенного класса является членом основного класса.

\vspace{2em}

Все члены вложенного класса -- это члены вложенного класса, к ним основной класс не имеет никакого специального доступа.

\vspace{2em}

Однако, т.к. вложенный класс является членом основного, члены вложенного класса имеют полный доступ к членам основного.

\end{frame}

\begin{frame}[fragile]{Пример прав доступа с вложенными классами}

\begin{lstlisting}
class C {
  private:
    class C_A {
        int x = 10;
      public:
        int get_x() const { return x + counter++; }
    };
    static int counter;
  public:
    static C_A get_a() { return C_A(); }
};

int C::counter = 0;
int foo() {
    const auto a = C::get_a();
    return a.get_x();
}
\end{lstlisting}

\end{frame}

\begin{frame}[fragile]{Различие между структурами и классами}

\begin{itemize}
  \item \lstinline{struct} -- по умолчанию, \lstinline{public} права доступа к членам, \lstinline{public} наследование \vspace{0.5em}
  \item \lstinline{class} -- по умолчанию, \lstinline{private} права доступа к членам, \lstinline{private} наследование
\end{itemize}

\vspace{1em}

\begin{lstlisting}
struct S : A // public inheritance implied
{
    int x; // public access implied
};

class C : B // private inheritance implied
{
  int y; // private access implied
};
\end{lstlisting}

\end{frame}

\begin{frame}[fragile]{Агрегатная инициализация}

\begin{lstlisting}
T obj = { a, b, c, ... };
T obj{a, b, c, ... };
{a, b, c, ...}; // temporary, type must be implied by context
\end{lstlisting}

\vspace{1em}

Агрегатная инициализация возможно для агрегатов:
\begin{itemize}
  \item Массив
  \item Класс с только публичными нестатическими полями, без конструкторов, без виртуальных методов, все базовые классы должны удовлетворять тем же требованиям
\end{itemize}

\vspace{1em}

Эффект агрегатной инициализации -- каждый подобъект объекта-агрегата инициализируется копией значения из списка инициализации.

\end{frame}

\begin{frame}[fragile]{Пример использования агрегатной инициализации}

\begin{lstlisting}
struct A {
  int a_a;
  int a_b;
};

struct B : A {
  char b_a;
  char b_b;
};

struct C : A, B {
  std::string c_a;
};

C c { 20, 30, -20, -30, '\0', 'X', "Hello" };
\end{lstlisting}

\end{frame}

%-------------------------------------------------

\section{Enum}

\begin{frame}[fragile]{Enum}

Перечисление (enumeration) -- это тип с конечным множеством значений, которые преобразуются к значениям интегрального типа.

\vspace{1em}

\begin{lstlisting}
enum Side {
  Buy,
  Sell,
  ShortSell
};

void foo(const Side side) {
  switch (side) {
    case Buy: ...
    case Sell: ...
  }
}
\end{lstlisting}

\end{frame}

\begin{frame}[fragile]{Классические открытые (unscoped) enum}

\begin{itemize}
  \item \lstinline{enum} [\emph{имя}] \lstinline|{| \emph{enumerator} [\lstinline{=} \emph{constexpr}]\lstinline{,} ... \lstinline|}|
  \item \lstinline{enum} \emph{имя} \lstinline{:} \emph{тип} \lstinline|{| \emph{enumerator} [\lstinline{=} \emph{constexpr}]\lstinline{,} ... \lstinline|}|
  \item \lstinline{enum} \emph{имя} \lstinline{:} \emph{тип}\lstinline{;}
\end{itemize}

\vspace{0.5em}

Конструкция объявляет новый тип \emph{имя}, если базовый тип не указан, то компилятор сам выбирает тип, не шире \lstinline{int} (если только объявленные значения не выходят за рамки \lstinline{int}/\lstinline{unsigned}).

\vspace{0.5em}

Каждый \emph{enumerator} становится константой объявленного типа в той же области видимости, что и объявление \lstinline{enum} (например, тело класса -- константы становятся его членами). Каждая такая константа может быть неявно приведена к интегральному типу по обычным правилам.

\end{frame}

\begin{frame}[fragile]{Строгие (scoped) enum}

\begin{itemize}
  \item \lstinline{enum} \lstinline{class} | \lstinline{struct} \emph{имя} \lstinline|{| \emph{enumerator} [\lstinline{=} \emph{constexpr}]\lstinline{,} ... \lstinline|}|
  \item \lstinline{enum} \lstinline{class} | \lstinline{struct} \emph{имя} \lstinline{:} \emph{тип} \lstinline|{| \emph{enumerator} [\lstinline{=} \emph{constexpr}]\lstinline{,} ... \lstinline|}|
  \item \lstinline{enum} \lstinline{class} | \lstinline{struct} \emph{имя} \lstinline{;}
  \item \lstinline{enum} \lstinline{class} | \lstinline{struct} \emph{имя} \lstinline{:} \emph{тип}\lstinline{;}
\end{itemize}

\vspace{0.5em}

Конструкция объявляет новый тип \emph{имя}, если базовый тип не указан, то он устанавливается в \lstinline{int}.

\vspace{0.5em}

Каждый \emph{enumerator} становится константой объявленного типа в области видимости \lstinline{enum} и их можно указывать в виде \emph{name}\lstinline{::}\emph{enumerator}.
Неявных преобразований строгого \lstinline{enum} к другим типам нет. Можно использовать явное преобразование.

\end{frame}

\begin{frame}[fragile]{Приведение к enum}

Значения интегральных, floating-point типов и других \lstinline{enum} могут быть \textbf{явно} приведены к любому \lstinline{enum} типу, даже если результирующее значение не равно ни одной enumeration константе.

\vspace{1em}

Если исходное значение не представимо в результирующем типе -- UB

\vspace{1em}

\begin{lstlisting}
enum class X { A, B, C};
X x = static_cast<X>(1001);
\end{lstlisting}

\end{frame}

\begin{frame}[fragile]{Пример строгих enum}

\begin{lstlisting}
enum class TimeInForce {
  Day = 0,
  GTD = 1,
  IOC = 2
};

void foo(const TimeInForce time_in_force) {
  switch (time_in_force) {
    case TimeInForce::Day: ...
    case TimeInForce::GTD: ...
  }
}
\end{lstlisting}

\end{frame}

\begin{frame}[fragile]{Преимущества строгих enum над классическими}

Имена элементов строгого \lstinline{enum} не засоряют область видимости -- важно для \lstinline{enum}, объявляемых в пространстве имен.

\vspace{2em}

Невозможно случайно получить неявное преобразование к совершенно другому типу.

\vspace{1em}

\begin{lstlisting}
enum Side { Buy, Sell };
enum TimeInForce { Day, GTD, IOC };

void foo(const Side side, const TimeInForce tif) {
    if (side == Day) { // obviously, an error, but compiles happily
        ...
    }
}
\end{lstlisting}

\end{frame}

%-------------------------------------------------

\section{Полные и неполные типы}

\begin{frame}[fragile]{Полные и неполные типы}

Полный тип -- полностью определенный тип.

\vspace{0.7em}

Неполный тип -- предварительное объявление класса или enum.

\vspace{1.2em}

Неполный тип можно использовать в некоторых случаях, когда компилятору для генерации соответствующего кода не нужно знать определение типа:

\vspace{0.5em}

\begin{lstlisting}
class Foo;
void foo(Foo * f);

enum class Side : int;
enum class TimeInForce : int;
TimeInForce bar(const Side side)
{
    return static_cast<TimeInForce>(side);
}
\end{lstlisting}

\end{frame}

%-------------------------------------------------

\section{Некоторые классы стандартной библиотеки}

\begin{frame}[fragile]{std::string}

\begin{lstlisting}
#include <string>

std::string s1; // empty string
std::string s2("Hello!"); // from a string literal
std::string s3 = s2, s4(s2); // from another string
std::string s5(10, 'x'); // results in "xxxxxxxxxx"
\end{lstlisting}

\end{frame}

\begin{frame}[fragile]{Некоторые методы, чтение}

\begin{lstlisting}
void foo(const std::string & s)
{
    if (!s.empty()) {
        const char * cs = s.c_str();
        ...
    }
    if (s.size() >= 5) {
        char x = s[4];
        ...
    }
    size_t i = s.find("el");
    if (i != s.npos) {
        const auto ss = s.substr(i, 5); // up to 5 characters starting with "el"
    }
}
\end{lstlisting}

\end{frame}

\begin{frame}[fragile]{Некоторые методы, запись}

\begin{lstlisting}
void bar(std::string & s)
{
    s.assign("Hello", 3); // s contains "Hel"
    s += '\t';
    s.replace(2, 4, "fOO"); // s contains "HefOO"
    s.clear();
    assert(s.empty());
}
\end{lstlisting}

\end{frame}

\begin{frame}[fragile]{std::array}

\begin{lstlisting}
#include <array>

std::array<int, 5> x = {1, 2, 3, 4, 5};
std::array<int, 10> y = x; // compilation error
static_assert(!x.empty());
static_assert(x.size() == 5);
x[2] = 22;
\end{lstlisting}

\end{frame}

\begin{frame}[fragile]{std::vector}

\begin{lstlisting}
#include <vector>

std::vector<unsigned> x(10, 0xAE);
std::vector<int> y(5);
y = x; // compilation error
assert(!x.empty());
assert(x.size() == 10);
auto z = y;
x[5] = 128;
y[3] = x[4];
std::memcpy(&x[0], &y[0], 5); // memory regions copying
\end{lstlisting}

\end{frame}

\begin{frame}[fragile]{Некоторые варианты использования std::vector}

\begin{lstlisting}
std::vector<std::string> strings;
strings.reserve(100);
for (std::string line; std::getline(std::cin, line); ) {
    strings.push_back(line);
}

std::cout << "Read " << strings.size() << " input lines" << std::endl;

strings.clear();
assert(strings.size() == 0);
assert(strings.size() != strings.capacity());

strings.shrink_to_fit();
assert(strings.size() == strings.capacity()); // maybe
\end{lstlisting}

\end{frame}

\begin{frame}[fragile]{std::pair}

\begin{lstlisting}
#include <utility>

std::pair<int, std::string> x(-10, "Cat");
std::cout << x.first << " : " << x.second << std::endl;

const auto y = std::make_pair(99, "Dog");

const auto z = std::make_pair<unsigned, std::string>(99, "Explicit dog");

std::pair<A, B> foo() {
    A a;
    B b;
    return {a, b};
}
\end{lstlisting}

\end{frame}

\begin{frame}[fragile]{std::tuple}

\begin{lstlisting}
#include <tuple>

auto many() {
    return std::make_tuple(1, 2, 'a', std::string{"str"});
}

const auto x = many();
std::cout << std::get<0>(x) << ", " << std::get<1>(x) << ", " << std::get<2>(x)
    << ", " << std::get<3>(x) << std::endl;

const auto y = std::tuple_cat(x, x, x);
static_assert(std::tuple_size_v<decltype(y)> == 12);
\end{lstlisting}

\end{frame}

\begin{frame}[fragile]{Structured bindings}

\begin{lstlisting}
std::tuple<A, B, C> foo();

auto [a, b, c] = foo();
a = A();
const auto x = b;

void foo(std::vector<std::pair<int, std::string>> & v) {
    for (auto & [n, s] : v) {
        n = s.size();
        s += '\n';
    }
}
\end{lstlisting}

\end{frame}

%-------------------------------------------------

\section{Наглядный пример}

\begin{frame}[fragile]{Команда wc}

\begin{cmdline}
$ cat a.txt
Hello, world
$ wc a.txt
       1       2      13 a.txt
\end{cmdline}

\vspace{2em}

Может использоваться для подсчёта строк, слов, символов и байт в текстовых файлах.

\end{frame}

\begin{frame}{Постановка задачи}

Для текстового ввода из стандартного потока ввода посчитать:

\vspace{1em}

\begin{itemize}
  \item Количество символов перевода строки \vspace{0.5em}
  \item Количество байтов \vspace{0.5em}
  \item Количество символов UTF-8 \vspace{0.5em}
  \item Количество слов (разделенных пробельными символами) \vspace{0.5em}
  \item Максимальную длину строки в символах
\end{itemize}

\end{frame}

\begin{frame}[fragile]{Полный вывод wc}

\begin{cmdline}
$ cat e.txt
\end{cmdline}
Привет, мир
\begin{cmdline}
$ wc -cmwlL e.txt
 1  2 12 21 11 e.txt
\end{cmdline}

\vspace{1em}

Порядок вывода:
\begin{enumerate}
  \item число строк
  \item число слов
  \item число символов
  \item число байт
  \item максимальная длина строки.
\end{enumerate}

\end{frame}


\begin{frame}[fragile]{Функция main}
\lstinputlisting[firstline=3,lastline=3,basicstyle=\ttfamily\small]{examples/wc.cpp}
...
\lstinputlisting[firstline=92,lastline=96,basicstyle=\ttfamily\small]{examples/wc.cpp}
\end{frame}

\begin{frame}[fragile]{Структура данных для собираемых данных}
\lstinputlisting[firstline=6,lastline=20,basicstyle=\ttfamily\srcbigsize]{examples/wc.cpp}
...
\lstinputlisting[firstline=90,lastline=90,basicstyle=\ttfamily\srcbigsize]{examples/wc.cpp}
\end{frame}

\begin{frame}[fragile]{Основная функция обработки}
\lstinputlisting[firstline=1,lastline=1,basicstyle=\ttfamily\srcbigsize]{examples/wc.cpp}
\lstinputlisting[firstline=4,lastline=4,basicstyle=\ttfamily\srcbigsize]{examples/wc.cpp}
...
\lstinputlisting[firstline=70,lastline=88,basicstyle=\ttfamily\srcbigsize]{examples/wc.cpp}
\end{frame}

\begin{frame}[fragile]{Подсчёт слов}
\lstinputlisting[firstline=2,lastline=2,basicstyle=\ttfamily\srcbigsize]{examples/wc.cpp}
...
\lstinputlisting[firstline=48,lastline=68,basicstyle=\ttfamily\srcbigsize]{examples/wc.cpp}
\end{frame}

\begin{frame}{Вкратце о UTF-8}

Один Unicode символ кодируется от 1 до 4 байт:

\vspace{1em}

\begin{center}
\begin{tabular}{ c | c | c | c | c }
  \hline
    Число байт & Байт 1 & Байт 2 & Байт 3 & Байт 4 \\
  \hline
    1 & 0xxxxxxx & - & - & - \\
    2 & 110xxxxx & 10xxxxxx & - & - \\
    3 & 1110xxxx & 10xxxxxx & 10xxxxxx & - \\
    4 & 11110xxx & 10xxxxxx & 10xxxxxx & 10xxxxxx \\
\end{tabular}
\end{center}

\end{frame}

\begin{frame}[fragile]{Подсчёт числа символов}
\lstinputlisting[firstline=22,lastline=46,basicstyle=\ttfamily\srcbigsize]{examples/wc.cpp}
\end{frame}

\begin{frame}[fragile]{Тестирование}

\begin{cmdline}
$ xxd a.txt
$ wc -cmwlL a.txt
0 0 0 0 0 a.txt
$ ./wc < a.txt
0	0	0	0	0

$ xxd b.txt
0000000: 31                                       1
$ wc -cmwlL b.txt
0 1 1 1 1 b.txt
$ ./wc < b.txt
0	1	1	1	1
\end{cmdline}

\end{frame}

\begin{frame}[fragile]{Тестирование, продолжение}

\begin{cmdline}
$ xxd c.txt
0000000: 0a                                       .
$ wc -cmwlL c.txt
1 0 1 1 0 c.txt
$ ./wc < c.txt
1	0	1	1	0

$ xxd d.txt
0000000: 310a                                     1.
$ wc -cmwlL d.txt
1 1 2 2 1 d.txt
$ ./wc < d.txt
1	1	2	2	1
\end{cmdline}

\end{frame}

\begin{frame}[fragile]{Тестирование, окончание}

\begin{cmdline}
$ xxd e.txt
0000000: d09f d180 d0b8 d0b2 d0b5 d182 2c20 d0bc  ............, ..
0000010: d0b8 d180 0a                             .....
$ wc -cmwlL e.txt
 1  2 12 21 11 e.txt
$ ./wc < e.txt
1	2	12	21	11
\end{cmdline}

\end{frame}

\begin{frame}[fragile]{Сравнение с wc из GNU coreutils}

\begin{cmdline}
$ wc -l wc.cpp
97 wc.cpp

$ wc -l coreutils-8.31/src/wc.c
895 coreutils-8.31/src/wc.c
\end{cmdline}

\vspace{1em}

На текстовом файле \textapprox8 Гб:

\begin{itemize}
  \item подсчёт только байт и строк -- медленнее на 20\%
  \item подсчёт всех метрик -- быстрее в \textapprox4 раза
\end{itemize}

\end{frame}

\end{document}