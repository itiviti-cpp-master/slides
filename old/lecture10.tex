\documentclass[unknownkeysallowed,xcolor=table]{beamer}
 
\usepackage [T2A,T1] {fontenc}
\usepackage[utf8]{inputenc}
\usepackage[english,russian]{babel}
\usepackage{amsmath}
\usepackage{listings}
\usepackage{url}
\usepackage{textcomp}
\usepackage{multirow}

\setbeamertemplate{navigation symbols}{}

\newcommand{\textapprox}{\raisebox{0.5ex}{\texttildelow}}

\lstnewenvironment{cmdline}
  {\lstset{basicstyle=\ttfamily\scriptsize,keywordstyle=\color{blue},
           language={bash}}}
  {}

\lstnewenvironment{cmdlinelarge}
  {\lstset{basicstyle=\ttfamily\small,keywordstyle=\color{blue},
           language={bash}}}
  {}

\colorlet{mygreen}{green!60!blue}
\colorlet{mymauve}{red!60!blue}
\definecolor{pblue}{rgb}{0.1, 0.2, 0.8}

\newcommand{\rarr}{$\rightarrow$}

\lstset{
      basicstyle=\ttfamily\small,
      commentstyle=\color{mygreen},
      keywordstyle=\color{blue},
      numberstyle=\tiny\color{blue},
      stringstyle=\color{mymauve},
      numbers=left,
      stepnumber=1,
      columns=fullflexible,
      breaklines=true,
      postbreak=\mbox{\textcolor{red}{\ensuremath{\hookrightarrow}\space}},
      literate={~} {\textapprox}{1},
      language={[11]C++}
}

\makeatletter
\newcommand{\srcmediumsize}{\@setfontsize{\srcmediumsize}{7pt}{7pt}}
\makeatother

\makeatletter
\newcommand{\srcbigsize}{\@setfontsize{\srcbigsize}{8pt}{8pt}}
\makeatother

\makeatletter
\newcommand{\srcsize}{\@setfontsize{\srcsize}{6pt}{6pt}}
\makeatother

\makeatletter
\newcommand{\srcsmallsize}{\@setfontsize{\srcsmallsize}{5pt}{5pt}}
\makeatother

\newcommand{\rot}[1]{\makebox[1em][l]{\rotatebox{45}{#1}}}

\title[C++]
{Программирование на языке C++}
 
\subtitle{Вводный курс}
 
\author[А.~Б.~Морозов]
{
  \texorpdfstring{Александр Морозов\newline\href{mailto:gelu.speculum@gmail.com}{gelu.speculum@gmail.com}}
  {Александр Морозов}
}
  
\date[ITMO 2019]
{ИТМО, весенний семестр 2019}
 
\logo{%
  \makebox[0.97\paperwidth]{%
    \includegraphics[align=c,width=2cm,keepaspectratio]{itmo_logo.png}
    \hfill
    \includegraphics[align=c,width=1.5cm,keepaspectratio]{new_itiviti_logo.png}
  }
}

\AtBeginSection[]
{
  \begin{frame}
    \frametitle{Содержание}
    \tableofcontents[currentsection]
  \end{frame}
}

\begin{document}

\frame{\titlepage}


%-------------------------------------------------

\section{Специальные методы классов}

\begin{frame}{Конструктор}
Конструктор -- это специальная функция-член класса, служащая для инициализации объектов этого класса.

\vspace{1em}

Имя функции совпадает с именем класса.

\vspace{1em}

При создании объекта класса всегда вызывается один конструктор.

\vspace{1em}

Конструктор нельзя вызвать явно.

\vspace{1em}

У класса может быть произвольное число конструкторов.
\end{frame}

\begin{frame}[fragile]{Деструктор}
Деструктор -- это специальная функция-член класса, служащая для выполнения произвольных действий непосредственно перед удалением объекта класса.

\vspace{1em}

Обычно используется для освобождения ресурсов, используемых классом, или иной финализации.

\vspace{1em}

Имя функции -- \lstinline{~ClassName}.

\vspace{1em}

При любом удалении объекта класса, его деструктор вызывается.

\vspace{1em}

У класса может быть только один деструктор.
\end{frame}

\begin{frame}{Особенности конструкторов и деструкторов}
\begin{itemize}
  \item нет возвращаемого значения \vspace{1em}
  \item нет cv-квалификаторов \vspace{1em}
  \item нет ref-квалификаторов
\end{itemize}
\end{frame}

\begin{frame}[fragile]{Пример конструкторов и деструктора}
\begin{lstlisting}
struct S {
    S(); // 1
    S(int); // 2
    S(char, double); // 3
    S(const S &); // 4
    ~S();
};

int main() {
    S s1; // 1st constructor is called
    S s2(10); // 2nd constructor is called
    S s3('a', 0); // 3rd constructor is called
    S s4 = s1; // 4th constructor is called
} // S::~S() for s1, s2, s3 and s4 is called
\end{lstlisting}
\end{frame}

\begin{frame}[fragile]{Список инициализации членов}
\begin{lstlisting}
struct S {
    S(int x) : a(x), b(a*2) {}
    
    int a;
    int b = 10;
};

struct SS : S {
    SS() : S(20), c(b) {}
    
    int c;
};
\end{lstlisting}
\end{frame}

\begin{frame}[fragile]{Варианты инициализации в списке}
\begin{lstlisting}
struct A {
    A() {}
    int a = 10;
};
struct B {
    B(int x) : b(x) {} // direct initialization
    int b;
};
template <class... Ts>
struct C : A, B, Ts... {
    C(int x, int y, const Ts &... ts)
        : B{x}  // list initialization
        , Ts(ts)... // pack expansion
        , c(y) // direct
    {}
    int c;
};
\end{lstlisting}
\end{frame}

\begin{frame}{Порядок инициализации объекта класса}
\begin{enumerate}
  \item базовые классы инициализируются в том порядке, в каком они перечислены в списке наследования \vspace{1em}
  \item нестатические поля класса инициализируются в том порядке, в каком они объявлены в определении класса \vspace{1em}
  \item выполняется тело конструктора
\end{enumerate}
\end{frame}

\begin{frame}{Порядок уничтожения объекта класса}
\begin{enumerate}
  \item тело деструктора \vspace{1em}
  \item нестатические поля класса разрушаются в порядке, обратном порядку их объявления в классе \vspace{1em}
  \item для базовых классов деструкторы вызываются в порядке, обратном их порядку в списке наследования
\end{enumerate}
\end{frame}

\begin{frame}[fragile]{Некорректное завершение выполнения конструктора}
\begin{lstlisting}
struct S {
    std::string s;
    std::vector<int> v;
    S * next = nullptr;
    
    S();
    S(int);
};
S::S()
    : s("Hello, world")
{
    v.resize(100);
    next = new S(1);
    throw 1;
}
void f() { S s; }
\end{lstlisting}
\end{frame}

\begin{frame}[fragile]{Делегирование конструкторов}
\begin{lstlisting}
struct S {
    S(int, double, char) {...}
    S() : S(10, 0.5, 'a') {...}
};
\end{lstlisting}

\vspace{2em}

У делегирующего конструктора список инициализации членов должен состоять из одного элемента -- вызова целевого конструктора.
\end{frame}

\begin{frame}[fragile]{Наследование конструкторов}
\begin{lstlisting}
struct A {
    A(int, double, char);
};

struct B : A {};

struct C : A {
    using A::A;
};

int main() {
    B b(10, 0.5, 'a'); // error
    C c(10, 0.5, '\n'); // OK
}
\end{lstlisting}
\end{frame}

\begin{frame}[fragile]{explicit конструкторы}
\begin{lstlisting}
struct A {
    A(int);
};
struct B {
    explicit B(int);
};
A f(const A &) {
    return 10; // OK, A::A(int) is called
}
B g(const B &) {
    return 10; // error
    return B(10); // OK, B::B(int) is called
}
int main() {
    f(10); // OK, A::A(int) is called
    g(10); // error
    g(B(10)); // OK B::B(int) is called
}
\end{lstlisting}
\end{frame}

\begin{frame}[fragile]{Удалённые и автоматически создаваемые методы}
Некоторые методы могут быть созданы автоматически -- как неявно, так и явно.

\vspace{0.5em}

Запретить неявную автоматическую генерацию можно ``удалив'' метод.
\begin{lstlisting}
struct A {}; // A::A() is implicitly-defined

struct B {
    B() = default; // B::B() implicit definition is forced
};

struct C {
    C() = delete; // C::C() is not defined
};
\end{lstlisting}
Некоторые методы могут быть удалены неявно.
\end{frame}

\begin{frame}{Методы, создаваемые автоматически}
\begin{itemize}
  \item конструктор по умолчанию \vspace{1em}
  \item конструктор копирования \vspace{1em}
  \item конструктор перемещения \vspace{1em}
  \item деструктор \vspace{1em}
  \item операторы присваивания
\end{itemize}
\end{frame}

\begin{frame}[fragile]{Конструктор по умолчанию}
\begin{lstlisting}
struct S {
    S() {...} // default constructor
};

S s; // default constructor is called

struct A {
    A() {...}
};
struct B : A {
    B() {...}
};

B b; // A::A() and B::B() are called
\end{lstlisting}
\end{frame}

\begin{frame}[fragile]{Автоматическое создание конструктора по умолчанию}
Если у класса нет явно определенных конструкторов, то конструктор по умолчанию создаётся автоматически и эквивалентен явно определенному конструктору без списка инициализации и с пустым телом:
\begin{lstlisting}
struct A {
};

struct B {
    B() {}
};
\end{lstlisting}
\end{frame}

\begin{frame}[fragile]{Удалённый конструктор по умолчанию}
\begin{lstlisting}
struct A {
    A() = delete;
};
struct B {
    ~B() = delete;
};
class C {
    C() = default;
};
class D {
    ~D() {}
};
struct E : B, C, D {
    int & n;
    const int x;
    A a;
};
\end{lstlisting}
\end{frame}

\begin{frame}[fragile]{Преобразующие конструкторы}
\begin{lstlisting}
struct A {
    A(int);
    A(char);
    A(double, int = 10);
};

void f(A);

int main() {
    f(10);
    f('a');
    f(0.5);
}
\end{lstlisting}
\end{frame}

%-------------------------------------------------

\section{Копирование и перемещение}

\begin{frame}{Конструктор копирования}
Это не шаблонный конструктор, принимающий первым аргументом lvalue ссылку на объект того же типа, что и класс конструктора, который может быть вызван с одним аргументом.

\vspace{1em}

Конструктор копирования вызывается всегда, когда объект класса инициализируется из lvalue выражения того же типа (класса).
\end{frame}

\begin{frame}[fragile]{Пример конструкторов копирования}
\begin{lstlisting}
struct S {
    S(int);
    S(const S &); // copy constructor 1
    S(S &); // copy constructor 2
};

S f(S s) {
    return s; // copy constructor 2 is called
}

const S s1(10);
S s2 = s1, s3(s1); // copy constructor 1 is called
f(S(10)); // copy constructor 1 is called
\end{lstlisting}
\end{frame}

\begin{frame}[fragile]{Автоматическое создание конструктора копирования}
Если у класса не объявлено явно никаких конструкторов копирования, то автоматически создаётся конструктор копирования.
\begin{lstlisting}
struct A {
    char c;
    A(char c_) : c(c_) {}
};
struct S {
    int i;
    A a;
    S(int i_, char c_) : i(i_), a(c_) {}
};
S s1(10, '\n');
S s2 = s1;
assert(s1.i == s2.i && s2.i == 10);
assert(s1.a.c == s2.a.c && s2.a.c == '\n');
\end{lstlisting}
\end{frame}

\begin{frame}[fragile]{Удалённый конструктор копирования}
\begin{lstlisting}
struct A {
    A() = default;
    A(const A &) = delete;
};

struct B {
    B(B &&); // user-defined move constructor
    // B(const B &) = default
};

class C {
    ~C() {}
};

class D : B, C {
    int && n;
    A a;
};
\end{lstlisting}
\end{frame}

\begin{frame}[fragile]{Копирующий оператор присваивания}
Это не шаблонный и не статический оператор присваивания, принимающий ровно один аргумент, имеющий тип \lstinline{T}, \lstinline{T&} или \lstinline{const T&} (если \lstinline{T} -- это данный класс).

\vspace{1em}

\begin{lstlisting}
struct S {
    S(const S &);
    S & operator = (const S &);
};

S s1, s2 = s1; // copy constructor is called
s1 = s2; // copy assignment is called
\end{lstlisting}
\vspace{1em}

Предполагаемое поведение такого оператора -- копирование значения одного объекта данного класса в другой объект этого же класса.
\end{frame}

\begin{frame}[fragile]{Автоматическое создание копирующего оператора присваивания}
\begin{lstlisting}
struct A {
    char c;
    A(char c_) : c(c_) {}
};
struct S {
    int i;
    A a;
    S(int i_, char c_) : i(i_), a(c_) {}
};
S s1(10, '\n');
S s2(20, 'x');
s2 = s1;
assert(s1.i == s2.i && s2.i == 10);
assert(s1.a.c == s2.a.c && s2.a.c == '\n');
\end{lstlisting}
\end{frame}

\begin{frame}[fragile]{Удалённый копирующий оператор присваивания}
\begin{lstlisting}
struct A {
    A(A &&); // move constructor
    // A & operator = (const A &) = default
};
struct B {
    void operator = (const B &) = delete;
};

struct C : B {
    A a;
    const int i;
    int & x;
};
\end{lstlisting}
\end{frame}

\begin{frame}{Конструктор перемещения}
Это не шаблонный конструктор, принимающий первым аргументом rvalue ссылку на объект того же типа, что и класс конструктора, может быть вызван с одним аргументом.

\vspace{1em}

Конструктор перемещения вызывается всегда, когда объект класса инициализируется из xvalue выражения того же типа (класса).

\vspace{1em}

Предназначение перемещающего конструктора -- забрать содержимое у аргумента - временного объекта и, по возможности, избежать затрат на копирование этого содержимого. 

При этом аргумент должен остаться в валидном, но не обязательно определённом, состоянии.
\end{frame}

\begin{frame}[fragile]{Пример конструкторов копирования и перемещения}
\begin{lstlisting}
struct S {
    S(const S &); // copy constructor
    S(S &&); // move constructor
};

void f(S s) {...}

int main() {
    S s1;
    f(s1); // copy constructor is called
    f(S{}); // move constructor is called
    f(std::move(s1)); // move constructor is called
}
\end{lstlisting}
\end{frame}

\begin{frame}[fragile]{Автоматическое создание конструктора перемещения}
\begin{lstlisting}
struct A {
    A(const A &) { std::cout << "copy" << std::endl; }
    A(A &&) { std::cout << "move" << std::endl; }
};
struct B {
    A a;
    // B(const B &);
    // B & operator = (const B &);
    // B & operator = (B &&);
    // ~B();
};

void f(B b) {}
int main() {
    f(B{});
}
\end{lstlisting}
\end{frame}

\begin{frame}[fragile]{Удалённый конструктор перемещения}
\begin{lstlisting}
class A {
    A(A &&);
};

struct B {
    B(const B &);
};

struct C {
    ~C() = delete;
};

class D : B, C {
    A a;
};
\end{lstlisting}
\end{frame}

\begin{frame}[fragile]{Перемещающий оператор присваивания}
Это не шаблонный и не статический оператор присваивания, принимающий ровно один аргумент, имеющий тип \lstinline{T&&} или \lstinline{const T&&}.

\vspace{1em}

\begin{lstlisting}
struct S {
    S & operator = (S &&);
};

S s1, s2;
s2 = S(); // move assignment is called
s1 = std::move(s2); // move assignment is called

\end{lstlisting}
\vspace{1em}

Предполагаемое поведение перемещающего оператор присваивания -- перемещение значения одного объекта данного класса в другой объект данного класса.
\end{frame}

\begin{frame}[fragile]{Автоматическое создание перемещающего оператора присваивания}
\begin{lstlisting}
struct A {
    void operator=(const A&) { std::cout<<"copy"<<std::endl; }
    void operator=(A&&) { std::cout<<"move"<<std::endl; }
};
struct B {
    A a;
    // B(const B &);
    // B(B &&);
    // B & operator = (const B &);
    // ~B();
};
int main() {
    B b1, b2;
    b1 = std::move(b2);
}
\end{lstlisting}
\end{frame}

\begin{frame}[fragile]{Удалённый перемещающий оператор присваивания}
\begin{lstlisting}
struct A {
    void operator = (A &&) = delete;
};

struct B {
    ~B() = default;
};

struct C : B {
    A a;
    const int n;
    char && r;
};
\end{lstlisting}
\end{frame}

\begin{frame}[fragile]{std::move}
\begin{lstlisting}
struct S;

void f(S &); // 1
void f(S &&); // 2
void f(const S &); // 3
void f(const S &&); // 4

S s1;
const S s2;
f(s1); // 1st is called
f(s2); // 3rd is called
f(std::move(s1)); // 2nd is called
f(std::move(s2)); // 4th is called
\end{lstlisting}
\end{frame}

%-------------------------------------------------

\section{Квалификаторы методов}

\begin{frame}[fragile]{Неявный аргумент при разрешении перегрузки}
\begin{lstlisting}
struct S {
    void f(int);
};
S s;
s.f(10); // correct syntax
f(s, 10); // implied syntax in overload resolution
\end{lstlisting}
\vspace{1em}
Другой взгляд:
\begin{lstlisting}
struct S {
    void f(int);
    void g(char) const;
};
void f(S * this, int) { ... }
void g(const S * this, char) { ... }
\end{lstlisting}
\end{frame}

\begin{frame}[fragile]{Перегрузка методов по cv-квалификаторам}
\begin{lstlisting}
struct S {
    void f() const; // 1
    void f() volatile; // 2
    void f(); // 3
};
void t1(S & s) {
    s.f(); // 3rd 'f' is called
}
void t2(const S & s) {
    s.f(); // 1st 'f' is called
}
void t3(volatile S & s) {
    s.f(); // 2nd 'f' is called
}
\end{lstlisting}
\end{frame}

\begin{frame}[fragile]{ref квалификаторы}
\begin{lstlisting}
struct S {
    void f() &; // 1
    void f() const &; // 2
    void f() &&; // 3
};

S s1;
const S s2;
s1.f(); // 1st 'f' is called
s2.f(); // 2nd 'f' is called
std::move(s1); // 3rd 'f' is called
\end{lstlisting}
Воображаемый альтернативный синтаксис:
\begin{lstlisting}
void f(S & self); // 1
void f(const S & self); // 2
void f(S && self); // 3
\end{lstlisting}
\end{frame}

%-------------------------------------------------

\section{Copy elision}

\begin{frame}[fragile]{Copy elision}
Отсутствие материализации: \vspace{1em}
\begin{itemize}
  \item инструкция return, когда её операнд -- prvalue того же типа, что и тип возвращаемого значения функции \vspace{1em}
  \item инициализация переменной, когда выражение инициализации -- prvalue того же типа
\end{itemize}

\vspace{1em}
\begin{lstlisting}
T f() {
    return T();
}
T x = T(T(T(f()))); // only default constructor of T is called
\end{lstlisting}
\end{frame}

\begin{frame}[fragile]{NRVO}
Named return value optimization: если в инструкции return операнд обозначает локальную переменную, которая:
\begin{itemize}
  \item не является параметром функции
  \item объект которой не является volatile
  \item объект которой имеет автоматический тип размещения
  \item тип объекта которой совпадает с типом возвращаемого значения функции
\end{itemize}
То компилятор \emph{может} не генерировать код вызова конструктора копирования/перемещения, как если бы это была copy elision.

\begin{lstlisting}
T f() {
    T t;
    return t;
}
T x = f(); // only default constructor of T is called
\end{lstlisting}
\end{frame}

\end{document}
