\documentclass[unknownkeysallowed,xcolor=table]{beamer}
 
\usepackage [T2A,T1] {fontenc}
\usepackage[utf8]{inputenc}
\usepackage[english,russian]{babel}
\usepackage{amsmath}
\usepackage{listings}
\usepackage{url}
\usepackage{textcomp}
\usepackage{multirow}

\setbeamertemplate{navigation symbols}{}

\newcommand{\textapprox}{\raisebox{0.5ex}{\texttildelow}}

\lstnewenvironment{cmdline}
  {\lstset{basicstyle=\ttfamily\scriptsize,keywordstyle=\color{blue},
           language={bash}}}
  {}
  
\colorlet{mygreen}{green!60!blue}
\colorlet{mymauve}{red!60!blue}
\definecolor{pblue}{rgb}{0.1, 0.2, 0.8}

\newcommand{\rarr}{$\rightarrow$}

\lstset{
      basicstyle=\ttfamily\small,
      commentstyle=\color{mygreen},
      keywordstyle=\color{blue},
      numberstyle=\tiny\color{blue},
      stringstyle=\color{mymauve},
      columns=fullflexible,
      breaklines=true,
      postbreak=\mbox{\textcolor{red}{\ensuremath{\hookrightarrow}\space}},
      literate={~} {\textapprox}{1},
      language={[11]C++}
}

\makeatletter
\newcommand{\srcmediumsize}{\@setfontsize{\srcmediumsize}{7pt}{7pt}}
\makeatother

\makeatletter
\newcommand{\srcbigsize}{\@setfontsize{\srcbigsize}{8pt}{8pt}}
\makeatother

\makeatletter
\newcommand{\srcsize}{\@setfontsize{\srcsize}{6pt}{6pt}}
\makeatother

\makeatletter
\newcommand{\srcsmallsize}{\@setfontsize{\srcsmallsize}{5pt}{5pt}}
\makeatother

\title[C++]
{Программирование на языке C++}
 
\subtitle{Вводный курс}
 
\author[А.~Б.~Морозов]
{
  \texorpdfstring{Александр Морозов\newline\href{mailto:gelu.speculum@gmail.com}{gelu.speculum@gmail.com}}
  {Александр Морозов}
}
  
\date[ITMO 2019]
{ИТМО, весенний семестр 2019}
 
\logo{%
  \makebox[0.97\paperwidth]{%
    \includegraphics[align=c,width=2cm,keepaspectratio]{itmo_logo.png}
    \hfill
    \includegraphics[align=c,width=1.5cm,keepaspectratio]{new_itiviti_logo.png}
  }
}

\AtBeginSection[]
{
  \begin{frame}
    \frametitle{Содержание}
    \tableofcontents[currentsection]
  \end{frame}
}

\begin{document}

\frame{\titlepage}

%-------------------------------------------------

\section{Представление сложных типов в памяти}

\begin{frame}{Классы}

Объект классового типа состоит из:
\begin{itemize}
  \item всех нестатических полей этого класса -- подобъекты-члены
  \item всех базовых классов -- подобъекты - базовые классы
\end{itemize}

\vspace{1em}

Общие правила:
\begin{itemize}
  \item размеры подобъектов-членов $\geq 1$ (ссылочные поля -- не обязательно)
  \item объектные представления подобъектов класса не могут пересекаться
  \item последовательность размещения подобъектов в целом не оговорена
  \item подобъекты-члены с одинаковыми правами доступа располагаются друг относительно друга в том же порядке, что и в объявлении
\end{itemize}

\end{frame}

\begin{frame}[fragile]{Empty base optimization}

Подобъект - базовый класс может иметь нулевой размер, если этот класс не содержит нестатических полей и не совпадает по типу с первым нестатическим полем данного класса.

\vspace{1em}

\begin{lstlisting}
struct X {};
struct Y {};

struct Z1 : X, Y {};

struct Z2 : X, Y { X x; };

struct Z3 : X, Y { X x; Y y; };

struct Z4 : X, Y { int n; };
\end{lstlisting}

\end{frame}

\begin{frame}[fragile]{Union}

Union -- специальный случай пользовательского типа, подобный классу, но экземпляр которого в один конкретный момент времени содержит только одно из его нестатических полей.

\vspace{1em}

\lstinline{union} \emph{[имя]} \lstinline|{| \emph{члены} \lstinline|}|

\vspace{1em}

\begin{itemize}
  \item нет виртуальных методов
  \item нет наследования (ни union, ни от union)
  \item нет ссылочных нестатических полей
  \item только для одного из нестатических полей можно объявить инициализацию
  \item наличие нетривиального специального метода у поля делает соответствующий метод \lstinline{union} удалённым по умолчанию
\end{itemize}

\end{frame}

\begin{frame}{Свойства union}

\begin{itemize}
  \item права доступа по умолчанию -- \lstinline{public} \vspace{0.5em}
  \item размер объекта \lstinline{union} равен размеру наибольшего его нестатического поля \vspace{0.5em}
  \item все остальные поля размещаются в той же области памяти, что и наибольшее \vspace{0.5em}
  \item определен доступ только к активному полю \lstinline{union}, доступ к неактивному -- UB \vspace{0.5em}
  \item время жизни поля начинается в момент активации и заканчивается в момент деактивации
\end{itemize}

\end{frame}

\begin{frame}{Активация полей union}

Поле может быть активировано:

\begin{itemize}
  \item инициализацией
  \item присвоением ему значения
  \item вызовом размещающего new
  \item участием в левой части выражения присваивания, если только эта операция не переопределена или поле имеет нетривиальный или удаленный конструктор по умолчанию
\end{itemize}

\vspace{0.5em}

Предыдущее активное поле при этом деактивируется. Если поле имеет нетривиальный деструктор, то его необходимо деактивировать явно, вызвав этот деструктор.

\vspace{0.5em}

Написать нетривиальную операцию копирования для \lstinline{union} можно только зная, какое именно поле активно в данный момент -- с обычными union проблематично.

\end{frame}

\begin{frame}[fragile]{Примеры активации}

\begin{lstlisting}
union A { double d; int x[4]; };
struct B { A a; };
union C { B b; int k; };
C c; // no active member
c.k = 10; // k is active
c.b.a.x[3] = -1; // b is active in c and x is active in c.b.a
// int i = c.k; <- UB
int i = c.b.a.x[3]; // OK

struct X { const int a; int b; };
union Y { X x; int k; };
Y y = { {1, 2} };
int n = y.x.a;
y.k = 4;
y.x.b = n; // UB - X default constructor is deleted
\end{lstlisting}

\end{frame}

\begin{frame}[fragile]{Анонимные union}

\lstinline{union}, определенные без имени и без одновременного определения переменной этого типа (или указателя, ссылки на этот тип), являются анонимными.

\vspace{1em}

\lstinline|union {| \emph{члены} \lstinline|};|

\vspace{1em}

Имеют дополнительные ограничения:
\begin{itemize}
  \item отсутствие методов \vspace{0.5em}
  \item отсутствие статических членов \vspace{0.5em}
  \item права доступа -- только public
\end{itemize}

\vspace{1em}

Имена полей анонимного \lstinline{union} попадают в окружающую область видимости и не должны конфликтовать с другими именами в этой области видимости.

\end{frame}

\begin{frame}[fragile]{Пример анонимного union}

\begin{lstlisting}
int f() {
    union {
        int x;
        double y;
    };
    x = 1;
    y = 0.5;
    return x; // UB
}
\end{lstlisting}

\end{frame}

\begin{frame}[fragile]{Union-подобные классы}

\lstinline{union}-подобный класс -- класс, в составе которого есть хотя бы один анонимный union.

\vspace{0.7em}

Пример использования -- tagged union:

\begin{lstlisting}
struct S {
    enum { CHAR, INT, DOUBLE } tag;
    union {
        char c;
        int i;
        double d;
    };
};
S s{S::CHAR, 'a'};
std::cout << s.c << std::endl;

S ss{S::INT, 111111};
\end{lstlisting}

\end{frame}

\begin{frame}[fragile]{Улучшенный tagged union}

\begin{lstlisting}[basicstyle=\ttfamily\srcsize]
class Var {
public:
    enum TAG {CHAR, INT, DOUBLE};
    
    Var() = delete;
    Var(const char c)
        : m_tag(CHAR), m_c(c)
    { }
    Var(const int i)
        : m_tag(INT), m_i(i)
    { }
    Var(const double d)
        : m_tag(DOUBLE), m_d(d)
    { }
    
    TAG get_tag() const { return m_tag; }
    char as_char() const { return m_c; }
    char & as_char() { return m_c; }
    int as_int() const { return m_i; }
    int & as_int() { return m_i; }
    double as_double() const 	{ return m_d; }
    double & as_double() { return m_d; }
private:
    TAG m_tag;
    union {
        char m_c;
        int m_i;
        double m_d;
    };
};
Var var(111111);
std::cout << var.get_tag() << " : " << var.as_int() << std::endl;
\end{lstlisting}

\end{frame}

\end{document}