\documentclass[unknownkeysallowed]{beamer}
 
\usepackage [T2A,T1] {fontenc}
\usepackage[utf8]{inputenc}
\usepackage[english,russian]{babel}
\usepackage{amsmath}
\usepackage{listings}
\usepackage{xcolor}
\usepackage{url}

\setbeamertemplate{navigation symbols}{}

\lstnewenvironment{cmdline}
  {\lstset{basicstyle=\ttfamily,keywordstyle=\color{blue},
           language={bash}}}
  {}
  
\colorlet{mygreen}{green!60!blue}
\colorlet{mymauve}{red!60!blue}
\definecolor{pblue}{rgb}{0.1, 0.2, 0.8}

\lstnewenvironment{cpp}
  {\lstset{
      basicstyle=\ttfamily,
      commentstyle=\color{mygreen},
      keywordstyle=\color{blue},
      numberstyle=\tiny\color{blue},
      stringstyle=\color{mymauve},
      language={[11]C++}}
    }
  {}
  
\makeatletter
\newcommand{\srcsize}{\@setfontsize{\srcsize}{6pt}{6pt}}
\makeatother

\makeatletter
\newcommand{\srcsmallsize}{\@setfontsize{\srcsmallsize}{5pt}{5pt}}
\makeatother

  
\lstset{
      basicstyle=\ttfamily\srcsize,
      commentstyle=\color{mygreen},
      keywordstyle=\color{blue},
      numberstyle=\tiny\color{blue},
      stringstyle=\color{mymauve},
      columns=fullflexible,
      breaklines=true,
      postbreak=\mbox{\textcolor{red}{\ensuremath{\hookrightarrow}\space}},
      language={[11]C++}
}

\title[C++]
{Программирование на языке C++}
 
\subtitle{Вводный курс}
 
\author[А.~Б.~Морозов]
{
  \texorpdfstring{Александр Морозов\newline\href{mailto:gelu.speculum@gmail.com}{gelu.speculum@gmail.com}}
  {Александр Морозов}
}
  
\date[ITMO 2019]
{ИТМО, весенний семестр 2019}
 
\logo{%
  \makebox[0.95\paperwidth]{%
    \includegraphics[align=c,width=2cm,keepaspectratio]{itmo_logo.png}
    \hfill
    \includegraphics[align=c,width=1.1cm,keepaspectratio]{itiviti_logo.png}
  }
}

\AtBeginSection[]
{
  \begin{frame}
    \frametitle{Содержание}
    \tableofcontents[currentsection]
  \end{frame}
}

\begin{document}

\frame{\titlepage}

%-------------------------------------------------

\section{Организационная информация}

\begin{frame}{Github репозиторий}

Зарегистрироваться на \href{https://github.com/}{https://github.com/} \vspace{3em}

Послать свой Github профиль Олегу Доронину \href{mailto:dorooleg@niuitmo.ru}{dorooleg@niuitmo.ru} \vspace{3em}

Вики курса: \href{https://github.com/itiviti-cpp/wiki/wiki}{https://github.com/itiviti-cpp/wiki/wiki}

\end{frame}

%-------------------------------------------------

\section{Текст, выражения, инструкции}

\begin{frame}{Первичная обработка текста}

  \begin{enumerate}
    \item Преобразование бинарного файла в последовательность символов
    \item Склейка ``многострочных'' строк
    \item Разбиение на комментарии, пробельные символы и токены предварительной обработки:
      \begin{itemize}
        \item Имена заголовочных файлов
        \item Идентификаторы
        \item Предварительные токены чисел
        \item Символьные и строковые литералы
        \item Операторы и символы пунктуации
        \item Прочие символы
      \end{itemize}
    \item Удаление комментариев
    \item Препроцессинг
    \item Склейка соседних строковых литералов
  \end{enumerate}
  
\end{frame}

\begin{frame}[fragile]{Примеры первичной обработки}

  \begin{cpp}
    #include <iostream> // include basic IO
    #define MOO(x) \
      struct x \
      { \
      }
    MOO(a);
    MOO(b);
    /*
     *   Multi line comment
     *
     */
    call("String" "to" "be" "concatenated");
  \end{cpp}

\end{frame}

\begin{frame}{Идентификаторы}

  \begin{itemize}
    \item Идентификатор - последовательность букв, цифр и символов подчеркивания произвольной длины
    \item Не может начинаться с цифры
    \item Чувствителен к регистру
    \item Может использоваться для объявления сущностей, но идентификаторы:
      \begin{itemize}
        \item совпадающие с ключевыми словами -- зарезервированы
        \item содержащие двойное подчеркивание -- зарезервированы
        \item начинающиеся с подчеркивания за которым следует заглавная буква -- зарезервированы
        \item начинающиеся с подчеркивания -- зарезервированы в глобальном пространстве имен
      \end{itemize}
  \end{itemize}

\end{frame}

\begin{frame}{Выражения}

Выражение - последовательность операторов и операндов, представляющее некоторое вычисление.\vspace{1em}

Может производить результат и побочные эффекты.\vspace{1em}

Выражение характеризуется типом и категорией значения.\vspace{1em}

Категории значения:
\begin{enumerate}
  \item lvalue
  \item prvalue
  \item xvalue
\end{enumerate}

\end{frame}

\begin{frame}{Категории значения}

\begin{enumerate}
  \item lvalue - функция или объект \vspace{1em}
  \item xvalue - ссылка на объект, время жизни которого истекает \vspace{1em}
  \item glvalue - lvalue или xvalue \vspace{1em}
  \item rvalue - xvalue, временный объект, часть временного объекта или значение, не связанное с объектом \vspace{1em}
  \item prvalue - rvalue, не являющееся xvalue \vspace{1em}
\end{enumerate}

\end{frame}

\begin{frame}{Порядок выполнения}

\url{https://en.cppreference.com/w/cpp/language/eval_order}

\end{frame}

\begin{frame}[fragile]{Инструкции}

\begin{itemize}

  \item Инструкция-выражение -- выражение, завершенное \lstinline[basicstyle=\ttfamily\small]{;}
  
  \item Блоки -- последовательность инструкций, заключенная в \lstinline[basicstyle=\ttfamily\small]| { } |
  
  \item Ветвления и множественный выбор
    \begin{lstlisting}[basicstyle=\ttfamily\small]
      if
      switch ([init statement] condition) block
    \end{lstlisting}

  \item Циклы
    \begin{lstlisting}[basicstyle=\ttfamily\small]
      while (condition) body
      do body while (condition);
      for (decl : init) body
      for
    \end{lstlisting}

  \item Объявления
  
  \item Try блоки

\end{itemize}

\end{frame}

\begin{frame}[fragile]{Объявления переменных}

Структура инструкции объявления:
\begin{enumerate}
  \item Уточняющий спецификатор: \lstinline[basicstyle=\ttfamily\small]{inline, constexpr, static, ...}
  \item Спецификация типа: \lstinline[basicstyle=\ttfamily\small]{int, const char *, auto, ...}
  \item Список объявлений через запятую, каждое состоит из
    \begin{itemize}
      \item Идентификатор
      \item Инициализация
        \begin{itemize}
          \item \lstinline[basicstyle=\ttfamily\small]{( ... )}
          \item \lstinline[basicstyle=\ttfamily\small]{= expression}
          \item \lstinline[basicstyle=\ttfamily\small]|{ initializer list }|
        \end{itemize}
    \end{itemize}
\end{enumerate}

В простейших случаях область видимости имени - блок, функция, пространство имен.

\end{frame}

%-------------------------------------------------

\section{Наглядный пример}

\begin{frame}{Постановка задачи}

\underline{Дано:}\\
Сложность действия $D$, дайспул $A$.\\
\vspace{1em}
\underline{Модель успеха:}\\
Для каждого брошенного кубика с выпавшим значением $V$ результат броска определяется как
\[
R =
\begin{cases}
  \text{Успех, если $V \ge D$}\\
  \text{Неудача, если $1 < V < D$}\\
  \text{Провал, если $V = 1$}\\
\end{cases}
\]
Исход действия, если $S$ - сумма успехов, $B$ - сумма провалов:
\[
O =
\begin{cases}
  \text{Успех, если $S > 0$ и $S > B$}\\
  \text{Критический провал, если $S = 0$ и $B > 0$}\\
  \text{Неудача, в иных случаях}
\end{cases}
\]
  
\end{frame}

\begin{frame}{Постановка задачи, продолжение}

\underline{Вычислить:}\\ \vspace{0.5em}
Для всех сложностей от 4 до 10 и дайспулов от 1 до 10\vspace{0.5em}
\begin{enumerate}
  \item Вероятность критического провала\vspace{0.5em}
  \item Вероятность неудачи\vspace{0.5em}
  \item Вероятность успеха\vspace{0.5em}
  \item Вероятности для каждого возможного числа успехов\vspace{0.5em}
\end{enumerate}

\end{frame}

\begin{frame}{Необходимые элементы}

\begin{enumerate}
  \item Генератор псевдослучайных чисел с равномерным распределением
  \item Типы данных: целые положительные, дробные
  \item Циклы, ветвления
  \item Задание параметров (количество экспериментов)
  \item Вывод результатов
\end{enumerate}

\end{frame}

\begin{frame}[fragile]{Генератор псевдослучайных чисел}

\lstinputlisting[firstline=6,lastline=6]{examples/monte_carlo.cpp}
\lstinputlisting[firstline=83,lastline=102]{examples/monte_carlo.cpp}
\lstinputlisting[firstline=128,lastline=135]{examples/monte_carlo.cpp}

\end{frame}

\begin{frame}[fragile]{Разбор аргументов командной строки}

\lstinputlisting[firstline=2,lastline=3]{examples/monte_carlo.cpp}
\lstinputlisting[firstline=101,lastline=126]{examples/monte_carlo.cpp}
\lstinputlisting[firstline=134,lastline=135]{examples/monte_carlo.cpp}

\end{frame}

\begin{frame}[fragile]{Использованные элементы}

\begin{enumerate}

\item Типы данных

\begin{itemize}
  \item Целые \lstinline[basicstyle=\ttfamily\small]{int, unsigned, unsigned long}
  \item Строковые \lstinline[basicstyle=\ttfamily\small]{char *}
  \item Спецификатор констант \lstinline[basicstyle=\ttfamily\small]{const}
\end{itemize}

\item Операторы

\begin{itemize}
  \item Сравнения \lstinline[basicstyle=\ttfamily\small]{==, !=, <, >, <=, >=}
  \item Логические \lstinline[basicstyle=\ttfamily\small]{&&, ||}
  \item Присваивания \lstinline[basicstyle=\ttfamily\small]{=}
  \item Инкремента и декремента \lstinline[basicstyle=\ttfamily\small]{++, --}
\end{itemize}

\item Цикл
\begin{lstlisting}[basicstyle=\ttfamily\small]
for (init statement; condition; next) body
\end{lstlisting}

\item Ветвление
\begin{lstlisting}[basicstyle=\ttfamily\small]
if ( [init statement] condition ) statement-true
if ( [init statement] condition ) statement-true else statement-false
\end{lstlisting}

\end{enumerate}

\end{frame}

\begin{frame}[fragile]{Проведение каждого вида экспериментов}

\lstinputlisting[firstline=11,lastline=18]{examples/monte_carlo.cpp}
\lstinputlisting[firstline=70,lastline=81]{examples/monte_carlo.cpp}

\end{frame}

\begin{frame}[fragile]{Основной цикл экспериментов}

\lstinputlisting[firstline=20,lastline=60, basicstyle=\ttfamily\srcsmallsize]{examples/monte_carlo.cpp}

\end{frame}

\begin{frame}[fragile]{Вычисление вероятностей}

\lstinputlisting[firstline=61,lastline=68, basicstyle=\ttfamily\footnotesize]{examples/monte_carlo.cpp}

\end{frame}

\begin{frame}[fragile]{Использованные элементы}

\begin{enumerate}

\item Типы данных

\begin{itemize}
  \item Дробные \lstinline[basicstyle=\ttfamily\small]{double}
  \item Массивы \lstinline[basicstyle=\ttfamily\small]{std::array<T, N>} требует \lstinline[basicstyle=\ttfamily\small]{#include <array>}
\end{itemize}

\item Операторы

\begin{itemize}
  \item Операция с присвоением \lstinline[basicstyle=\ttfamily\small]{+=, -=, *=, /=, %=, &=, |=, ^=, <<=, >>=}
  \item Доступ к элементу по индексу \lstinline[basicstyle=\ttfamily\small]{[]}
\end{itemize}

\item Алгоритм accumulate
\begin{lstlisting}[basicstyle=\ttfamily\small]
std::accumulate(Iterator begin, Iterator end, T init) -> T;
\end{lstlisting}

\end{enumerate}

\end{frame}

\begin{frame}[fragile]{Примеры результатов}

Число прогонов: 100
\begin{lstlisting}[basicstyle=\ttfamily\srcsize]
Difficulty: 6, ability: 1, Total: 1, Botch: 0.08, Failure: 0.41, Total success: 0.51, 1 successes: 0.51
Difficulty: 6, ability: 2, Total: 1, Botch: 0.07, Failure: 0.19, Total success: 0.74, 1 successes: 0.45, 2 successes: 0.29
Difficulty: 6, ability: 3, Total: 1, Botch: 0.04, Failure: 0.2, Total success: 0.76, 1 successes: 0.32, 2 successes: 0.33, 3 successes: 0.11
Difficulty: 6, ability: 4, Total: 1, Botch: 0.02, Failure: 0.13, Total success: 0.85, 1 successes: 0.25, 2 successes: 0.3, 3 successes: 0.21, 4 successes: 0.09
\end{lstlisting}

Число прогонов: 100000
\begin{lstlisting}[basicstyle=\ttfamily\srcsize]
Difficulty: 6, ability: 1, Total: 1, Botch: 0.09812, Failure: 0.39921, Total success: 0.50267, 1 successes: 0.50267
Difficulty: 6, ability: 2, Total: 1, Botch: 0.08731, Failure: 0.26173, Total success: 0.65096, 1 successes: 0.40115, 2 successes: 0.24981
Difficulty: 6, ability: 3, Total: 1, Botch: 0.06057, Failure: 0.19948, Total success: 0.73995, 1 successes: 0.31613, 2 successes: 0.30038, 3 successes: 0.12344
Difficulty: 6, ability: 4, Total: 1, Botch: 0.03678, Failure: 0.16127, Total success: 0.80195, 1 successes: 0.24723, 2 successes: 0.29147, 3 successes: 0.20126, 4 successes: 0.06199
\end{lstlisting}

\end{frame}

\begin{frame}[fragile]{Примеры результатов, продолжение}

\begin{lstlisting}[basicstyle=\ttfamily\srcsize]
Difficulty: 9, ability: 1, Total: 1, Botch: 0.100079, Failure: 0.699833, Total success: 0.200089, 1 successes: 0.200089
Difficulty: 9, ability: 2, Total: 1, Botch: 0.149929, Failure: 0.529866, Total success: 0.320205, 1 successes: 0.280211, 2 successes: 0.0399946
Difficulty: 9, ability: 3, Total: 1, Botch: 0.169026, Failure: 0.432979, Total success: 0.397995, 1 successes: 0.305919, 2 successes: 0.0840565, 3 successes: 0.0080193
Difficulty: 9, ability: 4, Total: 1, Botch: 0.169729, Failure: 0.377569, Total success: 0.452702, 1 successes: 0.307932, 2 successes: 0.120798, 3 successes: 0.0223816, 4 successes: 0.0015909
Difficulty: 9, ability: 5, Total: 1, Botch: 0.159743, Failure: 0.346425, Total success: 0.493832, 1 successes: 0.299783, 2 successes: 0.148219, 3 successes: 0.0399207, 4 successes: 0.0055922, 5 successes: 0.000317
Difficulty: 9, ability: 6, Total: 1, Botch: 0.14468, Failure: 0.328686, Total success: 0.526634, 1 successes: 0.28734, 2 successes: 0.167781, 3 successes: 0.0581719, 4 successes: 0.0119481, 5 successes: 0.0013281, 6 successes: 6.55e-05
Difficulty: 9, ability: 7, Total: 1, Botch: 0.127266, Failure: 0.318702, Total success: 0.554032, 1 successes: 0.273871, 2 successes: 0.180831, 3 successes: 0.0754381, 4 successes: 0.0202182, 5 successes: 0.0033358, 6 successes: 0.0003238, 7 successes: 1.45e-05
Difficulty: 9, ability: 8, Total: 1, Botch: 0.110265, Failure: 0.31263, Total success: 0.577105, 1 successes: 0.260438, 2 successes: 0.188735, 3 successes: 0.0909919, 4 successes: 0.0296128, 5 successes: 0.0063708, 6 successes: 0.0008812, 7 successes: 7.22e-05, 8 successes: 3e-06
Difficulty: 9, ability: 9, Total: 1, Botch: 0.0938176, Failure: 0.308309, Total success: 0.597873, 1 successes: 0.247572, 2 successes: 0.19368, 3 successes: 0.104544, 4 successes: 0.039455, 5 successes: 0.0104714, 6 successes: 0.0019022, 7 successes: 0.0002323, 8 successes: 1.53e-05, 9 successes: 5e-07
Difficulty: 9, ability: 10, Total: 1, Botch: 0.0791378, Failure: 0.305126, Total success: 0.615736, 1 successes: 0.235399, 2 successes: 0.195354, 3 successes: 0.115828, 4 successes: 0.0496338, 5 successes: 0.0154408, 6 successes: 0.0034679, 7 successes: 0.0005549, 8 successes: 5.4e-05, 9 successes: 3.3e-06, 10 successes: 0
\end{lstlisting}

\end{frame}

\begin{frame}[fragile]{Базовые типы данных}

\begin{itemize}
  \item Пустой
    \lstinline[basicstyle=\ttfamily\small]{void}
  \item Логический
    \lstinline[basicstyle=\ttfamily\small]{bool}
  \item Целые
    \begin{itemize}
      \item \lstinline[basicstyle=\ttfamily\small]{short}
      \item \lstinline[basicstyle=\ttfamily\small]{int}
      \item \lstinline[basicstyle=\ttfamily\small]{long}
      \item \lstinline[basicstyle=\ttfamily\small]{unsigned long long}
    \end{itemize}
  \item Символы
    \begin{itemize}
      \item \lstinline[basicstyle=\ttfamily\small]{char}
      \item \lstinline[basicstyle=\ttfamily\small]{unsigned char}
    \end{itemize}
  \item Дробные
    \begin{itemize}
      \item \lstinline[basicstyle=\ttfamily\small]{float}
      \item \lstinline[basicstyle=\ttfamily\small]{double}
      \item \lstinline[basicstyle=\ttfamily\small]{long double} (обычно 80 бит)
    \end{itemize}
  \item Строковые
    \begin{itemize}
      \item \lstinline[basicstyle=\ttfamily\small]{char *}
    \end{itemize}
\end{itemize}

\end{frame}

\begin{frame}[fragile]{Литералы}

Литералы -- это представление иммутабельных \emph{значений} в коде.

\begin{itemize}

  \item Целочисленные: \lstinline[basicstyle=\ttfamily\small]{10, 077, 0xC, 0b1011}\\
  Всегда положительны. Необязательный суффикс может уточнить тип:
    \begin{itemize}
      \item \lstinline[basicstyle=\ttfamily\small]{u / U} - беззнаковый тип
      \item \lstinline[basicstyle=\ttfamily\small]{l / L} - \lstinline[basicstyle=\ttfamily\small]{long}
      \item \lstinline[basicstyle=\ttfamily\small]{ll / LL} - \lstinline[basicstyle=\ttfamily\small]{long long}
      \item комбинация размера и беззнаковости
    \end{itemize}
    
  \item Символьные \lstinline[basicstyle=\ttfamily\small]{'a', '\n'}
  
  \item Дробные: \lstinline[basicstyle=\ttfamily\small]{0.0, 11., 3e5, 123.45e-11, .1E3, 0x1.p3}
  
  \item Строковые: \lstinline[basicstyle=\ttfamily\small]{"abc\nDEF"}
  
  \item Логические: \lstinline[basicstyle=\ttfamily\small]{true, false}
  
  \item Нулевой указатель: \lstinline[basicstyle=\ttfamily\small]{nullptr}

\end{itemize}

\end{frame}

\begin{frame}{Операторы и их приоритет}

\url{https://en.cppreference.com/w/cpp/language/operator_precedence}

\end{frame}

\end{document}