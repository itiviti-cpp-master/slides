\documentclass[unknownkeysallowed,xcolor=table]{beamer}
 
\usepackage [T2A,T1] {fontenc}
\usepackage[utf8]{inputenc}
\usepackage[english,russian]{babel}
\usepackage{amsmath}
\usepackage{listings}
\usepackage{url}
\usepackage{textcomp}
\usepackage{multirow}

\setbeamertemplate{navigation symbols}{}

\newcommand{\textapprox}{\raisebox{0.5ex}{\texttildelow}}

\lstnewenvironment{cmdline}
  {\lstset{basicstyle=\ttfamily,keywordstyle=\color{blue},
           language={bash}}}
  {}
  
\colorlet{mygreen}{green!60!blue}
\colorlet{mymauve}{red!60!blue}
\definecolor{pblue}{rgb}{0.1, 0.2, 0.8}

\newcommand{\rarr}{$\rightarrow$}

\lstset{
      basicstyle=\ttfamily\small,
      commentstyle=\color{mygreen},
      keywordstyle=\color{blue},
      numberstyle=\tiny\color{blue},
      stringstyle=\color{mymauve},
      columns=fullflexible,
      breaklines=true,
      postbreak=\mbox{\textcolor{red}{\ensuremath{\hookrightarrow}\space}},
      literate={~} {\textapprox}{1},
      language={[11]C++}
}

\makeatletter
\newcommand{\srcmediumsize}{\@setfontsize{\srcmediumsize}{7pt}{7pt}}
\makeatother

\makeatletter
\newcommand{\srcbigsize}{\@setfontsize{\srcbigsize}{8pt}{8pt}}
\makeatother

\makeatletter
\newcommand{\srcsize}{\@setfontsize{\srcsize}{6pt}{6pt}}
\makeatother

\makeatletter
\newcommand{\srcsmallsize}{\@setfontsize{\srcsmallsize}{5pt}{5pt}}
\makeatother

\title[C++]
{Программирование на языке C++}
 
\subtitle{Вводный курс}
 
\author[А.~Б.~Морозов]
{
  \texorpdfstring{Александр Морозов\newline\href{mailto:gelu.speculum@gmail.com}{gelu.speculum@gmail.com}}
  {Александр Морозов}
}
  
\date[ITMO 2019]
{ИТМО, весенний семестр 2019}
 
\logo{%
  \makebox[0.95\paperwidth]{%
    \includegraphics[align=c,width=2cm,keepaspectratio]{itmo_logo.png}
    \hfill
    \includegraphics[align=c,width=1.1cm,keepaspectratio]{itiviti_logo.png}
  }
}

\AtBeginSection[]
{
  \begin{frame}
    \frametitle{Содержание}
    \tableofcontents[currentsection]
  \end{frame}
}

\begin{document}

\frame{\titlepage}

%-------------------------------------------------

\section{Функции}

\begin{frame}{Функции}

Функция -- это именованная сущность имеющая тело, список аргументов и тип возвращаемого значения. \vspace{1.5em}

Тело -- обычно, блок инструкций. \vspace{1.5em}

Список аргументов -- ноль или более объявлений аргументов, разделенных запятой. \vspace{1.5em}

Тип возвращаемого значения: \lstinline{void} либо любой допустимый тип, кроме функции или C-массива. \vspace{1.5em}

Тип самой функции можно записать, как \\ \vspace{0.5em}
\emph{ret} \textbf{(} \emph{params} \textbf{)}

\end{frame}

\begin{frame}{Объявление и определение функций}

Для функций различаются объявление функции и определение функции. \vspace{2em}

Объявление (declaration) -- это задание типа функции и связывание типа с именем функции. \vspace{1.5em}

Определение (definition) -- это описание тела функции.

\end{frame}

\begin{frame}[fragile]{Форма объявления функции}

\emph{ret name} \textbf{(} \emph{params} \textbf{);} \vspace{1.5em}

\lstinline{auto} \emph{name} \textbf{(} \emph{params} \textbf{)} \lstinline{->} \emph{ret} \textbf{;} \vspace{1.5em}

\emph{ret} -- указание типа возвращаемого значения, либо \lstinline{auto}, тогда тип будет выведен компилятором на основе выражений в инструкции \lstinline{return}. \vspace{1.5em}

Объявление функции можно вставить в более широкую инструкцию объявления: 

\begin{lstlisting}
int a = 5, foo();
\end{lstlisting}

\end{frame}

\begin{frame}[fragile]{Объявление аргументов}

Список параметров - последовательность объявлений аргументов, разделенных запятой.

\begin{itemize}
  \item обычное объявление переменной: \lstinline{type name} \vspace{0.5em}
  \item объявление с инициализацией: \lstinline{type name = init} -- аргумент со значением по умолчанию \vspace{0.5em}
  \item объявление без имени: \lstinline{type} \vspace{0.5em}
  \item объявление без имени с инициализацией: \lstinline{type = init}
\end{itemize}

\vspace{0.7em}
Variadic functions: в конце списка параметров можно поставить \lstinline{...}, это будет означать список параметров переменной длины:

\begin{lstlisting}
int printf(const char * fmt, ...);
int print(...);
\end{lstlisting}

\end{frame}

\begin{frame}{Дополнительные особенности объявления аргументов}

При определении типа функции в объявлении типов аргументов производятся следующие замены:
\begin{itemize}
  \item типы, обозначающие C-массивы или функции заменяются на указатели соответственно на тип элемента массива или на функцию \vspace{0.7em}
  \item cv-квалификатор верхнего уровня отбрасывается
\end{itemize}
\vspace{0.7em}

Значения по умолчанию являются частью объявления функции, будут проанализированы (например, будет произведен поиск имен) в точке объявления, но будут использованы в момент вызова функции.\\ \vspace{0.7em}

В инициализаторе значения по умолчанию нельзя использовать локальные переменные.

\end{frame}

\begin{frame}[fragile]{Примеры объявлений функций}

\begin{lstlisting}
int foo(int a, const char * b, std::string &, char c = '\0', double = 0.5);

// declare the same function of type void(int)
void bar(const int x);
void bar(const volatile int x);
void bar(int x);

int a = 1;
int f(int = a);
void b()
{
  a = 2;
  {
    int a = 3;
    f(); // call f(2)
  }
}
\end{lstlisting}

\end{frame}

\begin{frame}{Определение функции}

Объявление, за которым следует определение тела.

Определение тела -- это:

\begin{itemize}
  \item блок \vspace{0.7em}
  \item try block функции \vspace{0.7em}
  \item \lstinline{= delete} \vspace{0.7em}
  \item \lstinline{= default}
\end{itemize}
\vspace{0.7em}

Аргументы функции работают как обычные переменные внутри тела функции. Их область видимости начинается с момента объявления аргумента и заканчивается концом тела функции.

Время жизни -- от момента инициализации при вызове функции и до конца исполнения тела функции.

\end{frame}

\begin{frame}[fragile]{Инструкция return}

\begin{itemize}
  \item \lstinline{return} \emph{выражение} \lstinline{;}
  \item \lstinline|return {| \emph{список инициализации} \lstinline|};|
\end{itemize}

\vspace{1em}

где \emph{выражение} используется для вычисления возвращаемого значения и не обязательно в \lstinline{void} функциях и запрещено в конструкторах/деструкторах.

\vspace{1em}

В функциях без возвращаемого значения \lstinline{return} в конце тела можно опустить.

\vspace{1em}

В функции \lstinline{main} можно опустить \lstinline{return x;}, тогда в конце тела выполнится \lstinline{return 0;}.

\end{frame}

\begin{frame}[fragile]{Пример определения функции}

\begin{lstlisting}
unsigned mod_div(unsigned, unsigned);
unsigned mod_div(unsigned, unsigned = 10);
unsigned mod_div(unsigned = 0xABCDEF, unsigned);

unsigned mod_div(unsigned v, const unsigned n)
{
    const unsigned d = (1 << n) - 1;
    unsigned m;
    for (m = v; v > d; v = m) {
        for (m = 0; v; v >>= n) {
            m += v & d;
        }
    }
    return m == d ? 0 : m;
}

unsigned x = mod_div();
\end{lstlisting}

\end{frame}

%-------------------------------------------------

\section{Единицы трансляции}

\begin{frame}{Единица трансляции}

Один cpp файл после обработки препроцессором составляет единицу трансляции, т.е. выделенную часть программы, которую компилятор обрабатывает независимо от других.

\vspace{1.5em}

Единица трансляции должна содержать объявления всех используемых сущностей.

\vspace{1.5em}

Порядок инициализации глобальных переменных:
\begin{itemize}
  \item в рамках одной единицы трансляции -- в порядке определения
  \item между единицами трансляции -- не определен
\end{itemize}

\end{frame}

%-------------------------------------------------

\section{Объявление и определение}

\begin{frame}{Объявление и определение}

В некоторых случаях объявление не идентично определению:
\begin{itemize}
  \item объявление функций \vspace{0.3em}
  \item предварительное объявление классов \vspace{0.3em}
  \item предварительное объявление enum \vspace{0.3em}
  \item объявление внешней (extern) переменной \vspace{0.3em}
  \item объявление статического члена класса \vspace{0.3em}
  \item объявление шаблонного параметра \vspace{0.3em}
  \item объявление псевдонима типа \vspace{0.3em}
  \item \lstinline{using} объявление  \vspace{0.3em}
  \item явное объявление инстанциации шаблона \vspace{0.3em}
  \item специализация шаблона без определения
\end{itemize}

\end{frame}

\begin{frame}{Объявление и определение}

Объявление вводит в употребление некоторое имя (переменной, функции и т.д.). В программе может быть много объявлений одного и того же имени, если они не противоречат друг другу.

\vspace{3em}

Определение полностью описывает сущность, связанную с именем, например, определяет тело функции.

\end{frame}

\begin{frame}{Уникальность определений}

One definition rule (ODR)

\vspace{2em}

В одной единице трансляции допустимо лишь одно определение любой переменной, функции, класса, enum или шаблона.

\vspace{2em}

Во всей программе допустимо лишь одно определение функции или переменной, которая ODR-используется. Инструментарий трансляции не обязан проверять это условие, но поведение программы в случае его нарушения не определено (UB).

\end{frame}

\begin{frame}{Повторение определений}

Во всей программе может быть более одного определения класса, enum, шаблона, если:

\vspace{1em}

\begin{itemize}
  \item каждый вариант определения состоит из одинаковой последовательности токенов \vspace{1em}
  \item поиск имени, связанного с каждым таким определением, во всех случаях дает ту же сущность \vspace{1em}
  \item любые связанные операторы и методы должны давать вызов одной и той же функции в каждом случае
\end{itemize}

\end{frame}

\begin{frame}{ODR-использование}

\begin{itemize}
  \item объект ODR-используется, если его значение читается (если только это не константа с известным на момент компиляции значением), записывается, его адрес берётся или с ним связывается ссылка \vspace{2em}
  \item функция ODR-используется, если она вызывается или её адрес берётся
\end{itemize}

\end{frame}

\begin{frame}[fragile]{Пример объявления и определения}

\begin{lstlisting}
struct S
{
    static const int x = 0; // declaration
};

const int S::x; // definition

int main(int argc, char ** argv)
{
    const int & x = S::x; // S::x is ODR-used
    return S::x; // S::x is not ODR-used
}
\end{lstlisting}

\end{frame}

\begin{frame}{inline}

Определение функции можно дополнить спецификатором \lstinline{inline}. Тогда допустимо иметь много определений одной и той же функции в программе, но требуется, чтобы они состояли из одной и той же последовательности токенов. Также спецификатор \lstinline{constexpr} неявно подразумевает \lstinline{inline} для функций.

\vspace{1em}

Можно объявить глобальную переменную имеющую статическое размещение как \lstinline{inline}.

\vspace{2em}

Имеет смысл использовать для размещения определения функции или переменной в заголовочном файле.

\end{frame}

\begin{frame}{Linkage}

Имя, обозначающее объект, ссылку, функцию, тип, шаблон, пространство имен или значение, может иметь linkage (связывание).

\vspace{0.5em}

\begin{itemize}
  \item отсутствие связывания: имя является локальным для своей области видимости  \vspace{0.5em}
  \item внешнее связывание: имя в разных единицах трансляции ссылается на одну и ту же сущность \vspace{0.5em}
  \item внутреннее связывание: имя в любых областях видимости данной единицы трансляции ссылается на одну и ту же сущность
\end{itemize}

\end{frame}

\begin{frame}{Назначение связывания по умолчанию}

\begin{itemize}
  \item отсутствие связывания: любая локальная переменная без спецификатора extern, локальные классы и т.п. \vspace{1em}
  \item внешнее: глобальные не-const переменные, const или не-const глобальные inline переменные, функции вне зависимости от их объявления, классы \vspace{1em}
  \item внутреннее: глобальные const переменные, члены анонимных union, глобальные переменные, функции, классы объявленные в анонимном пространстве имен
\end{itemize}

\end{frame}

\begin{frame}[fragile]{Явное назначение связывания}

\begin{itemize}
  \item \lstinline{extern} -- спецификатор объявления переменных, позволяющий явно задать внешнее связывание \vspace{0.5em}
  \item \lstinline{static} -- спецификатор определения глобальных переменных или функций, явно задающий внутреннее связывание
\end{itemize}

\begin{lstlisting}
extern int x; // declaration, not a definition
extern int y = 101; // external
int z = 10; // external
static int u = 33; // internal
const int w = -99; // internal
extern const int v = 1; // external
static const int t = 222; // internal
\end{lstlisting}

\end{frame}

\begin{frame}[fragile]{Language linkage}

Любое имя с внешним связыванием неявно обладает т.н. ``language linkage''.

\vspace{1em}

Можно задать явно: \\

\vspace{0.7em}

\lstinline{extern} \emph{string-literal} \lstinline|{ ... }|

\vspace{1em}

где \emph{string-literal}:

\begin{itemize}
  \item \lstinline{"C++"}
  \item \lstinline{"C"}
\end{itemize}

\end{frame}

%-------------------------------------------------

\section{Препроцессор}

\begin{frame}[fragile]{Директивы препроцессора}

\lstinline{#} \emph{директива [аргументы]}

\vspace{1em}

Директива должна размещаться на одной строке, занимает всю строку.

\begin{itemize}
  \item \lstinline{define}
  \item \lstinline{undef}
  \item \lstinline{include}
  \item \lstinline{if}, \lstinline{ifdef}, \lstinline{ifndef}, \lstinline{else}, \lstinline{elif}, \lstinline{endif}
  \item \lstinline{error}
  \item \lstinline{pragma}
  \item \lstinline{line}
\end{itemize}

\end{frame}

\begin{frame}[fragile]{error}

\lstinline{#error message}

\vspace{2em}

Директива \lstinline{#error message} прерывает процесс трансляции с печатью соответствующего сообщения.

\end{frame}

\begin{frame}[fragile]{line}

\begin{itemize}
  \item \lstinline{#line lineno}
  \item \lstinline{#line lineno "filename"}
\end{itemize}

\vspace{2em}

Директива \lstinline{#line lineno} меняет текущее значение номера строки в препроцессоре на lineno.

\vspace{0.5em}

Директива \lstinline{#line lineno "filename"} меняет текущее значение номера строки и текущее имя файла.

\vspace{1em}

Препроцессор автоматически определяет макросы \lstinline{__LINE__} и \lstinline{__FILE__}, раскрытие которых дает эти значения.
Может быть полезно для внешних кодогенераторов.

\end{frame}

\begin{frame}[fragile]{pragma}

Директива \lstinline{#pragma params} дает доступ к функционалу, определенному в конкретной реализации транслятора.

\vspace{1em}

Например, \lstinline{#pragma pack(x)} позволяет явно задать padding для сложных структур данных.

\end{frame}

\begin{frame}[fragile]{Условные директивы}

\begin{lstlisting}
#ifdef identifier
#ifndef identifier
#if expression
#elif expression
#else
#endif
\end{lstlisting}

\vspace{0.5em}

\lstinline{#ifdef} и \lstinline{#ifndef} -- проверяют, является ли идентификатор определенным идентификатором (в случае \lstinline{#ifndef} -- наоборот) препроцессора.

\lstinline{#if} и \lstinline{#elif} -- проверяют, преобразуется ли выражение к ненулевому значению.

\lstinline{#else} открывает альтернативный блок.

\lstinline{#endif} завершает всю конструкцию.

\vspace{0.5em}

Конструкции могут быть вложенными.

\end{frame}

\begin{frame}[fragile]{Условные директивы, конструкция}

\begin{lstlisting}
#ifdef / ifndef / if
// main case
...
#elif ...
// alternative case
...
#else
// else case
...
#endif
\end{lstlisting}

\vspace{1em}

По результатам проверки условий, препроцессор выбирает одну ``ветку'' и вся конструкция замещается текстом этой ветки (остальные удаляются).

\end{frame}

\begin{frame}[fragile]{Условные директивы, выражения}

Выражение \lstinline{#if} / \lstinline{#elif} должно быть константным, может дополнительно включать в себя операторы

\begin{itemize}
  \item \lstinline{defined id} или \lstinline{defined(id)}: результат этих операторов является 1, если идентификатор является определенным идентификатором препроцессора, иначе -- 0
  \item \lstinline{__has_include(...)}: результат 1, если препроцессор обнаруживает указанный включаемый файл, иначе -- 0
\end{itemize}

После раскрытия макросов, выполнения операторов \lstinline{defined}/\lstinline{__has_include}, любой идентификатор в выражении (\lstinline{true}/\lstinline{false}) заменяется на 0. Если результат полученного выражения равен нулю, то условие считается не выполненным, иначе -- выполненным.

\end{frame}

\begin{frame}[fragile]{Условные директивы, пример}

\begin{lstlisting}
#if !defined(FOO)
xxx
#elif X == 3
yyy
#else
#  ifndef BAR
zzz
#  endif
#endif
\end{lstlisting}

\end{frame}

\begin{frame}[fragile]{include}

\begin{itemize}
  \item \lstinline{#include <} \emph{файл} \lstinline{>} -- ``стандартные'' или ``глобальные'' файлы
  \item \lstinline{#include "} \emph{файл} \lstinline{"} -- ``локальные'' файлы
\end{itemize}

\vspace{2em}

Препроцессор заменяет такую директиву на содержимое указанного файла.

Можно указывать относительный путь.

\vspace{1em}

Выражение для \lstinline{__has_include} имеет такую же форму и смысл.

\vspace{1em}

Файлы будут включаться рекурсивно.

\end{frame}

\begin{frame}[fragile]{Заголовочные файлы}

Используются для выделения объявлений сущностей, используемых в разных единицах трансляции.

\vspace{1em}

Это примитивная подстановка текста целого файла -- нет защиты от повторного включения.

\vspace{1em}

``Include guards'':
\begin{lstlisting}
#ifndef SOME_UNIQUE_HEADER_NAME
#define SOME_UNIQUE_HEADER_NAME

...
#endif
\end{lstlisting}

\vspace{0.5em}

Альтернатива:
\begin{lstlisting}
#pragma once
\end{lstlisting}

\end{frame}

\begin{frame}[fragile]{Макросы}

Осуществляют текстовую подстановку.

\vspace{1em}

Директива \lstinline{#define} определяет идентификатор препроцессора, а \lstinline{#undef} -- удаляет определение.

\vspace{0.5em}

\begin{lstlisting}
#define identifier
#define identifier replacement
#define identifier(parameters) replacement
#define identifier(parameters, ...) replacement
#define identifier(...) replacement
#undef identifier
\end{lstlisting}

\vspace{1em}

В случае простых макросов препроцессор заменяет встреченный идентификатор на его строку замены.

\end{frame}

\begin{frame}[fragile]{Сложные макросы}

Сложные макросы -- примитивный аналог функций.

Препроцессор заменяет встреченные вхождения идентификатора с параметрами на строку замены, затем заменяет в строке замены идентификаторы параметров на указанные.

\begin{lstlisting}
#define MAX(a, b) a < b ? b : a

if (MAX(x, y) == 0) {
  ...
}

->

if (x < y ? y : x == 0) {
  ...
}
\end{lstlisting}

\end{frame}

\begin{frame}[fragile]{Ограничения макросов}

Рекурсия запрещена

\begin{lstlisting}
#define foo foo

foo
->
foo
\end{lstlisting}

\vspace{2em}

Вложенные вызовы разрешены

\begin{lstlisting}
#define foo(x) x

foo(foo(foo(1)))
->
1
\end{lstlisting}

\end{frame}

\begin{frame}{Особенности подстановки параметров}

В подстановке сложных макросов параметры дополнительно раскрываются.

\vspace{1em}

Этапы подстановки:
\vspace{0.5em}
\begin{enumerate}
  \item Определение мест вхождения параметров в строке замены и их первичная подстановка \vspace{0.7em}
  \item Раскрытие параметров в местах их подстановки \vspace{0.7em}
  \item В строке замены обрабатываются макросы (при этом явно запрещена рекурсия)
\end{enumerate}

\end{frame}

\begin{frame}[fragile]{Специальные операции подстановки}

В подстановке сложных макросов \lstinline{#} и \lstinline{##} играют особую роль:

\vspace{2em}

\begin{itemize}
  \item stringification (литерация): \lstinline{#a}, где \lstinline{a} -- параметр макроса; вместо обычной подстановки параметра, параметр превращается в корректный строковый литерал (с обрамляющими \lstinline{""}) и подставляется в строку замены \vspace{1em}
  \item concatenation (склейка): \lstinline{##} между любыми двумя параметрами или между параметром и другим токеном вызывает конкатенацию результатов их подстановки (препроцессор проверит корректность полученного токена)
\end{itemize}

\end{frame}

\begin{frame}[fragile]{Примеры stringification и concatenation}

\begin{lstlisting}
#define show(a, b, c) #a #b #c

show(x, 10, "hello\n")
->
"x10\"hello\\n\""


#define print(type) void print_ ## type (type x) { ... }

print(char)
->
void print_char(char x) { ... }
\end{lstlisting}

\end{frame}

\begin{frame}[fragile]{Особенности stringification и concatenation}

При этом \lstinline{#} и \lstinline{##} оказывает влияние на обработку параметров макросов: литерация или склейка производятся \emph{до} возможного раскрытия параметров.

\vspace{1em}

\begin{lstlisting}
#define foo ABC
#define concat(a) foo ## a

concat(__LINE__)
->
foo__LINE__
\end{lstlisting}

\vspace{1em}

При этом ни макрос \lstinline{foo}, макрос \lstinline{__LINE__} уже не будут обработаны.

\end{frame}

\begin{frame}[fragile]{Обход особенностей stringification и concatenation}

Обойти раннюю обработку операций \lstinline{#} и \lstinline{##} можно двойным перенаправлением:

\begin{lstlisting}
#define show(x) do_show(x)
#define do_show(x) #a

show(__LINE__)
->
"123"

#define concat(a, b) do_concat(a, b)
#define do_concat(a, b) a ## b

concat(0x, __LINE__)
->
0x123
\end{lstlisting}

\end{frame}

\begin{frame}[fragile]{Защита параметров макросов}

Неизвестно, что будет передано в качестве параметров макроса -- можно отчасти защититься:

\vspace{2em}

Неправильное синтаксическое дерево:
\begin{lstlisting}
#define MAX(a, b) a < b ? b : a

MAX(0, x & 0xFF)
->
0 < x & 0xFF ? x & 0xFF : 0
\end{lstlisting}

\vspace{1em}

\begin{lstlisting}
#define MAX(a, b) (a) < (b) ? (b) : (a)
MAX(0, x & 0xFF)
->
(0) < (x & 0xFF) ? (x & 0xFF) : (0)
\end{lstlisting}

\end{frame}

\begin{frame}[fragile]{Защита тела макросов}

\begin{lstlisting}
#define MAX(a, b) (a < b ? b : a)

if (MAX(x, y) == 0) {
  ...
}
->
if ((a < b ? b : a) == 0) {
  ...
}
\end{lstlisting}

\end{frame}

\begin{frame}[fragile]{Защита места вставки}

Неизвестно, куда макрос будет вставлен, часто желательно гарантировать, что это будет одна инструкция:

\vspace{1em}

\begin{lstlisting}
#define PRINT(x) do { \
    ...; \
    ...; \
} while (false)

// Now this is OK:
if (...)
    PRINT(...);
else
    ...;
\end{lstlisting}

\end{frame}

\begin{frame}[fragile]{Побочные эффекты в макросах}

Если передать в качестве параметра макроса выражение, оно будет исполнено в каждом месте подстановки:

\vspace{1em}

\begin{lstlisting}
#define MAX(a, b) (a) < (b) ? (b) : (a)

MAX(x++, --y);
->
(x++) < (--y) < (--y) : (x++);
\end{lstlisting}

\end{frame}

%-------------------------------------------------

\section{Наглядный пример}

\begin{frame}{CBOE BOE}

\url{https://cdn.batstrading.com/resources/participant_resources/BATS_Europe_BOE2_Specification.pdf}

\end{frame}

\begin{frame}{Типы данных}

Все бинарные типы кодируются little-endian.

\vspace{1em}

\begin{center}
\begin{tabular}{ |l|c|m{18em}| }
  \hline
    Тип & Размер & Описание \\
  \hline
    Binary & 1,2,4,8 & unsigned \\
    Binary price & 8 & signed, 4 неявных десятичных знака после запятой \\
    Alphanum & 1 & Латинские буквы и цифры \\
    Text & N & Последовательность ASCII символов, завершенная нулем \\
  \hline
\end{tabular}
\end{center}

Пример: \\
Binary price\hspace{1em} 12.45678 \rarr\hspace{1em} 97 E6 01 00 00 00 00 00 \\

\end{frame}

\begin{frame}{Структура сообщения New order}

\begin{columns}
\column{\dimexpr\paperwidth-10pt}
\begin{center}
\begin{tabular}{ |m{5em}|c c c |m{10em}| }
  \hline
    Поле & Смещ. & Размер & Тип & Описание \\
  \hline
    Start & 0 & 2 & Binary & 0xBA 0xBA \\[4pt]
    Message Length & 2 & 2 & Binary & Длина сообщения в байтах, не считая Start \\[4pt]
    Message Type & 4 & 1 & Binary & 0x38 \\[4pt]
    Matching Unit & 5 & 1 & Binary & 0x00 \\[4pt]
    Sequence Number & 6 & 4 & Binary & Порядковый номер сообщения \\[4pt]
    ClOrdID & 10 & 20 & Text & Уникальный идентификатор заказа \\[4pt]
    Side & 30 & 1 & Alphanum & '1' - Buy, '2' - Sell \\[4pt]
  \hline
\end{tabular}
\end{center}
\end{columns}

\end{frame}

\begin{frame}{Структура сообщения New order, продолжение}

\begin{columns}
\column{\dimexpr\paperwidth-10pt}
\begin{center}
\begin{tabular}{ |m{5em}|c c c |m{10em}| }
  \hline
    Поле & Смещ. & Размер & Тип & Описание \\
  \hline
    Order Qty & 31 & 4 & Binary & Объем заказа \\[4pt]
    Num of bitfields & 35 & 1 & Binary & Число использованных битовых масок \\[4pt]
    Bitfield1 & 36 & 1 & Binary & Битовая маска 1 \\[4pt]
    ... & ... & ... & ... & ... \\
    BitfieldN & 36 & 1 & Binary & Битовая маска N \\[4pt]
    Optional fields... & 37 & & & \\
  \hline
\end{tabular}
\end{center}
\end{columns}

\end{frame}

\begin{frame}{Опциональные поля New order}

\begin{center}
\begin{tabular}{ |c|c|c| }
  \hline
    Byte & Bit  & Field \\
  \hline
    \multirow{8}{*}{1} & 1 & ClearingFirm \\
      & 2 & ClearingAccount \\
      & 4 & Price \\
      & 8 & ExecInst \\
      & 16 & OrdType \\
      & 32 & TimeInForce \\
      & 64 & MinQty \\
      & 128 & MaxFloor \\
    \hline
    \multirow{8}{*}{2} & 1 & Symbol \\
      & 2 & -- \\
      & 4 & Currency \\
      & 8 & IdSource \\
      & 16 & SecurityId \\
      & 32 & SecurityExchange \\
      & 64 & Capacity \\
      & 128 & RoutingInst \\
  \hline
\end{tabular}
\end{center}

\end{frame}

\begin{frame}{Опциональные поля New order, продолжение}

\begin{center}
\begin{tabular}{ |c|c|c| }
  \hline
    Byte & Bit  & Field \\
  \hline
    \multirow{8}{*}{3} & 1 & Account \\
      & 2 & DisplayIndicator \\
      & 4 & -- \\
      & 8 & -- \\
      & 16 & PegDifference \\
      & 32 & PreventParticipantMatch \\
      & 64 & -- \\
      & 128 & ExpireTime \\
  \hline
\end{tabular}
\end{center}

\end{frame}

\begin{frame}{Используемые в примере поля}

\begin{center}
\begin{tabular}{ |l|c|m{12em}| }
  \hline
    Поле & Тип & Описание \\
  \hline
    Account & Text[16] & \\
    & & \\
    Capacity & Alphanum & 'A' - Agency, 'P' - Principal, 'R' - Riskless Principal \\
    & & \\
    MaxFloor & Binary[4] & \\
    & & \\
    OrdType & Alphanum & '1' - Market, '2' - Limit, 'P' - Pegged \\
    & & \\
    Price & Binary price & \\
    Symbol & Text[8] & \\
    & & \\
    TimeInForce & Alphanum & '0' - Day, '3' - IOC, '6' - GTD \\
  \hline
\end{tabular}
\end{center}

\end{frame}

\begin{frame}{Структура проекта}
\begin{itemize}
  \item main.cpp -- кодирование тестового сообщения и вывод на экран \vspace{0.5em}
  \item codec.h -- утилиты кодирования базовых типов \vspace{0.5em}
  \item fields.h и fields.inl -- утилиты кодирования полей протокола \vspace{0.5em}
  \item new\_order\_opt\_fields.inl -- перечень использованных полей сообщения New order \vspace{0.5em}
  \item requests.h и requests.cpp -- реализация кодирования сообщений протокола
\end{itemize}
\end{frame}

\begin{frame}[fragile]{codec.h}
\lstinputlisting[lastline=35,basicstyle=\ttfamily\srcsize]{examples/bats/codec.h}
\end{frame}

\begin{frame}[fragile]{codec.h, продолжение}
\lstinputlisting[firstline=37,basicstyle=\ttfamily\srcbigsize]{examples/bats/codec.h}
\end{frame}

\begin{frame}[fragile]{fields.inl}
\lstinputlisting[basicstyle=\ttfamily\scriptsize]{examples/bats/fields.inl}
\end{frame}

\begin{frame}[fragile]{fields.h}
\lstinputlisting[lastline=19,basicstyle=\ttfamily\scriptsize]{examples/bats/fields.h}
\end{frame}

\begin{frame}[fragile]{fields.h, продолжение}
\lstinputlisting[firstline=20,lastline=45,basicstyle=\ttfamily\srcmediumsize]{examples/bats/fields.h}
\end{frame}

\begin{frame}[fragile]{fields.h, продолжение следует}
\lstinputlisting[firstline=47,basicstyle=\ttfamily\srcmediumsize]{examples/bats/fields.h}
\end{frame}

\begin{frame}[fragile]{new\_order\_opt\_fields.inl}
\lstinputlisting[basicstyle=\ttfamily\small]{examples/bats/new_order_opt_fields.inl}
\end{frame}

\begin{frame}[fragile]{requests.h}
\lstinputlisting[lastline=32,basicstyle=\ttfamily\srcsize]{examples/bats/requests.h}
\end{frame}

\begin{frame}[fragile]{requests.h, продолжение}
\lstinputlisting[firstline=34,lastline=67,basicstyle=\ttfamily\srcsize]{examples/bats/requests.h}
\end{frame}

\begin{frame}[fragile]{requests.h, продолжение следует}
\lstinputlisting[firstline=69,basicstyle=\ttfamily\small]{examples/bats/requests.h}
\end{frame}

\begin{frame}[fragile]{requests.cpp}
\lstinputlisting[lastline=27,basicstyle=\ttfamily\srcmediumsize]{examples/bats/requests.cpp}
\end{frame}

\begin{frame}[fragile]{requests.cpp, продолжение}
\lstinputlisting[firstline=29,lastline=63,basicstyle=\ttfamily\srcsize]{examples/bats/requests.cpp}
\end{frame}

\begin{frame}[fragile]{requests.cpp, продолжение следует}
\lstinputlisting[firstline=65,lastline=74,basicstyle=\ttfamily\small]{examples/bats/requests.cpp}
\end{frame}

\begin{frame}[fragile]{requests.cpp, окончание}
\lstinputlisting[firstline=76,basicstyle=\ttfamily\srcsize]{examples/bats/requests.cpp}
\end{frame}

\begin{frame}[fragile]{main.cpp}
\lstinputlisting[lastline=30,basicstyle=\ttfamily\srcsize]{examples/bats/main.cpp}
\end{frame}

\begin{frame}[fragile]{main.cpp, продолжение}
\lstinputlisting[firstline=32,basicstyle=\ttfamily\small]{examples/bats/main.cpp}
\end{frame}

\begin{frame}[fragile]{Результат работы}

\begin{lstlisting}
ba ba 4c 00 38 00 01 00  00 00 4f 52 44 31 30 30
31 00 00 00 00 00 00 00  00 00 00 00 00 00 31 64
00 00 00 03 b4 41 01 7a  e8 01 00 00 00 00 00 32
30 0a 00 00 00 41 41 50  6c 00 00 00 00 50 41 43
43 33 33 31 00 00 00 00  00 00 00 00 00 00
\end{lstlisting}

\end{frame}

\end{document}
