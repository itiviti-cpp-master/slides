\documentclass[unknownkeysallowed,xcolor=table]{beamer}
 
\usepackage[T2A,T1]{fontenc}
\usepackage[utf8]{inputenc}
\usepackage[english,russian]{babel}
\usepackage{listings}
\usepackage{amsmath}
\usepackage{url}
\usepackage{textcomp}
\usepackage{multirow}
\usepackage{tikz}

\setbeamertemplate{navigation symbols}{}

\newcolumntype{C}[1]{>{\centering\let\newline\\\arraybackslash\hspace{0pt}}m{#1}}
\newcolumntype{S}[1]{>{\columncolor[HTML]{AAACED}\centering\let\newline\\\arraybackslash\hspace{0pt}}m{#1}}

\newcommand{\textapprox}{\raisebox{0.5ex}{\texttildelow}}

\newcommand{\rarr}{$\rightarrow$}
 
\colorlet{mygreen}{green!60!blue}
\colorlet{mymauve}{red!60!blue}
\definecolor{light-gray}{gray}{0.9}

\lstset{
      basicstyle=\ttfamily\small,
      commentstyle=\color{mygreen},
      keywordstyle=\color{blue},
      numberstyle=\tiny\color{blue},
      stringstyle=\color{mymauve},
      numbers=left,
      stepnumber=1,
      columns=fullflexible,
      breaklines=true,
      postbreak=\mbox{\textcolor{red}{\ensuremath{\hookrightarrow}\space}},
      literate={~} {\textapprox}{1},
      language={[11]C++}
}

\lstnewenvironment{cmdline}
  {\lstset{
      basicstyle=\ttfamily\scriptsize,
      keywordstyle=\color{blue},
      backgroundcolor=\color{light-gray},
      language={bash}
  }}
  {}

\lstnewenvironment{cmdlinelarge}
  {\lstset{
      basicstyle=\ttfamily\small,
      keywordstyle=\color{blue},
      backgroundcolor=\color{light-gray},
      language={bash}
  }}
  {}

\makeatletter
\newcommand{\srcmediumsize}{\@setfontsize{\srcmediumsize}{7pt}{7pt}}
\makeatother

\makeatletter
\newcommand{\srcbigsize}{\@setfontsize{\srcbigsize}{8pt}{8pt}}
\makeatother

\makeatletter
\newcommand{\srcsize}{\@setfontsize{\srcsize}{6pt}{6pt}}
\makeatother

\makeatletter
\newcommand{\srcsmallsize}{\@setfontsize{\srcsmallsize}{5pt}{5pt}}
\makeatother

\makeatletter
\newenvironment<>{btHighlight}[1][]
{\begin{onlyenv}#2\begingroup\tikzset{bt@Highlight@par/.style={#1}}\begin{lrbox}{\@tempboxa}}
{\end{lrbox}\bt@HL@box[bt@Highlight@par]{\@tempboxa}\endgroup\end{onlyenv}}

\newcommand<>\btHL[1][]{%
  \only#2{\begin{btHighlight}[#1]\bgroup\aftergroup\bt@HL@endenv}%
}
\def\bt@HL@endenv{%
  \end{btHighlight}%
  \egroup
}
\newcommand{\bt@HL@box}[2][]{%
  \tikz[#1]{%
    \pgfpathrectangle{\pgfpoint{1pt}{0pt}}{\pgfpoint{\wd #2}{\ht #2}}%
    \pgfusepath{use as bounding box}%
    \node[anchor=base west, fill=orange!30,outer sep=0pt,inner xsep=1pt, inner ysep=0pt, rounded corners=3pt, minimum height=\ht\strutbox+1pt,#1]{\raisebox{1pt}{\strut}\strut\usebox{#2}};
  }%
}
\makeatother

\title[C++]
{Программирование на языке C++}
 
\subtitle{Вводный курс}
 
\author[А.~Б.~Морозов]
{
  \texorpdfstring{Александр Морозов\newline\href{mailto:gelu.speculum@gmail.com}{gelu.speculum@gmail.com}}
  {Александр Морозов}
}
  
\date[ITMO 2022]
{ИТМО, весенний семестр 2022}
 
\logo{%
  \makebox[0.97\paperwidth]{%
    \includegraphics[align=c,width=2cm,keepaspectratio]{itmo_logo.png}
    \hfill
    \includegraphics[align=c,width=1.5cm,keepaspectratio]{itiviti_logo.png}
  }
}

\AtBeginSection[]
{
  \begin{frame}
    \frametitle{Содержание}
    \tableofcontents[currentsection]
  \end{frame}
}

\begin{document}
 
\frame{\titlepage}

%-------------------------------------------------
\section{Базовые элементы программы}

\begin{frame}[fragile]{Составляющие функции}
  \begin{lstlisting}[
      moredelim={**[is][\btHL<1>]{@1}{@}},
      moredelim={**[is][\btHL<2>]{@2}{@}},
      moredelim={**[is][\btHL<3>]{@3}{@}},
      moredelim={**[is][\btHL<4>]{@4}{@}}
    ]
    @1int@ @2main@@3(int argc, char ** argv)@
    {
      @4return 0;@
    }
  \end{lstlisting}
\end{frame}

\begin{frame}[fragile]{Вызов функций}
  \begin{lstlisting}[
      moredelim={**[is][\btHL<1>]{@1}{@}},
      moredelim={**[is][\btHL<2>]{@2}{@}},
      moredelim={**[is][\btHL<3>]{@3}{@}},
      moredelim={**[is][\btHL<4>]{@4}{@}}
  ]
  @1void@ f() { }

  void g(int, char, double@2 = 0.1@) { }

  int h(int a@2 = -1@)
  {
    @3g(a, 'x')@; // third parameter is 0.1
    return @4a + 2@;
  }

  int main()
  {
    @3f()@;
    @3g(h(), 'a', 0.5)@;
  }
  \end{lstlisting}
\end{frame}

\begin{frame}[fragile]{Блоки и объявления}
  \begin{lstlisting}[
      moredelim={**[is][\btHL<1>]{@1}{@}},
      moredelim={**[is][\btHL<2>]{@2}{@}},
      moredelim={**[is][\btHL<3>]{@3}{@}},
      moredelim={**[is][\btHL<4>]{@4}{@}},
      moredelim={**[is][\btHL<5>]{@4}{@}},
      moredelim={**[is][\btHL<6>]{@4}{@}},
      moredelim={**[is][\btHL<7>]{@4}{@}},
      moredelim={**[is][\btHL<8>]{@4}{@}},
      moredelim={**[is][\btHL<9>]{@4}{@}}
  ]
  int main(int argc, char ** argv)
  {
    @1int a;@
    {
      int @4b@ @5= a@;
      {
        @2int@ @3a, b = 101@;
      }
    }
    {
      long a, b = @61L@, c = @8-@@75@;
      char d = @9'X'@;
    }
  }
  \end{lstlisting}
\end{frame}

\begin{frame}[fragile]{Область видимости}
  \begin{lstlisting}
  int a = 11;

  void foo()
  {
    a++;
    {
      int a = a;
      a *= 3;
    }
  }

  int b;

  int main()
  {
    foo();
    return a + b;
  }
  \end{lstlisting}
  \footnote{\url{https://stackoverflow.com/questions/23415661/has-c-standard-changed-with-respect-to-the-use-of-indeterminate-values-and-und}}
\end{frame}

\begin{frame}[fragile]{Зависимости внутри одной инструкции объявления}
  \begin{lstlisting}
  int main()
  {
    int a = 11, b = a + 2;
  }
  \end{lstlisting}
  \vspace{1em}
  \url{https://stackoverflow.com/questions/24224115/interdependent-initialization-with-commas}
\end{frame}

\begin{frame}{Переменные, значения, объекты}
\end{frame}

\begin{frame}[fragile]{Время жизни объектов}
  \begin{lstlisting}
  int a = 11;

  int main()
  {
    int b;
    {
      int c = 11;
    }
    return b;
  }
  \end{lstlisting}
\end{frame}

\begin{frame}{Типы размещения}
  \begin{itemize}
    \item автоматический \vspace{1em}
    \item статический \vspace{1em}
    \item тред-локальный \vspace{1em}
    \item динамический
  \end{itemize}
\end{frame}

\begin{frame}[fragile]{Идентификаторы}
  \begin{itemize}
    \item \lstinline{[A-Za-z_][A-Za-z0-9_]*} \vspace{1em}
    \item совпадающие с ключевыми словами -- зарезервированы \vspace{1em}
    \item содержащие \lstinline{__} -- зарезервированы \vspace{1em}
    \item начинающиеся с \lstinline{_[A-Z]} -- зарезервированы \vspace{1em}
    \item начинающиеся с \lstinline{_} -- зарезервированы в глобальном пространстве имён
  \end{itemize}
\end{frame}

\begin{frame}{Составляющие текста программы}
  \begin{itemize}
    \item идентификаторы \vspace{2em}
    \item числовые литералы \vspace{2em}
    \item символьные и строковые литералы \vspace{2em}
    \item операторы и прочие символы пунктуации
  \end{itemize}
\end{frame}

\begin{frame}[fragile]{Имена}
  Имя -- идентифицирующее выражение, связанное с некой программной сущностью через определение. \vspace{2em}
  \begin{lstlisting}
    int a = 1; // declaration

    int f()
    {
        return a; // usage
    }
  \end{lstlisting}
  \vspace{2em}
  Использование \rarr поиск имён \rarr сущность
\end{frame}

\begin{frame}[fragile]{Литералы}
  \begin{itemize}
    \item булевские \lstinline{true}, \lstinline{false} \vspace{0.5em}
    \item целочисленные \vspace{0.5em}
    \item дробные \vspace{0.5em}
    \item символьные \lstinline{'a'} \vspace{0.5em}
    \item строковые \lstinline{"Hello\n"} \vspace{0.5em}
    \item \lstinline{nullptr}
  \end{itemize}
\end{frame}

\begin{frame}[fragile]{Выражения и операторы}
  \begin{lstlisting}[
      moredelim={**[is][\btHL<1>]{@1}{@}},
      moredelim={**[is][\btHL<2>]{@2}{@}},
      moredelim={**[is][\btHL<3>]{@3}{@}},
      moredelim={**[is][\btHL<4>]{@4}{@}}
  ]
  int main(int argc, char ** argv)
  {
    @1argc@@2++@ @2+@ @2++@@1argc@; // UB
    @3argc = ++argc * 3@;
    return @4(1 + 2 * 3)@ * @4argv@[@4argc@ - @41@];
  }
  \end{lstlisting}
\end{frame}

\begin{frame}{Список операторов и их свойства}
  \url{https://en.cppreference.com/w/cpp/language/operator_precedence}
\end{frame}

\begin{frame}[fragile]{Результат и побочные эффекты}
  \begin{lstlisting}
  int f(int a, int b)
  {
    return a + b;
  }

  int main()
  {
    int a = 0, b = -13;
    int c = a++ + ++b;
    b *= 2;
    return f(a, b);
  }
  \end{lstlisting}
\end{frame}

\begin{frame}[fragile]{Невычисляемый контекст}
  \begin{lstlisting}[
      moredelim={**[is][\btHL<1>]{@1}{@}}
  ]
  double f()
  {
    return 0.5;
  }

  int main()
  {
    decltype(@1f()@) a = f();
    auto b = f();
    return sizeof(@1b@);
  }
  \end{lstlisting}
\end{frame}

\begin{frame}[fragile]{Полное выражение}
  \begin{lstlisting}[
      moredelim={**[is][\btHL<1>]{@1}{@}}
  ]
  int main(int argc, char ** argv)
  {
    int a = @1argc + argc / 2, b = a@;
    decltype(@1a + 2@) c = 3;
    @1c += a / b@;
    return @1a + b@;
  }
  \end{lstlisting}
\end{frame}

\begin{frame}[fragile]{Константные выражения}
  \begin{lstlisting}
  int main()
  {
    const std::size_t len = 10;
    std::array<int, len> xxs;
  }
  \end{lstlisting}
\end{frame}

\begin{frame}[fragile]{Временные объекты}
  \begin{lstlisting}
  int & f(int & a) { return a; }

  int main(int argc, char ** argv)
  {
    return 11 + f(argc * 2);
  }
  \end{lstlisting}
\end{frame}

\begin{frame}[fragile]{Порядок исполнения}
  \begin{lstlisting}
  int main(int argc, char ** argv)
  {
    int a, b = argc * 3;
    a = argc++ + b;
    f(a, b);
    f(a++, a);
    bool x = a > 5 || b < 3;
    a = a++ + 2, b = a;
  }
  \end{lstlisting}
\end{frame}

\begin{frame}[fragile]{Контекст игнорирования результата}
  \begin{lstlisting}[
      moredelim={**[is][\btHL<1>]{@1}{@}}
  ]
  int main()
  {
    int a = 1, b = 2;
    @1a + b;@
    return @1a@, b;
  }
  \end{lstlisting}
\end{frame}

\begin{frame}[fragile]{Инструкции}
  \begin{lstlisting}
  int main(int argc, char ** argv)
  {
    int a = argc + 2;
    a *= 3;
    if (a < 3) {
      return 0;
    }
    if (a > 5)
    {
      a += 4;
    }
    else
      a -= 3;
    for (int i = 1; i < argc; ++i) {
      a += argv[i][0];
      continue;
      a -= 1;
    }
  }
  \end{lstlisting}
\end{frame}

\begin{frame}[fragile]{\lstinline{for}}
  \begin{lstlisting}[
      moredelim={**[is][\btHL<1>]{@1}{@}},
      moredelim={**[is][\btHL<2>]{@2}{@}},
      moredelim={**[is][\btHL<3>]{@3}{@}}
  ]
  int main(int argc, char ** argv)
  {
    for (@1int i = 0, j = 1;@ @2i + j < argc@; @3++i, ++j@) {
      i += 2;
      j -= 2;
    }
  }
  \end{lstlisting}
\end{frame}

\begin{frame}[fragile]{Минимальный \lstinline{for}}
  \begin{lstlisting}
  int main()
  {
    for (;;)
        ;
  }
  \end{lstlisting}
\end{frame}

\begin{frame}[fragile]{\lstinline{if}}
  \begin{lstlisting}
  int main(int argc, char ** argv)
  {
    if (argc > 3) {
    }
    if (int i, j, k; argc < 3) {
      i = 1;
      j = 2;
      k = 3;
    }
    else {
      k = 10;
    }
    return i + j; // error
  }
  \end{lstlisting}
\end{frame}

\begin{frame}[fragile]{\lstinline{switch}}
  \begin{lstlisting}
  int main(int argc, char ** argv)
  {
    int a = 0, b = 1;
    switch (argc) {
      case 1:
        a += 2;
        b -= 3;
      case 2:
        a *= b;
        break;
      case 3:
        return b;
      default:
        a += b * 3;
    }
  }
  \end{lstlisting}
\end{frame}

\begin{frame}[fragile]{Объявления и тело \lstinline{switch}}
  \begin{lstlisting}[basicstyle=\ttfamily\srcbigsize]
  int main(int argc, char ** argv)
  {
    switch (argc) {
      case 1:
        int a = 1;
        break;
      default:
        return argc;
    }
  }
  \end{lstlisting}
\end{frame}

\begin{frame}[fragile]{Объявления и тело \lstinline{switch}: правильно}
  \begin{lstlisting}
  int main(int argc, char ** argv)
  {
    switch (argc) {
      case 1: {
        int a = 1;
        break;
      }
      default:
        return argc;
    }
  }
  \end{lstlisting}
\end{frame}

\begin{frame}[fragile]{Ещё более странные вещи с \lstinline{switch}}
  \begin{lstlisting}[basicstyle=\ttfamily\srcbigsize]
    void g(const std::size_t count, char * to, const char * from)
    {
      std::size_t n = (count + 7) / 8;
      switch (count % 8) {
      case 0: do {  *to = *from++; [[fallthrough]];
      case 7:       *to = *from++; [[fallthrough]];
      case 6:       *to = *from++; [[fallthrough]];
      case 5:       *to = *from++; [[fallthrough]];
      case 4:       *to = *from++; [[fallthrough]];
      case 3:       *to = *from++; [[fallthrough]];
      case 2:       *to = *from++; [[fallthrough]];
      case 1:       *to = *from++;
              } while (--n);
      }
    }
  \end{lstlisting}
  \url{https://en.wikipedia.org/wiki/Duff's_device}
\end{frame}

%-------------------------------------------------
\section{Базовые типы}

\begin{frame}{Базовые типы}
  \begin{itemize}
    \item \lstinline{void} \vspace{0.5em}
    \item \lstinline{std::nullptr_t} \vspace{0.5em}
    \item арифметические \vspace{0.5em}
      \begin{itemize}
        \item дробные \vspace{0.5em}
        \item интегральные \vspace{0.5em}
          \begin{itemize}
            \item логический \lstinline{bool} \vspace{0.5em}
            \item символьные \lstinline{char}... \vspace{0.5em}
            \item знаковые целые \lstinline{int}... \vspace{0.5em}
            \item беззнаковые целые \lstinline{unsigned}... \vspace{0.5em}
          \end{itemize}
      \end{itemize}
  \end{itemize}
\end{frame}

\begin{frame}[fragile]{Требования к целым числовым типам}
  \scriptsize
  \centering
  \begin{tabular}{|C{8em}|C{6em}|C{6em}|S{3em}|S{3em}|S{3em}|}
    \hline
    \bfseries Тип & \bfseries Минимальное значение & \bfseries Максимальное значение & \bfseries ILP32 & \bfseries LP64 & \bfseries LLP64 \\
    \hline
    char & & & 8 & 8 & 8 \\
    signed char & -127 & 127 & 8 & 8 & 8 \\
    unsigned char & 0 & 255 & 8 & 8 & 8 \\
    \hline
    short & & & & & \\
    short int & \multirow{-2}{*}{-32767} & \multirow{-2}{*}{32767} & \multirow{-2}{*}{16} & \multirow{-2}{*}{16} & \multirow{-2}{*}{16} \\
    \hline
    unsigned short & & & & & \\
    unsigned short int & \multirow{-2}{*}{0} & \multirow{-2}{*}{65535} & \multirow{-2}{*}{16} & \multirow{-2}{*}{16} & \multirow{-2}{*}{16} \\
    \hline
    int & -32767 & 32767 & 32 & 32 & 32 \\
    \hline
    unsigned & & & & & \\
    unsigned int & \multirow{-2}{*}{0} & \multirow{-2}{*}{65535} & \multirow{-2}{*}{32} & \multirow{-2}{*}{32} & \multirow{-2}{*}{32} \\
    \hline
    long & & & & & \\
    long int & \multirow{-2}{*}{-2147483647} & \multirow{-2}{*}{2147483647} & \multirow{-2}{*}{32} & \multirow{-2}{*}{64} & \multirow{-2}{*}{32} \\
    \hline
    unsigned long & & & & & \\
    unsigned long int & \multirow{-2}{*}{0} & \multirow{-2}{*}{4294967295} & \multirow{-2}{*}{32} & \multirow{-2}{*}{64} & \multirow{-2}{*}{32} \\
    \hline
    long long & & & & & \\
    long long int & \multirow{-2}{*}{$-2^{63}-1$} & \multirow{-2}{*}{$2^{63}-1$} & \multirow{-2}{*}{64} & \multirow{-2}{*}{64} & \multirow{-2}{*}{64} \\
    \hline
    unsigned long long & & & & & \\
    unsigned long long int & \multirow{-2}{*}{0} & \multirow{-2}{*}{$2^{64}-1$} & \multirow{-2}{*}{64} & \multirow{-2}{*}{64} & \multirow{-2}{*}{64} \\
    \hline
  \end{tabular}
\end{frame}

\begin{frame}[fragile]{Требования к дробным числовым типам}
  \centering
  \begin{tabular}{|C{8em}|C{8em}|C{8em}|}
    \hline
    \bfseries Тип & \bfseries Минимальное число точно представимых десятичных цифр & \bfseries Максимально представимое число \\
    \hline
    float & 6 & 1E+37 \\
    double & 10 & 1E+37 \\
    long double & 10 & 1E+37 \\
    \hline
  \end{tabular}
\end{frame}

%-------------------------------------------------
\section{Приведение базовых типов}

\begin{frame}[fragile]{Числовые расширения}
  \begin{itemize}
    \item \lstinline{signed char} \rarr \lstinline{int}
    \item \lstinline{unsigned char} \rarr \lstinline{int} или \lstinline{unsigned int}
    \item \lstinline{short} \rarr \lstinline{int}
    \item \lstinline{unsigned short} \rarr \lstinline{int} или \lstinline{unsigned int}
    \item \lstinline{char} -- либо как \lstinline{signed char}, либо как \lstinline{unsigned char}
    \item \lstinline{float} \rarr \lstinline{double}
  \end{itemize}
\end{frame}

\begin{frame}[fragile]{Числовые преобразования}
  \begin{lstlisting}
  int main()
  {
    int a = true; // 1
    double b = true; // 1.0
    float c = b; // 1.0
    unsigned char d = c; // 1
    unsigned int e = -1; // 0xFFFFFFFF
    int f = 1.33; // 1
  }
  \end{lstlisting}
\end{frame}

\begin{frame}[fragile]{Стандартные арифметические преобразования}
  \begin{lstlisting}
  int main()
  {
    int a = 1;
    unsigned b = 2;
    long long c = -3;
    float d = 0.5;
    double e = -1.33;

    auto x = a + b; // unsigned
    auto y = b + c; // long long
    auto z = d + e; // double
    auto u = a + e; // double;
  }
  \end{lstlisting}
\end{frame}

\begin{frame}[fragile]{Явные преобразования}
  \begin{lstlisting}
  int main()
  {
    long long x = -1;
    auto y = static_cast<unsigned>(x) + 2; // unsigned
  }
  \end{lstlisting}
\end{frame}

\begin{frame}[fragile]{Old-style cast}
  \begin{minipage}{.45\textwidth}
    \begin{itemize}
      \item \lstinline{(T) x}
      \item \lstinline{T(x)}
    \end{itemize}
  \end{minipage}\hfill
  \begin{minipage}{.45\textwidth}
    \begin{itemize}
      \item \lstinline{const_cast<T>(x)} \vspace{0.5em}
      \item \lstinline{static_cast<T>(x)}* \vspace{0.5em}
      \item \lstinline{static_cast}* + \lstinline{const_cast} \vspace{0.5em}
      \item \lstinline{reinterpret_cast<T>(x)}
      \item \lstinline{reinterpret_cast} + \lstinline{const_cast}
    \end{itemize}
  \end{minipage}
\end{frame}

\end{document}
