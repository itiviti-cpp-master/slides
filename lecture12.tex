\documentclass[unknownkeysallowed,xcolor=table]{beamer}
 
\usepackage [T2A,T1] {fontenc}
\usepackage[utf8]{inputenc}
\usepackage[english,russian]{babel}
\usepackage{amsmath}
\usepackage{listings}
\usepackage{url}
\usepackage{textcomp}
\usepackage{multirow}

% to not copy line numbers
\usepackage{accsupp}
% to not copy line numbers

% to not copy line numbers
\newcommand{\noncopynumber}[1]{%
  \BeginAccSupp{method=escape,ActualText={}}%
    \color{blue} #1%
  \EndAccSupp{}%
}
% to not copy line numbers

\setbeamertemplate{navigation symbols}{}

\newcommand{\textapprox}{\raisebox{0.5ex}{\texttildelow}}

\colorlet{mygreen}{green!60!blue}
\colorlet{mymauve}{red!60!blue}
\definecolor{pblue}{rgb}{0.1, 0.2, 0.8}

\newcommand{\rarr}{$\rightarrow$}

\lstset{
      basicstyle=\ttfamily\small,
      commentstyle=\color{mygreen},
      keywordstyle=\color{blue},
%      numberstyle=\tiny\color{blue},
      numberstyle=\noncopynumber,
      stringstyle=\color{mymauve},
      numbers=left,
      stepnumber=1,
      columns=fullflexible,
      breaklines=true,
      postbreak=\mbox{\textcolor{red}{\ensuremath{\hookrightarrow}\space}},
      literate={~} {\textapprox}{1},
      language={[11]C++}
}

\makeatletter
\newcommand{\srcmediumsize}{\@setfontsize{\srcmediumsize}{7pt}{7pt}}
\makeatother

\makeatletter
\newcommand{\srcbigsize}{\@setfontsize{\srcbigsize}{8pt}{8pt}}
\makeatother

\makeatletter
\newcommand{\srcsize}{\@setfontsize{\srcsize}{6pt}{6pt}}
\makeatother

\makeatletter
\newcommand{\srcsmallsize}{\@setfontsize{\srcsmallsize}{5pt}{5pt}}
\makeatother

\title[C++]
{Программирование на языке C++}
 
\subtitle{Вводный курс}
 
\author[А.~Б.~Морозов]
{
  \texorpdfstring{Александр Морозов\newline\href{mailto:gelu.speculum@gmail.com}{gelu.speculum@gmail.com}}
  {Александр Морозов}
}
  
\date[ITMO 2020]
{ИТМО, весенний семестр 2020}
 
\logo{%
  \makebox[0.97\paperwidth]{%
    \includegraphics[align=c,width=2cm,keepaspectratio]{itmo_logo.png}
    \hfill
    \includegraphics[align=c,width=1.5cm,keepaspectratio]{itiviti_logo.png}
  }
}

\AtBeginSection[]
{
  \begin{frame}
    \frametitle{Содержание}
    \tableofcontents[currentsection]
  \end{frame}
}

\begin{document}

\frame{\titlepage}

%-------------------------------------------------

\section{Специальные методы классов}

\begin{frame}{Конструктор}
  Конструктор -- это специальная функция-член класса, служащая для инициализации объектов этого класса.

  \vspace{1em}

  Имя функции совпадает с именем класса.

  \vspace{1em}

  При создании объекта класса всегда вызывается один конструктор.

  \vspace{1em}

  Конструктор нельзя вызвать явно.

  \vspace{1em}

  У класса может быть произвольное число конструкторов.
\end{frame}

\begin{frame}[fragile]{Деструктор}
  Деструктор -- это специальная функция-член класса, служащая для выполнения произвольных действий непосредственно перед удалением объекта класса.

  \vspace{1em}

  Обычно используется для освобождения ресурсов, используемых классом, или иной финализации.

  \vspace{1em}

  Имя функции -- \lstinline{~ClassName}.

  \vspace{1em}

  У деструктора не может быть аргументов.

  \vspace{1em}

  При любом удалении объекта класса, его деструктор вызывается.

  \vspace{1em}

  У класса может быть только один деструктор.

  \vspace{1em}

  Деструктор можно вызвать явно.
\end{frame}

\begin{frame}{Особенности конструкторов и деструкторов}
  \begin{itemize}
    \item нет возвращаемого значения \vspace{1em}
    \item нет cv-квалификаторов \vspace{1em}
    \item нет ref-квалификаторов
  \end{itemize}
\end{frame}

\begin{frame}[fragile]{Пример конструкторов и деструктора}
  \begin{lstlisting}[basicstyle=\ttfamily\srcbigsize]
  struct S {
      S(); // 1
      S(int); // 2
      S(char, double); // 3
      S(const S &); // 4
      ~S();
  };

  S get_s();

  int main()
  {
      S s1; // 1st constructor is called
      S s2(10); // 2nd constructor is called
      {
          const S & s_ref = get_s();
          S s3('a', 0); // 3rd constructor is called
          S s4 = get_s(); // probably, 4th constructor is called
                          // then S::~S()
      } // S::~S() for s3, s4 and tmp object is called
  } // S::~S() for s1 and s2 is called
  \end{lstlisting}
\end{frame}

\begin{frame}[fragile]{Список инициализации членов}
  \begin{lstlisting}
  struct S {
      S(int x) : a(x), b(a*2) {}

      int a;
      int b = 10;
  };

  struct SS : S {
      SS() : S(20), c(b) {}

      int c;
  };
  \end{lstlisting}
\end{frame}

\begin{frame}[fragile]{Варианты инициализации в списке}
  \begin{lstlisting}
  struct A {
      A() {}
      int a = 10;
  };
  struct B {
      B(int x) : b(x) {} // direct initialization
      int b;
  };
  template <class... Ts>
  struct C : A, B, Ts... {
      C(int x, int y, const Ts &... ts)
        : B{x}  // list initialization
        , Ts(ts)... // pack expansion
        , c(y) // direct
    {}
    int c;
  };
  \end{lstlisting}
\end{frame}

\begin{frame}{Порядок инициализации объекта класса}
  \begin{enumerate}
    \item базовые классы инициализируются в том порядке, в каком они перечислены в списке наследования \vspace{1em}
    \item нестатические поля класса инициализируются в том порядке, в каком они объявлены в определении класса \vspace{1em}
    \item выполняется тело конструктора
  \end{enumerate}
\end{frame}

\begin{frame}{Порядок уничтожения объекта класса}
  \begin{enumerate}
    \item тело деструктора \vspace{1em}
    \item нестатические поля класса разрушаются в порядке, обратном порядку их объявления в классе \vspace{1em}
    \item для базовых классов деструкторы вызываются в порядке, обратном их порядку в списке наследования
  \end{enumerate}
\end{frame}

\begin{frame}[fragile]{Некорректное завершение выполнения конструктора}
  \begin{lstlisting}
  struct S {
      std::string s;
      std::vector<int> v;
      S * next = nullptr;

      S();
  };

  S::S() : s("Hello, world")
  {
      v.resize(100);
      next = new S(1);
      throw 1;
  }

  S s;
  \end{lstlisting}
\end{frame}

\begin{frame}[fragile]{Делегирование конструкторов}
  \begin{lstlisting}
  struct S {
      S(int, double, char) {...}
      S() : S(10, 0.5, 'a') {...}
  };
  \end{lstlisting}

  \vspace{2em}

  У делегирующего конструктора список инициализации членов должен состоять из одного элемента -- вызова целевого конструктора.
\end{frame}

\begin{frame}[fragile]{Наследование конструкторов}
  \begin{lstlisting}
  struct A {
      A(int, double, char);
  };

  struct B : A {};

  struct C : A {
      using A::A;
  };

  int main() {
      B b(10, 0.5, 'a'); // error
      C c(10, 0.5, '\n'); // OK
  }
  \end{lstlisting}
\end{frame}

\begin{frame}[fragile]{\lstinline{explicit} конструкторы}
  \begin{lstlisting}
  struct A {
      A(int);
  };

  struct B {
      explicit B(int);
  };

  A f(const A &)
  {
      return 10; // OK, A::A(int) is called
  }

  B g(const B &)
  {
      return 10; // error
      return B(10); // OK, B::B(int) is called
  }

  int main()
  {
      f(10); // OK, A::A(int) is called
      g(10); // error
      g(B(10)); // OK B::B(int) is called
  }
  \end{lstlisting}
\end{frame}

\begin{frame}[fragile]{Конструкторы приведения типа}
  Любые не-\lstinline{explicit} конструкторы.

  \vspace{1em}

  Могут участвовать в последовательности неявного приведения типа.

  \begin{lstlisting}
  struct A {
      A(int);
  };

  void f(A);

  f(10);
  \end{lstlisting}
\end{frame}

\begin{frame}[fragile]{Удалённые и автоматически создаваемые методы}
  Некоторые методы могут быть созданы автоматически -- как неявно, так и явно.

  \vspace{0.5em}

  Запретить неявную автоматическую генерацию можно ``удалив'' метод.
  \begin{lstlisting}
  struct A {}; // A::A() is implicitly-defined

  struct B {
      B() = default; // B::B() implicit definition is forced
  };

  struct C {
      C() = delete; // C::C() is not defined
  };
  \end{lstlisting}

  Некоторые методы могут быть удалены неявно.
\end{frame}

\begin{frame}{Методы, создаваемые автоматически}
  \begin{itemize}
    \item конструктор по умолчанию \vspace{1em}
    \item конструктор копирования \vspace{1em}
    \item конструктор перемещения \vspace{1em}
    \item деструктор \vspace{1em}
    \item операторы присваивания
  \end{itemize}
\end{frame}

\begin{frame}[fragile]{Конструктор по умолчанию}
  \begin{lstlisting}
  struct S {
      S(); // default constructor
  };

  S s; // default constructor is called

  struct A {
      A();
  };
  struct B : A {
      B();
  };

  B b; // A::A() and B::B() are called
  \end{lstlisting}
\end{frame}

\begin{frame}[fragile]{Автоматическое создание конструктора по умолчанию}
  Если у класса нет явно определенных конструкторов, то конструктор по умолчанию создаётся автоматически и эквивалентен явно определенному конструктору без списка инициализации и с пустым телом:
  \begin{lstlisting}
  struct A {
  };

  struct B {
    B() {}
  };
  \end{lstlisting}
\end{frame}

\begin{frame}{Удалённый конструктор по умолчанию}
  \begin{itemize}
    \item наличие ссылочного поля без инициализатора \vspace{0.5em}
    \item наличие не статического константного поля без явно определённого конструктора по умолчанию или инициализатора \vspace{0.5em}
    \item наличие не статического поля без инициализатора, имеющего удалённый либо недоступный конструктор по умолчанию \vspace{0.5em}
    \item наличие прямого или виртуального предка с удалённым либо недоступным конструктором по умолчанию \vspace{0.5em}
    \item наличие прямого или виртуального предка с удалённым либо недоступным деструктором
  \end{itemize}
\end{frame}

\begin{frame}{Удалённый деструктор}
  \begin{lstlisting}
  struct S {
      ~S() = delete;
  };

  static S ss; // error
  S s; // error
  S * ps = new S; // error
  S * ps = new (ptr) S(); // OK
  \end{lstlisting}
\end{frame}

%-------------------------------------------------

\section{Копирование и перемещение}

\begin{frame}{Конструктор копирования}
  Это не шаблонный конструктор, принимающий первым аргументом lvalue ссылку на объект того же типа, что и класс конструктора, который может быть вызван с одним аргументом.

  \vspace{1em}

  Конструктор копирования вызывается всегда, когда объект класса инициализируется из lvalue выражения того же типа (класса).
\end{frame}

\begin{frame}[fragile]{Пример конструкторов копирования}
  \begin{lstlisting}
  struct S {
      S(int);
      S(const S &); // copy constructor 1
      S(S &); // copy constructor 2
  };

  S f(S s) {
      return s; // copy constructor ? is called
  }

  const S s1(10);
  S s2 = s1, s3(s1); // copy constructor ? is called
  f(S(10)); // copy constructor ? is called
  \end{lstlisting}
\end{frame}

\begin{frame}{Автоматическое создание конструктора копирования}
  \begin{itemize}
    \item \lstinline{C(const C &)}, если
      \begin{itemize}
        \item все прямые или виртуальные предки имеют конструктор копирования \lstinline{B(const B &)}
        \item все не статические поля имеют конструктор копирования \lstinline{M(const M &)}
      \end{itemize}
    \item \lstinline{C(C &)} в противном случае
  \end{itemize}

  \vspace{1em}

  Такой конструктор выполняет копирование всех подобъектов создаваемого объекта, в обычном порядке инициализации.
\end{frame}

\begin{frame}{Удалённый конструктор копирования}
  \begin{itemize}
    \item наличие не копируемого не статического поля \vspace{0.5em}
    \item наличие не копируемого прямого или виртуального предка \vspace{0.5em}
    \item наличие прямого или виртуального предка с удалённым или недоступным деструктором \vspace{0.5em}
    \item наличие rvalue ссылочного не статического поля
  \end{itemize}

  \vspace{1em}

  Наличие пользовательского конструктора перемещения или перемещающего оператора присваивания $\eq$ отсутствие автоматически созданного конструктора копирования.
\end{frame}

\begin{frame}[fragile]{Копирующий оператор присваивания}
  Это не шаблонный оператор присваивания, принимающий ровно один аргумент, имеющий тип \lstinline{T}, \lstinline{T&} или \lstinline{const T&} (если \lstinline{T} -- это данный класс).

  \vspace{1em}

  \begin{lstlisting}
  struct S {
      S(const S &);
      S & operator = (const S &);
  };

  S s1, s2 = s1; // copy constructor is called
  s1 = s2; // copy assignment is called
  \end{lstlisting}

  \vspace{1em}

  Предполагаемое поведение такого оператора -- копирование значения одного объекта данного класса в другой объект этого же класса.
\end{frame}

\begin{frame}{Автоматическое создание копирующего оператора присваивания}
  \begin{itemize}
    \item \lstinline{C & operator = (const C &)}, если
      \begin{itemize}
        \item все прямые или виртуальные предки имеют копирующий оператор присваивания \lstinline{B & operator = (const B &)}
        \item все не статические поля имеют копирующий оператор присваивания \lstinline{M & operator = (const M &)}
      \end{itemize}
    \item \lstinline{C & operator = (C &)} в противном случае
  \end{itemize}

  \vspace{1em}

  Такой оператор выполняет копирующее присваивание всех подобъектов левого объекта из соответствующих подобъектов правого, в обычном порядке инициализации.
\end{frame}

\begin{frame}{Удалённый копирующий оператор присваивания}
  \begin{itemize}
    \item наличие не присваиваемого не статического поля \vspace{0.5em}
    \item наличие не присваиваемого прямого или виртуального предка \vspace{0.5em}
    \item наличие ссылочного не статического поля \vspace{0.5em}
    \item наличие константного не статического поля, чей тип не является классом
  \end{itemize}

  \vspace{1em}

  Наличие пользовательского конструктора перемещения или перемещающего оператора присваивания $\eq$ отсутствие автоматически созданного копирующего оператора
  присваивания.
\end{frame}

\begin{frame}{Конструктор перемещения}
  Это не шаблонный конструктор, принимающий первым аргументом rvalue ссылку на объект того же типа, что и класс конструктора, может быть вызван с одним аргументом.

  \vspace{1em}

  Конструктор перемещения вызывается всегда, когда объект класса инициализируется из xvalue выражения того же типа (класса).

  \vspace{1em}

  Предназначение перемещающего конструктора -- забрать содержимое у аргумента - временного объекта и, по возможности, избежать затрат на копирование этого содержимого. 

  При этом аргумент должен остаться в валидном, но не обязательно определённом, состоянии.
\end{frame}

\begin{frame}[fragile]{Пример конструкторов копирования и перемещения}
  \begin{lstlisting}
  struct S {
      S(const S &); // copy constructor
      S(S &&); // move constructor
  };

  void f(S s);

  int main() {
    S s1;
    f(s1); // copy constructor is called
    f(S{}); // move constructor is called
    f(std::move(s1)); // move constructor is called
  }
  \end{lstlisting}
\end{frame}

\begin{frame}{Автоматическое создание конструктора перемещения}
  Создаётся, если
  \begin{itemize}
    \item нет ни одного пользовательского конструктора перемещения \vspace{0.5em}
    \item нет ни одного пользовательского конструктора копирования \vspace{0.5em}
    \item нет ни одного пользовательского оператора присваивания (копирующего или перемещающего) \vspace{0.5em}
    \item нет пользовательского деструктора
  \end{itemize}

  \vspace{1em}

  Такой конструктор выполняет инициализацию всех подобъектов создаваемого объекта, в обычном порядке инициализации, из xvalue выражения соответствующего подообъекта
  объекта-донора.
\end{frame}

\begin{frame}[fragile]{Удалённый конструктор перемещения}
  \begin{itemize}
    \item наличие не статического поля, для которого нельзя вызвать конструктор перемещения \vspace{0.5em}
    \item наличие прямого или виртуального предка, для которого нельзя вызвать конструктор перемещения \vspace{0.5em}
    \item наличие прямого или виртуального предка, для которого нельзя вызвать деструктор
  \end{itemize}

  \vspace{1em}

  Удалённый конструктор перемещения не участвует в разрешении перегрузки.
  \begin{lstlisting}
  struct S {
      S(const S &);
      S(S &&) = delete;
  };

  S get_s();

  S s = get_s(); // OK
  \end{lstlisting}
\end{frame}

\begin{frame}[fragile]{Перемещающий оператор присваивания}
  Это не шаблонный оператор присваивания, принимающий ровно один аргумент, имеющий тип \lstinline{T&&} или \lstinline{const T&&}.

  \vspace{1em}

  \begin{lstlisting}
  struct S {
      S & operator = (S &&);
  };

  S s1, s2;
  s2 = S(); // move assignment is called
  s1 = std::move(s2); // move assignment is called
  \end{lstlisting}
  \vspace{1em}

  Предполагаемое поведение перемещающего оператор присваивания -- перемещение значения одного объекта данного класса в другой объект данного класса.
\end{frame}

\begin{frame}{Автоматическое создание перемещающего оператора присваивания}
  Создаётся, если
  \begin{itemize}
    \item нет ни одного пользовательского конструктора перемещения \vspace{0.5em}
    \item нет ни одного пользовательского конструктора копирования \vspace{0.5em}
    \item нет ни одного пользовательского копирующего оператора присваивания \vspace{0.5em}
    \item нет пользовательского деструктора
  \end{itemize}

  \vspace{1em}

  Такой оператор выполняет перемещающее присваивание всех подобъектов левого объекта из соответствующих подобъектов правого, в обычном порядке инициализации.
\end{frame}

\begin{frame}[fragile]{Удалённый перемещающий оператор присваивания}
  \begin{itemize}
    \item наличие не статического константного поля \vspace{0.5em}
    \item наличие не статического ссылочного поля \vspace{0.5em}
    \item наличие не статического поля, для которого нельзя вызвать перемещающее присваивание \vspace{0.5em}
    \item наличие прямого или виртуального предка, для которого нельзя вызвать перемещающее присваивание
  \end{itemize}

  \vspace{1em}

  Удалённый конструктор перемещения не участвует в разрешении перегрузки.
  \begin{lstlisting}
  struct S {
      S(const S &);
      S(S &&) = delete;
  };

  S get_s();

  S s = get_s(); // OK
  \end{lstlisting}
\end{frame}

%-------------------------------------------------

\section{Copy elision}

\begin{frame}[fragile]{Материализация}
  \textbf{prvalue} \rarr \textbf{xvalue}

  \vspace{2em}

  \begin{lstlisting}
  T f();

  int a = 1 + 2;
  int && b = 1 + 2;
  T x = f();
  const M & m = f().member;
  \end{lstlisting}
\end{frame}

\begin{frame}[fragile]{Copy elision}
  Отсутствие материализации: \vspace{1em}
  \begin{itemize}
    \item инструкция \lstinline{return}, когда её операнд -- prvalue того же типа, что и тип возвращаемого значения функции \vspace{1em}
    \item инициализация переменной, когда выражение инициализации -- prvalue того же типа
  \end{itemize}

  \vspace{1em}
  \begin{lstlisting}
  T f() {
    return T();
  }
  T x = T(T(T(f()))); // only default constructor of T is called
  \end{lstlisting}
\end{frame}

\begin{frame}[fragile]{NRVO}
  Named return value optimization: если в инструкции \lstinline{return} операнд обозначает локальную переменную, которая:
  \begin{itemize}
    \item не является параметром функции
    \item объект которой не является \lstinline{volatile}
    \item объект которой имеет автоматический тип размещения
    \item тип объекта которой совпадает с типом возвращаемого значения функции
  \end{itemize}
  То компилятор \emph{может} не генерировать код вызова конструктора копирования/перемещения, как если бы это была copy elision.

  \begin{lstlisting}
  T f() {
    T t;
    return t;
  }
  T x = f(); // only default constructor of T is called
  \end{lstlisting}
\end{frame}

\end{document}
