\documentclass[unknownkeysallowed,xcolor=table]{beamer}
 
\usepackage[T2A,T1]{fontenc}
\usepackage[utf8]{inputenc}
\usepackage[english,russian]{babel}
\usepackage{listings}
\usepackage{amsmath}
\usepackage{url}
\usepackage{textcomp}
\usepackage{multirow}
\usepackage{tikz}

\setbeamertemplate{navigation symbols}{}

\newcommand{\textapprox}{\raisebox{0.5ex}{\texttildelow}}

\newcommand{\rarr}{$\rightarrow$}
 
\colorlet{mygreen}{green!60!blue}
\colorlet{mymauve}{red!60!blue}
\definecolor{light-gray}{gray}{0.9}

\lstset{
      basicstyle=\ttfamily\small,
      commentstyle=\color{mygreen},
      keywordstyle=\color{blue},
      numberstyle=\tiny\color{blue},
      stringstyle=\color{mymauve},
      numbers=left,
      stepnumber=1,
      columns=fullflexible,
      breaklines=true,
      postbreak=\mbox{\textcolor{red}{\ensuremath{\hookrightarrow}\space}},
      literate={~} {\textapprox}{1},
      language={[11]C++}
}

\lstnewenvironment{cmdline}
  {\lstset{
      basicstyle=\ttfamily\scriptsize,
      keywordstyle=\color{blue},
      backgroundcolor=\color{light-gray},
      language={bash}
  }}
  {}

\lstnewenvironment{cmdlinelarge}
  {\lstset{
      basicstyle=\ttfamily\small,
      keywordstyle=\color{blue},
      backgroundcolor=\color{light-gray},
      language={bash}
  }}
  {}

\makeatletter
\newcommand{\srcmediumsize}{\@setfontsize{\srcmediumsize}{7pt}{7pt}}
\makeatother

\makeatletter
\newcommand{\srcbigsize}{\@setfontsize{\srcbigsize}{8pt}{8pt}}
\makeatother

\makeatletter
\newcommand{\srcsize}{\@setfontsize{\srcsize}{6pt}{6pt}}
\makeatother

\makeatletter
\newcommand{\srcsmallsize}{\@setfontsize{\srcsmallsize}{5pt}{5pt}}
\makeatother

\title[C++]
{Программирование на языке C++}
 
\subtitle{Вводный курс}
 
\author[А.~Б.~Морозов]
{
  \texorpdfstring{Александр Морозов\newline\href{mailto:gelu.speculum@gmail.com}{gelu.speculum@gmail.com}}
  {Александр Морозов}
}
  
\date[ITMO 2020]
{ИТМО, весенний семестр 2020}
 
\logo{%
  \makebox[0.97\paperwidth]{%
    \includegraphics[align=c,width=2cm,keepaspectratio]{itmo_logo.png}
    \hfill
    \includegraphics[align=c,width=1.5cm,keepaspectratio]{itiviti_logo.png}
  }
}

\AtBeginSection[]
{
  \begin{frame}
    \frametitle{Содержание}
    \tableofcontents[currentsection]
  \end{frame}
}

\begin{document}
 
\frame{\titlepage}

%-------------------------------------------------
\section{Приведение базовых типов}

\begin{frame}[fragile]{Числовые расширения}
  \[
    T1 \triangleright T2 \implies \mathbf{value}(1) \equiv \mathbf{value}(2)
  \]
  
  \vspace{1em}

  \begin{itemize}
    \item \lstinline{signed char}, \lstinline{short} \rarr \lstinline{int} \vspace{1em}
    \item \lstinline{unsigned char}, \lstinline{unsigned short} \rarr \lstinline{int} или \lstinline{unsigned int} \vspace{1em}
    \item \lstinline{char} \rarr \lstinline{int} или \lstinline{unsigned int} \vspace{1em}
    \item \lstinline{float} \rarr \lstinline{double}
  \end{itemize}
\end{frame}

\begin{frame}[fragile]{Интегральные преобразования}
  \[
    \forall T1, T2: T1 \in \mathbb{I}, T2 \in \mathbb{I}, T1 \mapsto T2
  \]

  \vspace{0.5em}

  \begin{itemize}
    \item $T2 \in \mathbb{U}, \mathbf{sizeof}(T1) > \mathbf{sizeof}(T2) \implies$ результат -- младшие биты исходного представления \vspace{0.5em}
    \item $T2 \in \mathbb{U}, T1 \in \mathbb{S}, \mathbf{sizeof}(T1) \leq \mathbf{sizeof}(T2) \implies$ результат -- знаковое дополнение \vspace{0.5em}
    \item $T2 \in \mathbb{U}, T1 \in \mathbb{U}, \mathbf{sizeof}(T1) \leq \mathbf{sizeof}(T2) \implies$ результат -- дополнение нулями \vspace{0.5em}
    \item $T2 \in \mathbb{S}$ и исходное значение помещается в $T2 \implies$ значение не меняется \vspace{0.5em}
    \item $T2 \in \mathbb{S}$ и исходное значение \textbf{не} помещается в $T2 \implies$ implementation defined
  \end{itemize}
\end{frame}

\begin{frame}[fragile]{Дробные преобразования}
  \[
    \forall T1, T2: T1 \in \mathbb{F}, T2 \in \mathbb{F}, T1 \mapsto T2
  \]

  \vspace{1em}

  \begin{itemize}
    \item если исходное значение \emph{точно} представимо в целевом типе, оно не меняется \vspace{1em}
    \item если исходное значение попадает между двумя корректными значениями целевого типа, то результат -- одно из этих значений (implementation defined) \vspace{1em}
    \item иначе -- \textbf{UB}
  \end{itemize}
\end{frame}

\begin{frame}[fragile]{Преобразования между дробными и интегральными типами}
  \begin{itemize}
    \item дробное $\mapsto$ целое путём отбрасывания дробной части, если целая часть помещается в целевой тип, иначе -- \textbf{UB} \vspace{2em}
    \item целое $\mapsto$ дробное, возможно, с округлением (implementation defined); если исходное значение слишком велико -- \textbf{UB}
  \end{itemize}
\end{frame}

\begin{frame}[fragile]{Преобразования с \lstinline{bool}}
  \[
    \forall T: T \in \mathbb{I} \lor \mathbb{F} \lor \mathbb{E} \lor \mathbb{P}, T \mapsto \mathtt{bool}
  \]

  \[
    \forall T: T \in \mathbb{I} \lor \mathbb{F}, \mathtt{bool} \mapsto T
  \]

  \vspace{1em}

  Преобразования к \lstinline{bool}
  \begin{itemize}
    \item $value \neq 0 \implies$ \lstinline{true} \vspace{0.5em}
    \item $value = 0 \implies$ \lstinline{false}
  \end{itemize}

  \vspace{1em}
  Преобразования из \lstinline{bool}
  \begin{itemize}
    \item \lstinline{false} \rarr \lstinline{0} \vspace{0.5em}
    \item \lstinline{true} \rarr \lstinline{1}
  \end{itemize}
\end{frame}

\begin{frame}[fragile]{Стандартные арифметические преобразования}
  \begin{enumerate}
    \item расширение операндов интегрального типа \vspace{0.5em}
    \item приведение к общему типу
      \begin{itemize}
        \item если один операнд \lstinline{long double}, второй приводится к \lstinline{long double}
        \item если один операнд \lstinline{double}, второй приводится к \lstinline{double}
        \item если один операнд \lstinline{float}, второй приводится к \lstinline{float}
        \item если знаковость одинакова, то подтягивается ранг
        \item если ранг беззнакового $\geq$ ранга знакового, то знаковый \rarr тип беззнакового
        \item если тип знакового может представить все значения типа беззнакового, то беззнаковый \rarr тип знакового
        \item иначе оба \rarr беззнаковый вариант типа знакового
      \end{itemize}
  \end{enumerate}

  \vspace{0.5em}

  Ранг: \lstinline{bool} < \lstinline{signed char} < \lstinline{short} < \lstinline{int} < \lstinline{long} < \lstinline{long long}
\end{frame}

\begin{frame}[fragile]{Унарные плюс и минус}
  \begin{itemize}
    \item \lstinline{+x} -- числовое расширение \vspace{2em}
    \item \lstinline{-x}
      \begin{itemize}
        \item числовое расширение \vspace{1em}
        \item для беззнаковых целых -- $-x = 2^n-x$ \vspace{1em}
        \item для остальных -- инвертирование знака
      \end{itemize}
  \end{itemize}
\end{frame}

\begin{frame}{Целочисленное переполнение}
  \begin{itemize}
    \item беззнаковая целочисленная арифметика -- по модулю $2^n$ \vspace{2em}
    \item знаковое переполнение -- \textbf{UB}
  \end{itemize}
\end{frame}

\begin{frame}[fragile]{Old-style cast}
  \centering
  \begin{tabular}{|c|c|}
    \lstinline{(T) x} & \lstinline{const_cast<T>(x)} \\
    \lstinline{(T) x} & \lstinline{static_cast<T>(x)}* \\
    \lstinline{(T) x} & \lstinline{static_cast}* + \lstinline{const_cast} \\
    \lstinline{(T) x} & \lstinline{reinterprer_cast<T>(x)} \\
    \lstinline{(T) x} & \lstinline{reinterprer_cast} + \lstinline{const_cast} \\
  \end{tabular}
\end{frame}

%-------------------------------------------------
\section{Объекты}

\begin{frame}{Объекты}
  \begin{itemize}
    \item размер (\lstlisting{sizeof})
    \item выравнивание (\lstlisting{alignof})
  \end{itemize}
\end{frame}

\end{document}
